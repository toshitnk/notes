\section{1次元量子系}
\subsection{自由粒子}
一次元自由粒子のSchr\"{o}dinger方程式は
\begin{align}
    -\frac{\hbar^2}{2m}\drvn{2}{}{x}\psi(x)=E\psi(x)
\end{align}
である.変形して,
\begin{align}
    \drvn{2}{}{x}\psi(x)=-k^2\psi(x)\quad;\quad k\coloneqq\frac{\sqrt{2mE}}{\hbar}>0
\end{align}
である.この一般解は任意定数$C_1,C_2\in\C$または$A\in\C,\delta\in\R$を用いて
\begin{align}
    \psi(x)=C_1e^{ikx}+C_2e^{-ikx}=A\sin\qty(kx+\delta)\label{free_sol}
\end{align}
と書ける.
\footnote{以前\url{https://www.mns.kyutech.ac.jp/~okamoto/education/quantum/quantum-1dim100802.pdf}を参照して,自由粒子では重ね合わせの状態は取らないなど議論していたが,それは誤りであると栗本先生から指摘を受けた.私が誤っていた点は,演算子が可換ならば同時固有状態を取ることが出来るが,すべての状態が同時固有状態なわけでないということ.固有状態の線形結合は一般に固有状態でないからそれは当たり前.}
\begin{comment}
\begin{cl}\label{thm:eikx}
自由粒子では,この重ね合わせ状態は許されず,
\begin{align}
    \psi(x)=Ce^{ikx}\quad\text{xor}\quad\psi(x)=Ce^{-ikx}
\end{align}
のどちらかになる.%\footnote{この議論は\url{https://www.mns.kyutech.ac.jp/~okamoto/education/quantum/quantum-1dim100802.pdf}による.}
\end{cl}
\begin{pr}
自由粒子ではHamiltonianと運動量が可換なので,エネルギー固有状態は同時に運動量の固有状態でなければならない.\eqref{free_sol}のとき,
\begin{align}
    -i\hbar\drv{\psi(x)}{x}&=\hbar k \qty(C_1e^{ikx}-C_2e^{ikx})
\end{align}
となるので,運動量の固有状態になるのは$C_1=0\ \text{xor}\ C_2=0$のときである.
\end{pr}

\subsection*{変な境界条件の入れ方}
この規格化には
\begin{enumerate}
    \item Dirac deltaを用いる方法
    \item fixed boundary condition $\psi(0)=\psi(L)=0$を用いる方法
    \item periodic boundary condition (PBC) $\psi(x)=\psi(x+L)$を用いる方法
\end{enumerate}
を用いるのが通常であるが,逆張りオタクなので$\psi(x+L)=e^{i\theta}\psi(x)\ (0\leq\theta<2\pi)$と入れてみる.
\end{comment}

これは,無限に高い井戸型ポテンシャル
\begin{align}
    V(x)=\begin{cases}
    0\quad&(0\leq x\leq L)\\
    \infty\quad&(x\leq0,\ L\leq x)
    \end{cases}
\end{align}
で規格化したのち,井戸の幅を無限大にするというのが一つの方法である.\footnote{もう一つ,Dirac deltaを使う規格化がある.}この方法を用いるとき,ナイーブな差異として境界条件の取り方があるが,意外と面白いので紹介する.
\subsubsection*{Dirichlet型境界条件}
まず,一番簡単なのは「境界で波動関数がゼロ」i.e.,$\psi(0)=\psi(L)=0$とすることである.これをDirichlet型境界条件\footnote{これはPDEの言葉だろうか?例えばLaplacianはHermitianであってほしいわけだが,そうであるためには表面項が消える必要があって,その時にDerichletを入れたりする.他に,微分が境界で消えるNeumannであったりこの後のPBCも表面項を消す境界条件になっていたりする.}という.このとき,Eq\eqref{free_sol}のsineの解を採用すると簡単で,
\begin{align}
    A\sin\delta=A\sin(kL+\delta)=0
\end{align}
なので,$\delta=0,\ kL=n\pi\  (n\in\Z_{>0})$となる.一つの固有状態を規格化しておくと,
\begin{align}
    \int_{0}^{L}\diff x\abs{A}^2\sin^2(kx)=1
\end{align}
の条件から,$\abs{A}=\sqrt{2/L}$と決まり,固有関数は
\begin{align}
    \psi(x)=\sqrt{\frac{2}{L}}\sin\qty(\frac{n\pi}{L}x)\quad;\quad n\in\Z_{>0}
\end{align}
となる.

ところで,これは何の固有状態であるかというと,エネルギーの固有状態である.エネルギー,i.e., Hamiltonian, の構成上,これは運動量の固有状態になっていてほしいが,そうはなっていない.\footnote{このことは谷村先生に2021年03月02日の集中講義で質問し答えていただいた.}実際,
\begin{align}
    -i\hbar\pdv{}{x}\sqrt{\frac{2}{L}}\sin\qty(\frac{n\pi}{L}x)=-i\hbar\sqrt{\frac{2}{L}}\frac{n\pi}{L}\cos\qty(\frac{n\pi}{L}x)
\end{align}
とsineがcosineになっており,固有関数でない.Hamiltonianでは二回微分することでsineがsineに戻っていた.



2状態$\psi,\phi$を考えると,
\begin{align}
    (\phi,P\psi)&=\int_{0}^{L}\diff x\phi^{*}(x)\qty(-i\hbar\pdv{}{x})f(x)\\
    &=\qty[\pdv{}{x}\phi^{*}(x)\qty(-i\hbar\pdv{}{x})\psi(x)]_{0}^{L}+\int_{0}^{L}\diff x\qty(-i\hbar\pdv{}{x}\phi(x))^{*}\psi(x)\\
    &=(P\phi,\psi)
\end{align}
となる.これは,運動量はHermitianということ.
ところが,$P$と$P^{\dagger}$\footnote{関数$f,g$と演算子$A$があって,内積を$(f,Ag)=(Bf,g)$とする演算子$B$をもって$A^{\dagger}\coloneqq B$と定め,$A$のHermitian conjugateという.}の定義域を見てみると,$\mathrm{dom}P=\qty{\psi\colon\R\to\C\mid \psi(0)=\psi(L)=0}$なのに対し,$\mathrm{dom}P^{\dagger}=\qty{\phi\colon\R\to\C\mid \text{特に制限なし}}$である.\footnote{これは部分積分した第一項を見れば,$\psi$の条件で表面項が消えてくれるので$\phi$に制限を書ける必要がないことが分かる.}

すなわちDirichlet境界条件の下で運動量演算子がHermitianだがself adjointでない\footnote{最近\url{https://arxiv.org/abs/quant-ph/0103153}が話題になっていた気がする.}ということである.\footnote{Hermitianとは$P=P^{\dagger}$のことで,self adjointはそれに加え両者の定義域が一致していることまで要求する.self adjoint operatorはHermitian operatorより狭いクラスということ.self adjoint\ $\subset$\ Hermitianみたいな感じ.物理量はself adjointのクラス.}
\subsubsection*{周期的境界条件}
別の境界条件の取り方として,周期的境界条件(Periodic boundary condition; PBC)$\psi(x)=\psi(x+L)$がある.Eq\eqref{free_sol}でexponentialのほうを使うと$kL=2n\pi(n\in\Z_{\geq0})$となる.先ほどと違い等号がつくのは,exponentialでは引数が$0$でもnull stateにならないから.また,ground state以外では縮退していて,固有状態を全て別々に扱うことにすると,負の整数も許して勘定することが出来る.すなわち固有状態$\langle\ldots,e^{2i\pi x/L\cdot(-1)},1,e^{2i\pi x/L\cdot 1},\ldots\rangle$たちが全状態空間を張るということ.具体的な規格化などは次にやる.

この境界条件の下では上に列挙したエネルギー固有状態たちは運動量の同次固有状態になり,運動量はちゃんとself adjointである.
\subsubsection*{Twisted boundary condition}
もう少し変な境界条件の取り方もできて,量子力学の基本的要請からphaseの違いは同一視できるので「一周して同じ状態」という境界条件をPBCより一般に$\psi(x+L)=e^{i\theta}\psi(x)$といれることが出来る.これをtwisted boundary conditionという.

境界条件より,
\begin{align}
    kL=\theta+2\pi n\quad n\in\Z
\end{align}
となる.,正負をまとめることで$k$および$n$は負の値も取ることに注意.

$k$の定義から,エネルギー固有値は
\begin{align}
    E_n=\frac{\hbar^2}{2mL^2}\qty(\theta+2\pi n)^2
\end{align}
となり,固有状態は
\begin{align}
    \psi_n(x)=\frac{1}{\sqrt{L}}\exp\qty(i\frac{\theta+2\pi n}{L}x)
\end{align}
となる.

ここで,
\begin{itemize}
\item $\theta=0$のとき,i.e.,通常のPBCのとき,
\begin{align}
    E_n=\frac{2\pi^2\hbar^2}{mL^2}n^2
\end{align}
となる.spectrumは$n\neq0$のとき正負の組で二重に縮退するが,ground stateは縮退しない.

\item $\theta=\pi$のとき,
\begin{align}
    E_n=\frac{2\pi^2\hbar^2}{mL^2}\qty(n+\frac{1}{2})^2
\end{align}
で,すべての$n\in\Z$において縮退する.

\item $\theta\neq0,\pi$のとき,すべての$n\in\Z$において縮退しない.
\end{itemize}

さて,$\theta$の制限を外して,$\theta:0\to2\pi$に連続的に(断熱的に)動かすことを考える.すると,$\ket{n}$だった状態が$\theta=2\pi$をまたいだ瞬間$\ket{n+1}$に移ってしまう.\footnote{これは境界を同一視した,i.e.,一次元円周$S^1$のtopologyを持たせた,ことで空間が単連結でなくなったことによるらしい.}これはspectral flowと呼ばれるらしく,非常に面白い.
