\begin{abstract}
		量子力学では,\sa が物理量に対応している.
		また,運動量やHamiltonianといった物理量は
		空間並進や時間発展などの操作を与えるunitary作用素を与える.
		物理では,作用素の定義域を気にすることは少ないが,
		数学的には定義域が重要で,``定義できるところで定義する''ことが大切になる.
		定義域を直接議論することはむずかしいが,Stoneの定理により\onepara から定義域を含めた\sa を構成することができる\cite{Hall2013}.

		なお,本稿の内容はspm22ndの特別講演で教わった.
\end{abstract}
\section{Stoneの定理}
\begin{defn}[\onepara]
		$\Hilb$上の\onepara とは,$t\in\R$でパラメトライズされたunitary作用素の族$\qty{U(t)}_{t\in\R}$で,
		\begin{itemize}
				\item $U(0) = 1_{\Hilb}$
				\item $U(s)U(t) = U(s+t) $
				\item 任意の$\psi\in\Hilb$に対し,$\|\lim_{t\to0}U(t)\psi\|=\|\psi\| $
		\end{itemize}
		を満たすものをいう.
\end{defn}
\begin{thm}[Stone's theorem]
	$\qty{U(t)}_{t\in\R}$を$\Hilb$上の\onepara とする.
	$\psi\in\Hilb$に対し,
	\begin{align}
			\Dom A &\coloneqq \qty{\psi\in\Hilb \mid \lim_{t\to0}\frac{\|U(t)\psi - \psi \|}{it} = 0}, \\
			A\psi &\coloneqq \lim_{t\to0}\frac{U(t)\psi - \psi}{it}
	\end{align}
	と定めると,組$(A, \Dom A)$は$\Hilb$上denseに定義された\sa になる.

	この形から,\sa に対する\onepara を$\qty{e^{itA}}_{t\in\R}$と書く.
\end{thm}

\section{位置}
以下,$\Hilb =\Lsq$として考える.

\onepara を
\begin{align}
		\qty(e^{itX}\psi)(x) \coloneqq e^{itx}\psi(x)
\end{align}
と定める.これは\onepara である.
これにより
\begin{align}
		X\psi(x) &\coloneqq \lim_{t\to0}\frac{e^{itx}\psi(x) - \psi(x)}{it}\\
		  &= \left.\frac{\psi(x)}{i}\dv{e^{itx}}{t}\right|_{t=0}\\
		  &= x\psi(x)
\end{align}
と定めた$X\sim x$は\sa である.

これが位置演算子である.

\section{運動量}
\onepara を
\begin{align}
		\qty(e^{itP}\psi(x))(x) &\coloneqq \psi(x + \hbar t)
\end{align}
と定める.
これは\onepara である.

これにより,
\begin{align}
		P\psi(x) &= \lim_{t\to 0}\frac{\psi(x+\hbar t)-\psi(x)}{it}\\
		  &= -i\hbar\dv{\psi(x)}{x}
\end{align}
と定めた$P \sim -i\hbar\dv*{}{x}$は運動量演算子である.
%\footnote{Stone - von Neumannの定理というのがあって,CCRを満たすものは一意であることが言える.}
