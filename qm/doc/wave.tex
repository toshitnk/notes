\section{波動力学}
\subsection{基底による表示}
状態$\ket{\alpha}\in\h$は位置の固有状態の線形結合として
\begin{align}
    \ket{\alpha}=\int\diff x^{\prime}\ket{x^{\prime}}\braket{x^{\prime}}{\alpha}
\end{align}
と展開できる.この展開係数$\psi(x,t)\coloneqq\braket{x^{\prime}}{\alpha}$を波動関数という.

この位置のように,任意の状態を線形結合により表せる物理量には例えば運動量$P$がある.運動量の固有状態の組を$\qty{\ket{p}}$とする.すなわち,$P\ket{p}=p\ket{p}$.これを位置基底により表現すると
\begin{align}
    \bra{x}P\ket{p}&=-i\hbar\pdv{}{x}\braket{x}{p}=p\braket{x}{p}\\
    \pdv{}{x}\braket{x}{p}&=\frac{ip}{\hbar}\braket{x}{p}\\
    \pdv{}{x}\psi_p(x,t)&=\frac{ip}{\hbar}\psi_p(x,t)
\end{align}
の微分方程式を満たすことが分かる.複素定数$N$としてこの解を
\begin{align}
    \psi_p(x,t)=N\exp\qty(\frac{ip}{\hbar}x)
\end{align}
とかく.この定数は$\ket{x}$の規格化によって決定できて,
\begin{align}
    \braket{x^{\prime}}{x}&=\int\diff p\braket{x^{\prime}}{p}\braket{p}{x}\\
    &=\int\diff p |N|^{2}\exp\qty(\frac{ip\qty(x-x^{\prime})}{\hbar})\label{N_nomalization}\\
    &=\delta\qty(x-x^{\prime})\\
    &=\frac{1}{2\pi\hbar}\int\diff p \exp\qty(\frac{ip\qty(x-x^{\prime})}{\hbar})\label{delta_fourier}
\end{align}
となる.\eqref{delta_fourier}ではDirac deltaのFourier変換を使った.ここで,\eqref{N_nomalization}と\eqref{delta_fourier}を比較すると
\begin{align}
    |N|=\frac{1}{\sqrt{2\pi\hbar}}
\end{align}
として決定できる.
結局
\begin{align}
    \psi_p(x,t)=\braket{x}{p}=\frac{1}{\sqrt{2\pi\hbar}}\exp\qty(\dfrac{ipx}{\hbar})
\end{align}
となる.また,このHermitian conjugateをとると
\begin{align}
    \phi_x(p)\coloneqq\braket{p}{x}=\frac{1}{\sqrt{2\pi\hbar}}\exp\qty(\dfrac{-ipx}{\hbar})
\end{align}

また,これらより,任意の状態$\ket{\alpha}$に対して
\begin{align}
    \psi_{\alpha}(x,t)&=\braket{x}{\alpha}=\int\diff p \braket{x}{p}\braket{p}{\alpha}\\
    &=\frac{1}{\sqrt{2\pi\hbar}}\int\diff p\phi_{\alpha}(p,t)\exp\qty(\frac{ipx}{\hbar})
\end{align}
および,
\begin{align}
    \phi_{\alpha}(p,t)&=\braket{p}{\alpha}=\int\diff x \braket{p}{x}\braket{x}{\alpha}\\
    &=\frac{1}{\sqrt{2\pi\hbar}}\int\diff x\psi_{\alpha}(x,t)\exp\qty(\frac{-ipx}{\hbar})
\end{align}