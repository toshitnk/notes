\documentclass[dvipdfmx,a4paper, lang=english]{jsarticle}
\usepackage[T1]{fontenc}
\usepackage{lmodern}
\usepackage[utf8]{inputenc}
\usepackage[dvipdfmx, linktocpage]{hyperref} %リンクを有効にする
\usepackage{tikz}
\usepackage[all]{xy}
\usepackage{here}
\usepackage{url} %同上
\usepackage{amsmath,amssymb} %数式とか
\usepackage{mathtools} 
\usepackage{braket,physics} %ブラケット
\usepackage{bm} %ベクトル
\usepackage[top=20truemm,bottom=20truemm,left=15truemm,right=15truemm]{geometry} %余白設定
\usepackage{latexsym} 
\usepackage{amsthm}

\newtheoremstyle{break}
{\topsep}{\topsep}%
{}{}%
{\bfseries}{}%
{\newline}{}%
\theoremstyle{break}
\newtheorem{thm}{Theorem}[section]
\newtheorem{defn}[thm]{Definition}
\newtheorem{eg}[thm]{Example}
\newtheorem{cl}[thm]{Claim}
\newtheorem{lem}[thm]{Lemma}
\newtheorem{ax}[thm]{Axiom}
\newtheorem{cor}[thm]{Corollary}
\newtheorem{fact}[thm]{Fact}
\newtheorem{rem}[thm]{Remark}
\makeatletter
\newenvironment{pr}[1][\proofnam]{\par
%\newenvironment{Proof}[1][\Proofname]{\par
\topsep6\p@\@plus6\p@ \trivlist
\item[\hskip\labelsep{\itshape #1}\@addpunct{\bfseries}]\ignorespaces
}{%
\endtrivlist
}
\newcommand{\proofnam}{\underline{Derivation.}}
\makeatother
\usepackage{latexsym} 
\newcommand{\R}{\mathbb{R}}
\newcommand{\Z}{\mathbb{Z}}
\newcommand{\N}{\mathbb{N}}
\newcommand{\C}{\mathbb{C}}
\newcommand{\D}{\mathcal{D}}
\renewcommand{\O}{\mathcal{O}}
\newcommand{\SO}{\mathrm{SO}}
\newcommand{\SU}{\mathrm{SU}}
\newcommand{\so}{\mathfrak{so}}
\newcommand{\su}{\mathfrak{su}}
\newcommand{\GL}{\mathrm{GL}}
%\newcommand{\g}{\mathfrak{g}}
\newcommand{\diff}{\mathrm{d}}
\newcommand{\drv}[2]{\frac{\mathrm{d} #1}{\mathrm{d} #2}} %常微分
\newcommand{\drvn}[3]{\frac{\mathrm{d}^{#1} #2}{\mathrm{d} #3^{#1}}}%高階常微分
\newcommand{\pdrvn}[3]{\frac{\partial^{#1}#2}{\partial #3^{#1}}}%高階偏微分
\newcommand{\h}{\mathcal{H}}
\newcommand{\Hilb}{\mathcal{H}}
\newcommand{\Lsq}{\mathrm{L^2}(\R)}
\DeclareMathOperator{\Dom}{Dom}
\newcommand{\onepara}{one-parameter strongly continuous unitary group}
\newcommand{\sa}{self-adjoint operator}
\usepackage{mathrsfs} %花文字 \mathscr
\usepackage{ascmac}
\usepackage{framed,color}%枠等
\definecolor{lightgray}{rgb}{0.75,0.75,0.75}
\usepackage{comment}
\usepackage[dvipdfmx]{color, hyperref}
\usepackage{pxjahyper}
\usepackage{cite}
\usepackage{xcolor}
\hypersetup{
    colorlinks=true,
    citecolor=blue,
    linkcolor=teal,
    urlcolor=orange,
 }
\makeatletter
\@addtoreset{equation}{section}
\makeatother
\numberwithin{equation}{section}
\renewcommand{\thefootnote}{\roman{footnote}.}
\renewcommand{\appendixname}{Appendix }

\renewcommand{\abstractname}{Abstract}
\renewcommand{\headfont}{\bfseries}
\title{Quamtum Mechanics}
\author{Toshiya Tanaka}
\date{\today}

\begin{document}
\maketitle

\begin{abstract}
		量子力学では,\sa が物理量に対応している.
		また,運動量やHamiltonianといった物理量は
		空間並進や時間発展などの操作を与えるunitary作用素を与える.
		物理では,作用素の定義域を気にすることは少ないが,
		数学的には定義域が重要で,``定義できるところで定義する''ことが大切になる.
		定義域を直接議論することはむずかしいが,Stoneの定理により\onepara から定義域を含めた\sa を構成することができる\cite{Hall2013}.

		なお,本稿の内容はspm22ndの特別講演で教わった.
\end{abstract}
\section{Stoneの定理}
\begin{defn}[\onepara]
		$\Hilb$上の\onepara とは,$t\in\R$でパラメトライズされたunitary作用素の族$\qty{U(t)}_{t\in\R}$で,
		\begin{itemize}
				\item $U(0) = 1_{\Hilb}$
				\item $U(s)U(t) = U(s+t) $
				\item 任意の$\psi\in\Hilb$に対し,$\|\lim_{t\to0}U(t)\psi\|=\|\psi\| $
		\end{itemize}
		を満たすものをいう.
\end{defn}
\begin{thm}[Stone's theorem]
	$\qty{U(t)}_{t\in\R}$を$\Hilb$上の\onepara とする.
	$\psi\in\Hilb$に対し,
	\begin{align}
			\Dom A &\coloneqq \qty{\psi\in\Hilb \mid \lim_{t\to0}\frac{\|U(t)\psi - \psi \|}{it} = 0}, \\
			A\psi &\coloneqq \lim_{t\to0}\frac{U(t)\psi - \psi}{it}
	\end{align}
	と定めると,組$(A, \Dom A)$は$\Hilb$上denseに定義された\sa になる.

	この形から,\sa に対する\onepara を$\qty{e^{itA}}_{t\in\R}$と書く.
\end{thm}

\section{位置}
以下,$\Hilb =\Lsq$として考える.

\onepara を
\begin{align}
		\qty(e^{itX}\psi)(x) \coloneqq e^{itx}\psi(x)
\end{align}
と定める.これは\onepara である.
これにより
\begin{align}
		X\psi(x) &\coloneqq \lim_{t\to0}\frac{e^{itx}\psi(x) - \psi(x)}{it}\\
		  &= \left.\frac{\psi(x)}{i}\dv{e^{itx}}{t}\right|_{t=0}\\
		  &= x\psi(x)
\end{align}
と定めた$X\sim x$は\sa である.

これが位置演算子である.

\section{運動量}
\onepara を
\begin{align}
		\qty(e^{itP}\psi(x))(x) &\coloneqq \psi(x + \hbar t)
\end{align}
と定める.
これは\onepara である.

これにより,
\begin{align}
		P\psi(x) &= \lim_{t\to 0}\frac{\psi(x+\hbar t)-\psi(x)}{it}\\
		  &= -i\hbar\dv{\psi(x)}{x}
\end{align}
と定めた$P \sim -i\hbar\dv*{}{x}$は運動量演算子である.
%\footnote{Stone - von Neumannの定理というのがあって,CCRを満たすものは一意であることが言える.}

\bibliography{books, papers}
\bibliographystyle{ytamsalpha}
\end{document}
