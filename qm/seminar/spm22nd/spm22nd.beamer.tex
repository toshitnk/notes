\documentclass[dvipdfmx]{beamer}

\usepackage{pxjahyper}

% theme
\usefonttheme{professionalfonts}
\usefonttheme{serif}
\usetheme{Madrid}
\setbeamertemplate{caption}[numbered]
\setbeamertemplate{navigation symbols}{}
\setbeamertemplate{theorems}[numbered]
% graphics
\usepackage{graphicx}
\usepackage{here}

% math
\usepackage{amsmath, amssymb}
\usepackage{physics}
\usepackage{mathrsfs}
\usepackage{mathtools}

% theoremstyle
\usepackage{amsthm}
\newtheoremstyle{break}
{\topsep}{\topsep}%
{}{}%
{\bfseries}{}%
{\newline}{}%
\theoremstyle{break}
\newtheorem{thm}{Theorem}[section]
\newtheorem{defn}[thm]{Definition}
\newtheorem{eg}[thm]{Example}
\newtheorem{cl}[thm]{Claim}
\newtheorem{cor}[thm]{Corollary}
\newtheorem{rem}[thm]{Remark}
\newtheorem{ax}[thm]{Axiom}
\newtheorem{prob}{Problem}[section]

\makeatletter
\newenvironment{pr}[1][\proofnam]{\par
\topsep6\p@\@plus6\p@ \trivlist
\item[\hskip\labelsep{\itshape #1}\@addpunct{\bfseries}]\ignorespaces
}{%
\endtrivlist
}
\newcommand{\proofnam}{\underline{Derivation.}}
\makeatother

% link

\usepackage{url}
\usepackage{hyperref} 
\usepackage{xcolor}
\usepackage{pxjahyper}
\hypersetup{
	colorlinks=true,
	citecolor=blue,
	linkcolor=teal,
	urlcolor=orange,
}

% my command
\newcommand{\hilb}{\mathcal{H}}
\DeclareMathOperator{\Dom}{Dom}

\newcommand{\R}{\mathbb{R}}
\newcommand{\C}{\mathbb{C}}
\newcommand{\Z}{\mathbb{Z}}




% title
%\author{Toshiya Tanaka}
%\institute[富大理物]{富山大学理学部物理学科}
\title[\textcolor{white}{CCRについて}]{第22回数物セミナー特別講義\\
正準交換関係について}
\subtitle{Stone - von Neumannの定理とその応用としてのSegal - Bergmann空間}
\begin{document}
\begin{frame}
		\maketitle
\end{frame}

\begin{frame}{Outline}
		\tableofcontents
\end{frame}
\section{作用素についての準備}
\begin{frame}{作用素についての準備}
		\begin{itemize}
				\item Hilbert空間とは完備な内積空間
				\item Hilbert空間$\hilb$上の作用素$(A, \Dom A)$とは,\par
						denseな部分空間$\Dom A \subset \hilb$と線形写像$A\colon \Dom A \to \hilb'$の組
				\item $(A, \Dom A)$が有界なとき,\par
						つまり,ある定数$C$が存在して,$\forall \psi \in \Dom A$, $\|A\psi\|\leq C\|\psi\|$のとき,$\Dom A = \hilb$として拡張が一意に決まる.
		\end{itemize}
\end{frame}

\begin{frame}{例}
		\begin{eg}[有限次元ベクトル空間$\C^n$上の線形写像]
				$v = \sum_{j=1}{n}a_je_j $, $w = \sum_{j=1}^{n}b_je_j $のとき,
				$\langle v, w \rangle \coloneqq \sum_{j=1}^{n}\bar{a}_jb_j $
		\end{eg}
		
		\begin{eg}
				\begin{align}
						L^2(\R) & \coloneqq \qty{\text{可測関数}f\colon\R\to\C \mid \int_{\R}\abs{f(x)}^2\dd{x} < \infty},\\
						\langle f, g\rangle & \coloneqq \int_{\R}\bar{f(x)}g(x)\dd{x},\\
						\Dom (X) & \coloneqq \qty{f\in L^2(\R) \mid xf(x)\in L^2(\R)}
				\end{align}
				\begin{align}
						X  \colon \Dom(X)&\to L^2(\R)\\
						f(x) &\mapsto xf(x)
				\end{align}
		\end{eg}
\end{frame}

\begin{frame}{作用素について}
		\begin{itemize}
				\item 作用素$(A, \Dom A)$と$(B, \Dom B)$が同じであるとは
						\begin{itemize}
								\item $\Dom A = \Dom B$
								\item $\forall \psi \in \Dom A$; $A\psi = B\psi$
						\end{itemize}
				\item 以降,$A = B$と表す
		\end{itemize}
		
		\begin{itemize}
				\item 作用素$(A, \Dom A)$の共役作用素$(A^{*}, \Dom A^{*})$とは
				\begin{align}
						\Dom A^{*} & \coloneqq \qty{\psi\in\hilb\mid\exists\phi\in\hilb; \forall \langle \psi, A\psi\rangle = \langle\phi, \psi\rangle},\\
								&= \qty{\psi\in\hilb\mid \text{$\hilb$上の汎関数$\langle\psi, A\bullet\rangle$が有界である.}},\\
						A^{*}\psi &\coloneqq \phi
				\end{align}
				\item 作用素が有界であるとき,$\Dom A^{*} = \hilb$
		\end{itemize}
\end{frame}

\begin{frame}{作用素について}
		\begin{itemize}
				\item 作用素$(A, \Dom A)$が自己共役であるとは
						\begin{align}
								A = A^{*}
						\end{align}
				\item このとき,
						\begin{align}
						\forall \psi \in A;\quad \langle \psi, A\psi \rangle = \langle \psi, A\psi\rangle = \overline{\langle \psi, A\psi \rangle} \label{symmetric}
						\end{align}
						より,$\langle \psi, A\psi \rangle \in \R$.逆は成り立たない.
				\item Eq. \eqref{symmetric}の条件を対称という.
		\end{itemize}
\end{frame}

\begin{frame}{作用素について}
		\begin{ax}
				量子力学では,
				\begin{itemize}
						\item 状態をHilbert空間の元
						\item 物理量をHilbert空間上の自己共役作用素
				\end{itemize}
				として考える.
		\end{ax}
\end{frame}

\section{Stoneの定理}
\begin{frame}{Stoneの定理}
		\begin{itemize}
				\item 一径強連続ユニタリ群$\qty{U(t)}_{t\in \R}$とは
				任意の$t\in\R$で
				\begin{itemize}
						\item $U(t)\colon \hilb \to \hilb $がユニタリ作用素
						\item $U(s+t) = U(s)U(t)$
						\item $\lim_{t\to 0}\|U(t)\psi - \psi\| = 0$
				\end{itemize}
		\end{itemize}
\end{frame}

\begin{frame}{Stoneの定理}
		\begin{thm}[Stoneの定理]
				任意の一径強連続ユニタリ群$\qty{U(t)}$に対し
				\begin{itemize}
						\item $\Dom A \coloneqq \qty{\psi \in \hilb \mid (U(t)\psi - \psi)/(it)\text{が$t\to0$で収束}}$
						\item $A\psi \coloneqq \lim_{t\to0}(U(t)\psi - \psi)/(it)$
				\end{itemize}
				とさだめると,$A$は自己共役作用素である.
		\end{thm}
\end{frame}
\section{Stone - von Neumannの定理}
\section{Segal - Bergmann変換}

\end{document}










%\documentclass[dvipdfmx, a4paper]{jsarticle}
%\usepackage[utf8]{inputenc}
%\usepackage[top=10truemm, bottom=20truemm, left=15truemm, right=15truemm]{geometry} % mergin
%\renewcommand{\headfont}{\bfseries}
%
%% graphics
%\usepackage{graphicx}
%\usepackage{here}
%
%% link
%
%\usepackage{url}
%\usepackage[dvipdfmx, linktocpage]{hyperref} 
%\usepackage{xcolor}
%\usepackage{pxjahyper}
%\hypersetup{
%	colorlinks=true,
%	citecolor=blue,
%	linkcolor=teal,
%	urlcolor=orange,
%}
%
%% math
%
%\usepackage{amsmath, amssymb} 
%\usepackage{physics}
%\usepackage{mathrsfs}
%\usepackage{mathtools}
%
%% theoremstyle
%\usepackage{amsthm}
%\newtheoremstyle{break}
%{\topsep}{\topsep}%
%{}{}%
%{\bfseries}{}%
%{\newline}{}%
%\theoremstyle{break}
%\newtheorem{thm}{Theorem}[section]
%\newtheorem{defn}[thm]{Definition}
%\newtheorem{eg}[thm]{Example}
%\newtheorem{cl}[thm]{Claim}
%\newtheorem{cor}[thm]{Corollary}
%\newtheorem{fact}[thm]{Fact}
%\newtheorem{rem}[thm]{Remark}
%\newtheorem{prob}{Problem}[section]
%
%\makeatletter
%\newenvironment{pr}[1][\proofnam]{\par
%\topsep6\p@\@plus6\p@ \trivlist
%\item[\hskip\labelsep{\itshape #1}\@addpunct{\bfseries}]\ignorespaces
%}{%
%\endtrivlist
%}
%\newcommand{\proofnam}{\underline{Derivation.}}
%\makeatother
%
%
%% my command
%
%\newcommand{\R}{\mathbb{R}}
%\newcommand{\C}{\mathbb{C}}
%\newcommand{\Z}{\mathbb{Z}}
%
%\newcommand{\eq}[1]{Eq. \eqref{#1}}
%\newcommand{\theorem}[1]{Thm. \ref{#1}}
%\newcommand{\definition}[1]{Def. \ref{#1}}
%\newcommand{\proposition}[1]{Prop. \ref{#1}}
%\newcommand{\example}[1]{e.g.\ref{#1}}
%\newcommand{\claim}[1]{Cl. \ref{#1}}
%\newcommand{\corolary}[1]{Cor. \ref{#1}}
%\newcommand{\remark}[1]{Rem. \ref{#1}}
%\newcommand{\problem}[1]{Prob. \ref{#1}}
%
%\renewcommand{\O}{\mathcal{O}}
%
%
%% number
%
%%\makeatletter
%%\@addtoreset{equation}{section}
%%\makeatother
%%\numberwithin{equation}{section}
%%\renewcommand{\thefootnote}{\roman{footnote}.}
%%\renewcommand{\appendixname}{Appendix }
%
%
%%English
%
%\renewcommand{\tablename}{Tab. }
%\renewcommand{\figurename}{Fig. }
%
%
%
%\title{}
%\author{Toshiya Tanaka}
%\date{\today}
%
%\begin{document}
%	\maketitle
%
%	\bibliography{books}
%	\bibliographystyle{ytamsalpha}
%\end{document}
%
