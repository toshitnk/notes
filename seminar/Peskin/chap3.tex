\section{The Dirac Field}
\begin{itemize}
	\item \textref{3.20}の不変性の書き方が,分かりづらく感じる.
		Lorentz変換$\Lambda$により,位置は$x\mapsto x'= \Lambda x$と変換し,
		場は$\phi(x) \mapsto \phi'(x')$と変換するが,座標は動かさず,場を動かすという立場を取っていて,変換後も変換前と同じ$x$を使うと思うと,$x' = x$
		と起き直して,もとの$x$は$\Lambda^{-1}x$になるので,$\phi'(x)=\phi(\Lambda^{-1}x)$という書き方をしている.
	\item \textref{3.17}の定義は,他の本だと片方$g_{\mu\nu}$になっているが,
		これは足を片方上げた状態であるのでconsistent.
		これにより,$\alpha, \beta$などは単なる行列の足ではなく,
		Lorentzベクトルの足なので,\textref{3.20}, \textref{3.21}を計算するときは,足の上下に応じて,空間成分ならマイナスをかけないといけない.
\end{itemize}
