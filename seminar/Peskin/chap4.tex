\section{Interacting Fields and Feynman Diagrams}
\begin{itemize}
	\item section 4.1は今後の方針が書かれている.
		今まで考えていたのは,相互作用のない自由場だったが,
		相互作用のある場が現実である.

		Lagrangianを書き下すルールの一つにくりこみ可能性があり,次元解析だけで,ある程度形が限定される.また,理論の対称性などを考慮するとさらに限定できる.
		そうしてLagrangianが書かれたとして,どのように物理量を計算するのか.
		厳密に解ける模型はほぼないので,自由場からの摂動で解く.その方法を見つけるのが,以後の話である.
	\item pp.94, 95の最後で$p^0\propto(1+\i\varepsilon)$にとるのは,Feynman propagatorのpoleの避け方を思うと,多少納得できる.
		もとの$T\colon-\infty(1-\i\varepsilon) \to \infty(1-\i\varepsilon)$で積分を行うことを考えると,振動項を無視して,
		\begin{equation}
			\int_{-\infty(1-\i\varepsilon)}^{\infty(1-\i\varepsilon)}\dd{t}\e^{-\i p^0 t} \sim -\e^{p^0\varepsilon\infty} \to -\infty
		\end{equation}
		が発散する.ここで,$p^0\propto(1 + \i\varepsilon)$とすると,$(1-\i\varepsilon)(1+\i\varepsilon) = 1 + \varepsilon^2\in\R$になり,発散しなくなると思う.
	\item p.96の$(2\pi)^4\delta^{(4)}(0) = 2TV$は
		\begin{equation}
			\delta^{(4)}(p) = \intp[4]{x}\e^{-\i px}
		\end{equation}
		のFourier変換を使うと,\textref{4.49}の形になり,$t$積分の$2T$と$x$積分の$V$で無限大の$\delta^{(4)}(0)$を有限の$2TV$の$T, V\to \infty$極限と解釈する.
	\item Feynman ruleでSymmetry factorで割るのは,同じものは一度だけ数えるためである.
		この思想は統計力学の等重率の原理に似ていて,全ての状態(diagram)を同じ重みで数えるためである.
		積分する4つの$\phi(z)$の入れ替えによる$4!$や摂動の$n$次の項で内点を入れ替える$n!$を考慮したとして,diagramの対称性に関してダブリがでる場合がある.
		例えば,次の場合\footnote{tikz-feynhandの練習も兼ねて書いてみた.},$z$同士で縮約をとる次の項はダブル上のルールだけだとダブルカウントしてしまう.
		\begin{equation}
			\begin{tikzpicture}
				\begin{feynhand}
					\vertex  (a) at(-1, 0) {$x$};
					\vertex  (b) at (1, 0) {$y$};
					\vertex (c) at (0, 0);
					\vertex (d) at (0, 1);
					\propag [plain] (a) to (b);
					\propag [plain] (c) to [out=45, in=0] (d) to [out=180, in=135] (c);
				\end{feynhand}
			\end{tikzpicture}
			=
			-\i\lambda\int \dd[4]{z} D(x-z)D(z-y)D(z_1-z_2) -\i\lambda\int \dd[4]{z} D(x-z)D(z-y)D(z_2-z_1)
		\end{equation}
		わかりやすさのために,まず区別して$z_1$と$z_2$としたが,これらは同じもので,重み$1$で足さなければいけないので,Symmetry factor $2$で割る必要がある.
	\item 他にも,統計力学っぽいところがあって,例えば,\textref{4.52}の和と積を入れ替える一連の計算は分配関数を計算するときによく使うテクニックである.
	\item \textref{4.55}の一つ前の式の右辺
		\begin{equation}
			\mel{\Omega}{\T(\phi(x)\phi(y))}{\Omega}\lim_{T\to\infty(1-\i\varepsilon)}(\abs{\braket{0}{\Omega}}^2e^{-\i E_0 2T})
		\end{equation}
		について,$\mel{\Omega}{\T(\phi(x)\phi(y))}{\Omega} $は外点につながっているdiagramの和で,$\e^{-\i E_0 2T}$は真空泡と議論があるが,忘れ去られている
		$\abs{\braket{0}{\Omega}}^2$は\textref{4.55}の比例に吸収させると思う.
	\item \textref{4.58}は今まで浸かっていた``disconnected''というワードは外点にdisconnectedの意味で使われていたが,
		このように二つに別れてしまうdiagramのこともdisconnectedというので,注意が必要であることをいう.
		前者は以前の議論で,分母分子でキャンセルするのでcorrelation functionに寄与しないのに対し,後者は
		correlation functionに寄与するという違いがある.
	\item \textref{4.64}は,不安定状態を考えているので,$p^0\neq E_{\vec{p}}=\sqrt{\vec{p}^2 + m^2}$であることに注意して,
		\begin{align}
			\frac{1}{p^2-m^2 + \i m\Gamma}
			& = \frac{1}{(p^0)^2 - E_{\vec{p}}^2  +\i m\Gamma}\\
			& \sim \frac{1}{2E_{\vec{p}}(p^0-E_{\vec{p}}) + \i m\Gamma}\label{eq:on-shell_approx}\\
			& = \frac{1}{2E_{\vec{p}}(p^0-E_{\vec{p}}+\i m\Gamma/(2E_{\vec{p}}))}
		\end{align}
		となる.Eq. \eqref{eq:on-shell_approx}では
		$p^0$が$E_{\vec{p}}$に近いとしている.
	\item Peskinのin, out stateは相互作用場で$T\to \pm\infty$にしたと定義されているが,
		他の本\cite{Kugo1989}, \cite{Sakamoto2020}では$T\to\pm \infty$で自由場と一致しているとして,定めているので,in, out場の定義が違うので,注意が必要である.あとで,相互作用場での$\ket{\vec{p}}$を自由場での$\ket{\vec{p}}_0$で表す操作をするが,そちらが通常のin, out場の定義である.
	\item \textref{4.68}のimpact parameterによるphaseのずれだが,通常の平行移動とおもうと,指数の方はプラスで出るように思う.
		しかし,$k_{\mathcal{B}}$の全空間で積分しているし,これは先の計算でデルタ関数として現れるだけなので,
		あまり議論に影響はしない.

	\item \textref{4.70}のS-matrixの定義は,二行目から見るのがわかりやすくて,同じ時間$t=0$で$\ket*{\vec{k}_{\mathcal{A}}\vec{k}_{\mathcal{B}}}$と$\bra{\vec{p}_1\vec{p}_2\cdots}$があって,
		$\ket*{\vec{k}_{\mathcal{A}}\vec{k}_{\mathcal{B}}}$を$H$で$-T$時間発展させたのがin stateで
		$\bra{\vec{p}_1\vec{p}_2\cdots}$を$T$時間発展させたのがout stateである.
	\item S-matrixを考えれば良いのだが,常に存在する自明な項があるので,この本ではそれを除いて考える.
		常に出てくる項とは
		\begin{enumerate}
			\item $\mathsf{S} = 1 + \i \mathsf{T}$
				\footnote{$\mathsf{S}$をscattering matrix, $\mathsf{T}$をtransfer matrixと言う.日本語ではそれぞれ散乱行列と転送行列であり,その$\mathsf{S}$と$\mathsf{T}$であるとゼミでボケようと思ったが,それまでに炎上したので自粛した.余裕のある人はこのネタを供養してください.}
				と分けたときの,$1$
			\item 運動量保存の$(2\pi)^4\delta^{(4)}(p_{\mathcal{A}}+p_{\mathcal{B}} - \sum_fp_f)$\footnote{out stateの添字に$f$を使っているのは,final stateのfだと思う.}
		\end{enumerate}
		である.これらを除いたものがinvariant matrix element $\M$である.
	\item \textref{4.74}は運動量状態$\prod_f 1/\sqrt{2E_f}\ket{\vec{p}_1\cdots}$へのamplitudeを求めている.
		なぜ,\textref{4.68}の$\phi_f(\vec{p}_f)$などがないのかと悩んだが,これは波束をFourier変換したときの
		展開係数なので単なる一つのモードを考えている今は当然必要ない.
		また,$1/\sqrt{2E_f}$はLorentz不変なnormalizationのためのfactorで,求めたいのは確率なので,二乗すると,必要な式を得る.
	\item \textref{4.76}の$\dd{\sigma}$を求める計算は,まず$\bar{k}_i$の6つの積分をまず計算することで,barのついている変数がデルタ関数で潰れることを確認する.
		
		まず,ここではimplicitにビーム方向に$z$軸をとって,impact parameter方向の二次元を$\perp$と取っている.ここで,$\bar{k}_{\mathcal{B}}^{\perp}$積分で$\bar{k}_{\mathcal{B}}^{\perp} = k_{\mathcal{B}}^{\perp}$を得て,
		これと,前のデルタ関数の結果を先取りすることで,$\bar{k}_{\mathcal{A}}^{\perp} = \sum_fp_f^{\perp} - k_{\mathcal{B}}^{\perp} = k_{\mathcal{A}}$となる.

		z成分は\textref{4.77}の計算が必要だが,ここで使っているデルタ関数の公式は
		\begin{equation}
			\delta(f(x)) = \sum_{\text{zeros}}\frac{1}{\abs{f'(x)}}\delta(x-x_0)
		\end{equation}
		と全ての零点を拾わなければならないが,ここでは和がない.
		デルタ関数のなかにある$\bar{k}_{\mathcal{A}}^z$についての関数の零点は,一般に2つあることが予期されるので正確にはこれではいけないはずである.

		これは次のように解釈できる.今の状況は運動量が狭い範囲にある波束を考えているので,
		その範囲は零点を一つしか拾わないように設定されていると思う.
		実際,次の計算でそのような近似を使うので,そこまでは和の状況で残しておき,そこでひとつだけ零点を拾うという議論のほうが筋は良い気がする.

		今の積分は$\delta(\bar{E}_{\mathcal{A}} + \bar{E}_{\mathcal{B}}-\sum_fE_f)$で$\bar{k}_{\mathcal{A}}^z = \sum_fp_f^z - \bar{k}_{\mathcal{B}}^z$となっているものを計算していて,
		\begin{align}
			&\sqrt{(k_{\mathcal{A}}^{\perp})^2+(\textcolor{red}{\bar{k}_{\mathcal{A}}^z})^2+m_{\mathcal{A}}^2} + \sqrt{(k_{\mathcal{B}}^{\perp})^2 + (\sum_fp_f^z-\textcolor{red}{\bar{k}_{\mathcal{A}}^z})^2+m_{\mathcal{B}}^2}\\
			& = \sum_fE_f = E_{\mathcal{A}} + E_{\mathcal{B}}\\
			& = \sqrt{(k_{\mathcal{A}}^{\perp})^2+(\textcolor{red}{k_{\mathcal{A}}^z})^2+m_{\mathcal{A}}^2} + \sqrt{(k_{\mathcal{B}}^{\perp})^2 + (\sum_fp_f^z-\textcolor{red}{k_{\mathcal{A}}^z})^2+m_{\mathcal{B}}^2}
		\end{align}
		であるので,$\bar{k}_{\mathcal{A}}^z = k_{\mathcal{A}^z}$となり,
		$\bar{k}_{\mathcal{B}}^z = \sum_fp_f^z-k_{\mathcal{A}}^z = k_{\mathcal{B}}^z$となる.
		これで,$\vec{\bar{k}}_{i}^z = \vec{k}_i^z$が言えるので,エネルギーについても$\bar{E}_i=E_i$となり,
		\textref{4.78}のように二乗でまとめることができる.
	\item $k/E = v$の書き換えは,静止系からLorentz変換したことを考えると
		\begin{equation}
			\mqty(\gamma & \gamma\beta\\ \gamma\beta & \gamma)\mqty(m\\0) = \mqty(E\\k)
		\end{equation}
		なので,$k/E = \beta = v$となる.

	\item \textref{4.82}でもデルタ関数の公式を使って,$\delta(E_{\text{CM}}-E_1-E_2)=(p_1/E_1 + p_1/E_2)^{-1}\delta(p-p_0)$としているが,これも真面目に考えるとおかしくて,$p_1$積分は$0\to \infty$で零点を計算をするとプラスマイナスの組ででてくるので複数拾うことはないにしろ,$E_{\text{CM}} < m_1+m_2$だと解を持たないのでゼロになる.
		零点をしらべるより,デルタ関数を変数変換して調べたほうがわかりやすくて\cite{Schwartz:2014sze},$x\coloneqq \sqrt{p_1^2+m_1^2} + \sqrt{p_1^2 + m_2^2} - E_{\text{CM}}$すると,
		\begin{equation}
		\int_{m_1+m_2-E_{\text{CM}}}^{\infty}\delta(x)
		\end{equation}
		となるが,これは$m_1+m_2 > E_{\text{CM}}$だと$0$を拾わない.
		
		しかし,これは物理的に考えるとtotal energyがmassより小さかったら,そのような散乱は起こらず興味のない結果になってしまうので,これが満たされているのは当然とも思える.
	\item \textref{4.102}の$1/2$のfactorはloopのsymmetry factorで割っている.
\end{itemize}
