\section{The Klien--Gordon Field}
\begin{itemize}
	\item p.14で,$E = \sqrt{p^2+m^2}$の方で計算したpropagator
		\begin{equation}
			U(t) = \frac{1}{2\pi^2\abs{\vec{x}-\vec{x}_0}}\int_0^\infty\dd{p}\sin(p\abs{\vec{x}-\vec{x}_0})\e^{-\i t\sqrt{p^2+m^2}}
		\end{equation}
		を直接評価する方法は,本でも引用されているように,\cite[p.491]{GRA80}\footnote{この本はネット上でも見れるが,1200ページくらいあり,とても重いので注意.}を見れば良い.
		\begin{equation}
			\int_0^{\infty}x\e^{-\beta\sqrt{\gamma^2+x^2}}\sin\beta x \dd{x} = \frac{b\beta\gamma^2}{\beta^2+b^2}K_2(\gamma\sqrt{\beta^2+b^2})
		\end{equation}
		という式があり,この収束する積分の指数の肩を虚数倍ひねる,いわゆる解析接続をしていると解釈できる.元の積分は$p$が無限で発散し,虚数の指数関数は振動するので,被積分関数を見ると発散してしまうことがわかる.\footnote{この一連の議論はd氏に教えていただいた.}
	\item \textref{2.31}で$\delta(0)$の無限大をむしすることについて,
		\begin{itemize}
			\item GRを考えるときは無視できない.
			\item SUSYを入れると出ない.
		\end{itemize}
	\item \textref{2.33}の計算は奇関数が対称区間の積分で消えることを考える.素直に代入して,
		\begin{align}
			\vec{P} &= -\int\dd[3]{x}\pi(\vec{x})\vec{\nabla}\phi(\vec{x})\\
					&= -\int\dd[3]{x}\intp{p}\intp{p'}\qty(\qty(-\i\sqrt{\frac{\omega_{p}}{2}})(a_p-a_{-p}^{\dag})\e^{\i\vec{p}\cdot\vec{x}}(\i\vec{p}^{\, \prime})\frac{1}{\sqrt{2\omega_{p'}}}(a_{p'}+a_{-p'}^\dag)\e^{-\i\vec{p}\cdot\vec{x}})
		\end{align}
		となり,
		\begin{equation}
			\intp{x}\e^{\i(\vec{p}+\vec{p}{\, '})\cdot\vec{x}} = \delta^{(3)}(\vec{p}+\vec{p}\,')
		\end{equation}
		を使うと,
		\begin{equation}
			\vec{P} = -\intp{p}(-\vec{p})\frac{1}{2}(a_p-a_{-p}^{\dag})(a_{-p}+a_{p}^\dag)
		\end{equation}
		となる.今,$(a_pa_{-p}-a_{-p}^\dag a_{-p}+a_pa_p^{\dag}+a_{-p}a_{p}^\dag)$
		となるが,$p$と$-p$が交互に入っているものは奇関数になり,消える.		\begin{equation}
			\intp{p}\vec{p}(a_pa_{-p}+a_{-p}^{\dag}a_{p}) = 0.
		\end{equation}
		すると,添字の運動量は,符号が一致したものしか残らず,全空間の積分なので符号を変えて,全て$+p$で計算するテクニックが使える.
		これより,
		\begin{align}
			\vec{P} &= \intp{p}\vec{p}\frac{1}{2}\qty(a_p^{}a_p^{\dag}-a_{-p}^\dag a_{-p}^{})\\
					&= \intp{p}\vec{p}\qty(a_p^{\dag}a_{p}^{}+\frac{1}{2}\qty[a_{p}{}, a_{p}^{\dag}])\\
					&= \intp{p}\vec{p}a_p^\dag a_p^{}
		\end{align}
		となる.最後の$\qty[a_{p}{}, a_{p}^{\dag}])= \delta(0)$は偶関数と思うと消える.
	\item \textref{2.40}は三次元空間の$\dd[3]p/(2E_p)$でmeasureを入れたものだが,$\R^{1,3}$に埋め込むと$p^0>0$の方の双曲超平面上の積分と思える.
	\item \textref{2.41}, \textref{2.42}から,$\phi(x)\ket{0}_{\text{QFT}}\sim\ket{\vec{x}}_{\text{QM}}$, $\mel{0}{\phi(x)}{p}_{\text{QFT}}\sim\braket{x}{p}_{\text{QM}}=\e^{\i px}$と対応がつく.
	\item section 2.4では今まで,時間に依存しないSch\"{o}dinger描像でやっていたものを,Heisenberg描像に移す.やりかたは,QMと同じように$\O_{\text{Heisenberg}} = \e^{\i Ht}\O_{\text{Schr\"{o}dinger}}\e^{-\i Ht}$とする.
	\item 生成消滅演算子のHeisenberg描像は$\e^{\i Ht} = \sum_{n=0}^{\infty}\frac{\i Ht}{n!}$と$H^na_p=a_p(H-E_p)$など\footnote{生成消滅のこのような関係式は,片方について調べると,もう片方はエルミート共役を取れば直ちに成り立つことを確認できる.}
		を用いて,$a_p\e^{-iE_pt}$などになり,場の演算子も綺麗にまとまる.
	\item \textref{2.51}の最後の評価は,鞍点ではないが,振動の遅いところが積分に最も寄与すると思うと,$E=m$の値を採用すると考える.
	\item branch cutをまわる積分\textref{2.52}の計算の仕方は\href{https://physics.stackexchange.com/questions/285126/an-integral-in-peskins-quantum-field-theory-p-27}{physics SE. 285126}の回答がわかりやすい.
		まず,複素関数で平方根をナイーブに考えてしまうと,
		$\sqrt{\i}$などと書いたときに,$\e^{\i\pi/4}$でも$\e^{3\i\pi/4}$でも二乗したら$\i$になるので,$\i\mapsto\e^{\i\pi/4}, \e^{5\i\pi/4}$となってしまい
		写像としてよくない.
		このようなときに$\i=\e^{\i\pi/2}\mapsto\e^{\i\pi/4}$, $\i=\e^{5i\pi/2}\mapsto \e^{5\i\pi/4}$と考える.
		一般に,複素数の変革$\theta$が$\theta\in[0, 2\pi)$と$\theta\in[2\pi, 4\pi)$で区別し,複素平面二枚を定義域と思うという手法を考える.
		この拡張した定義域をRiemann面という.

		このとき,$\theta\colon2\pi-\epsilon\to2\pi+\epsilon$は連続的につながっていて,この二枚をつないでくっつけるところをbranch cutという.

		\textref{2.52}の積分は,実軸上の積分路を連続変形することにより,branch cutに沿う積分に変換するという方針で行う.このとき注意すべきことは,積分路は常に一枚目にあり,branch cutの両辺では定義域の偏角が$2\pi$ずれており,したがって.
		符号が違う値が出てくることである.

		具体的に積分は次のようにして行う.
		まず,球座標に変数変換して,積分する.
	
		\begin{align}
				D(x) &= \int_{0}^{\infty}\dd{k}\int_{0}^{\pi}\dd{\theta}\int_{0}^{2\pi}\dd{\phi}k^2\sin\theta\frac{\e^{-\i\vec{k}\cdot\vec{x}\cos\theta}}{(2\pi^3)2\sqrt{k^2+m^2}}\\
					 &= \frac{\i}{8\pi^2}\int_{0}^{\infty}\dd{k}\frac{k}{\sqrt{k^2+m^2}}\qty(\e^{-\i kx}-\e^{\i kx})\\
					 &= \frac{-\i}{8\pi^2}\int_{-\infty}^{\infty}\dd{k}\frac{k}{x\sqrt{k^2+m^2}}\e^{\i kx}\\
					 &= \frac{-1}{8\pi^2x}\pdv{}{x}\int_{-\infty}^{\infty}\dd{k}\frac{\e^{\i kx}}{\sqrt{k^2+m^2}}.
		\end{align}
	
		これより,積分
		\begin{equation}
				I \coloneqq \int_{-\infty}^{\infty}\dd{k}\frac{\e^{\i kx}}{\sqrt{k^2+m^2}} 
		\end{equation}
		を計算すればよい.
	
		被積分関数のpoleは$k=\pm \i m $にあり,$\sqrt{k^2+m^2} = \sqrt{k+\i m}\sqrt{k- \i m} $なので,branch cutは$(-\infty, -\i m] $, $[\i m, \infty) $に取れば良い.
		
		ここで,積分路をCauchyの積分定理を用いて,次のように変形する.
		\begin{equation}
				I = \int_{\i\infty}^{\i m}\dd{k}\frac{\e^{\i kx}}{-\sqrt{k^2+m^2}}+\int_{\i m}^{\i\infty} \dd{k}\frac{\e^{\i kx}}{\sqrt{k^2+m^2}} =2 \int_{\i m}^{\i\infty} \dd{k}\frac{\e^{\i kx}}{\sqrt{k^2+m^2}}
		\end{equation}
		となる.
		これにより,$k=\i(y+m) $と変数変換すれば
		\begin{equation}
				I = 2\int_{0}^{\infty}\dd{y}\frac{\e^{-(m+y)x}}{\sqrt{(y+m)^2-m^2}}
		\end{equation}
		となる.さらに,$y+m=u $,$u=\cosh\eta $と変数変換することで,
		\begin{align}
				I &=2 \int_{1}^{\infty} \dd{u}\frac{\e^{-mux}}{\sqrt{u^2-1}}\\
				  &=2 \int_{0}^{\infty} \dd{\eta}\e^{-mx\cosh\eta}
		\end{align}
		となる.propagatorに戻ると,
		\begin{align}
				D(x) &= \frac{-1}{4\pi^2}\pdv{}{x}\int_{0}^{\infty} \dd{\eta}\e^{-mx\cosh\eta}\\
					 &= \frac{-1}{4\pi^2}\int_{0}^{\infty}\dd{\eta}(-m\cosh\eta)\e^{-mx\cosh\eta}
		\end{align}
		となり,$\sinh\eta=s $とおくことで,
		\begin{align}
				D(x) &= \frac{m}{4\pi^2}\int_{0}^{\infty}\dd{s}\e^{-mx\sqrt{1+s^2}}\\
					 &\simeq \frac{m}{4\pi^2}\frac{1}{2}\sqrt{\frac{2\pi}{mx}}\e^{-mx}\\
					 &= \frac{m^2}{4\pi^2}\sqrt{\frac{\pi}{2(mx)^3}}\e^{-mx}
		\end{align}
		となる.
	\item \textref{2.56}ではstep function\footnote{自分が知っている定義は,$\theta(0) = 1/2$とするものだが,$\theta(0)=0$とする定義もあるらしいが,どこかで綻びはないのか.}やdelta functionの微分を考える必要がある.
		これらは積分をされたときの振る舞いにより,定義するのがうまい方法なので,
		次のように考えることができる.
		\begin{align}
			\int_{-\infty}^{\infty} \dd{x}f(x)\dv{\theta(x)}{x}
			&= \qty[f(x)\theta(x)]_{-\infty}^{\infty} - \int_{-\infty}^{\infty}\dd{x} \dv{f(x)}{x}\theta(x)\\
			&= f(\infty)-\int_{0}^{\infty}\dd{x} \dv{f(x)}{x}\\
			&= f(\infty)-f(\infty)+f(0)\\
			&= f(0)\\
			&= \int_{-\infty}^{\infty}f(x)\delta(x)
		\end{align}
		により,$\dv*{\theta(x)}{x}=\delta(x)$,
		\begin{align}
			\int_{-\infty}^{\infty}\dd{x}f(x)\dv{\delta(x)}{x}
			&= \qty[f(x)\delta(x)]_{-\infty}^{\infty}-\int_{-\infty}^{\infty}\dv{f(x)}{x}\delta(x)\\
			&= \int_{-\infty}^{\infty}\dd{x}\qty(-\dv{f(x)}{x}\delta(x))
		\end{align}
		により,$f(x)\dv*{\delta(x)}{x}=-\dv*{f(x)}{x}\delta(x)$となる.
	\item \textref{2.56}の微分は$x$に関しておこなっている.また,$\delta(x^0-y^0)$があるので,積分したら同時刻になると思い,CCR$[\phi(x), \pi(x)]=i\delta^{(4)}(x-y)$を使っている.
	\item また,$(\partial^2+m^2) \mel{0}{[\phi(x), \phi(y)]}{0} = 0$はKlein--Gordon方程式からゼロになる.
	\item propagatorのpoleの避け方は経路をいじって避けるというよりは,pole自身を$i\epsilon$でずらして,$\epsilon\to0$の極限をとると思うのが良い
		\footnote{経路をずらした場合は,どのように避けても値はpoleを上下にずらした場合の平均で同じ値になると思う.\href{https://mathrelish.com/physics/cauchy-principal-value-in-complex-space\#toc\_id\_3\_1}{こちらの記事}が参考になる.}.
	\item p.33の最後は$p^2=m^2$はそのような粒子が実際に観測されるということだけで,一連の議論には使っていないと思う.
\end{itemize}
