\documentclass[dvipdfmx, a4paper]{jsarticle}
\usepackage[utf8]{inputenc}
\usepackage[top=10truemm, bottom=20truemm, left=15truemm, right=15truemm]{geometry} % mergin
\renewcommand{\headfont}{\bfseries}

% graphics
\usepackage{graphicx}
\usepackage{here}

% link

\usepackage{url}
\usepackage[dvipdfmx, linktocpage]{hyperref} 
\usepackage{xcolor}
\usepackage{pxjahyper}
\hypersetup{
	colorlinks=true,
	citecolor=blue,
	linkcolor=teal,
	urlcolor=orange,
}

% math

\usepackage{amsmath, amssymb} 
\usepackage{physics}
\usepackage{mathrsfs}
\usepackage{mathtools}

% theoremstyle
\usepackage{amsthm}
\newtheoremstyle{break}
{\topsep}{\topsep}%
{}{}%
{\bfseries}{}%
{\newline}{}%
\theoremstyle{break}
\newtheorem{thm}{Theorem}[section]
\newtheorem{defn}[thm]{Definition}
\newtheorem{eg}[thm]{Example}
\newtheorem{cl}[thm]{Claim}
\newtheorem{cor}[thm]{Corollary}
\newtheorem{fact}[thm]{Fact}
\newtheorem{rem}[thm]{Remark}
\newtheorem{prob}{Problem}[section]

\makeatletter
\newenvironment{pr}[1][\proofnam]{\par
\topsep6\p@\@plus6\p@ \trivlist
\item[\hskip\labelsep{\itshape #1}\@addpunct{\bfseries}]\ignorespaces
}{%
\endtrivlist
}
\newcommand{\proofnam}{\underline{Derivation.}}
\makeatother


% my command

\newcommand{\R}{\mathbb{R}}
\newcommand{\C}{\mathbb{C}}
\newcommand{\Z}{\mathbb{Z}}

\newcommand{\eq}[1]{Eq. \eqref{#1}}
\newcommand{\theorem}[1]{Thm. \ref{#1}}
\newcommand{\definition}[1]{Def. \ref{#1}}
\newcommand{\proposition}[1]{Prop. \ref{#1}}
\newcommand{\example}[1]{e.g.\ref{#1}}
\newcommand{\claim}[1]{Cl. \ref{#1}}
\newcommand{\corolary}[1]{Cor. \ref{#1}}
\newcommand{\remark}[1]{Rem. \ref{#1}}
\newcommand{\problem}[1]{Prob. \ref{#1}}

\renewcommand{\O}{\mathcal{O}}

\newcommand{\su}{\mathfrak{su}}
\newcommand{\SU}{\mathrm{SU}}
\newcommand{\g}{\mathfrak{g}}
\newcommand{\so}{\mathfrak{su}}
\newcommand{\SO}{\mathrm{SO}}

% number

\makeatletter
\@addtoreset{equation}{section}
\makeatother
\numberwithin{equation}{section}
\renewcommand{\thefootnote}{\roman{footnote}.}
\renewcommand{\appendixname}{Appendix }


%English

\renewcommand{\tablename}{Tab. }
\renewcommand{\figurename}{Fig. }
\renewcommand{\refname}{References}


\title{B4セミナー}
\author{Toshiya Tanaka}
\date{\today}

\begin{document}
	\maketitle
	\section{Introduction}
	\begin{itemize}
			\item この文書では教科書の誤植や説明不足だと感じたことをまとめています.
			\item 公式の誤植リストは\href{https://www.slac.stanford.edu/~mpeskin/ConceptsBook.html}{ここ}にあります\footnote{数式がtex記法でベタ打ちなのでつらい.MathJax使ってほしかった.}.ここにあるものも書きますが.公式文書を読むことをおすすめします.
			\item 「\href{https://www.ms.u-tokyo.ac.jp/~yasuyuki/sem.htm}{セミナーの準備の仕方について(河東泰之)}」はゼミ準備の心構えとして有名な文書なので一読するとよいでしょう.
					特に
					\begin{quote}
							黙って「何々である」とか,``It is easy to see...", ``We may assume that...", ``It is enough to show..."などと書いてあるのはすべて,なぜなのか徹底的に考えなくてはいけません.「本に書いてあるから」とか「先生がそう言うから」などの理由で,なんとなく分かったような気になるのは絶対にアウトです.そういうところは「なぜですか」と聞かれるに決まっているんですから,どうきかれてもすぐに答えられるように準備をしておく必要があります.
					\end{quote}
					ということは意識すべきです.
			\item ゼミの準備の仕方として,もう少し易しめに書かれているものとして,
					「\href{https://speakerdeck.com/kaityo256/book-reading}{輪講の準備の仕方(渡辺宙志)}」
					も参考になります.特に,12ページの
					\begin{quote}
							先生からの質問は詰問ではない
					\end{quote}
					ということは心に留めておくとよいと思います.
			\item この文書についても同じで,「この文書に書いてあるから」というのは根拠にはなりません.\footnote{誤植を訂正しているはずが,さらに誤植だったということは大いにありえます.}
	\end{itemize}

	
	\section{Symmetries of Space-Time}
\subsection{Relativistic particle kinematics}
\subsection{Natural units}
\begin{itemize}
		\item SIと自然単位系の変換を実際に計算すること.

		\item Eq. (2.30)は冪が間違い.正しくは$10^{-23} $.
\end{itemize}
\subsection{A little theory of discrete groups}
\begin{itemize}
	\item 物理では対称性が大事で,離散対称性と連続対称性に分けられる.前者の例はparityや時間反転に関する対称性で,後者の例は並進や回転などに関する対称性である.
	これらの操作は群(group)という構造を成し,それを古典力学や量子力学の理論は,理論の(線形)空間に作用させて不変であることを要請する.
	このようなことを調べる数学の分野を表現論という.物理でよくいう「群論」は群の表現論である.群の演算を線形空間の行列で表現するときに,量子力学ではこの線形空間が状態空間(Hilbert空間)で行列が物理量である.表現論は物理の本だと\cite{九後1989}や\cite{BN10398292}の7章などが標準的で,数学の本だと線形代数と接続がよいものとして\cite{BC13565134}がある.
		\item 対称群$\Pi_3 $の元が数学でよくある置換でないことに注意.$\Pi_3 $は元自身ではなく文字の場所にラベルをつけている.
\end{itemize}
\subsection{A little theory of continuous groups}
\begin{itemize}
		\item 対称性と保存則の関係は\cite[Chap.10, Sec.4]{BB17130464}に詳しい.
		\item 連続対称性に関する操作はLie群という構造を成し,Lie群は単位元近傍の微小変換を調べれば大体わかることが知られている.
				Lie群の単位元近傍は線形空間になっていて,行列の交換子で閉じる.この構造をLie環という.
				物理で重要な例は$\su(2) $というLie環\footnote{Lie群$G$に対し,Lie環はフラクトゥール小文字で$\g$と書くことがある.物理屋さんはあまりこの違いを気にしないので両方アルファベット大文字で書いてしまう文献もしばしばある.}で,量子力学でよくやる各運動量演算子$[J_j, J_k] = \i\epsilon_{jkl}J_l $がそれである.
				角運動量が回転に対する保存量だったことを思い出すと,$J_i$は微小回転に対応している.
				表現の次元はspinの大きさで勘定することがある.
				角運動量の大きさ$j $に対し,行列のサイズは$2j + 1 $になるので,.$2j+1 $次元表現をspin $j $表現と言ったりする.例えば,$1 $次元表現はspin $0 $表現で,$2 $次元表現はspin $1/2 $表現で$J_i $はPauli行列たちである.このあたりの話は\cite[Chap.3]{BB03663366}, \cite{BN10398292}あたりが詳しい.
		\item Eq. (2.55)は\href{https://en.wikipedia.org/wiki/Baker%E2%80%93Campbell%E2%80%93Hausdorff_formula}{BCH formula}から示せると言われる.確かに左辺を計算して,BCH formulaからLie bracketの入れ子の和の指数関数でかけることがいえ,Lie bracketで閉じているものがLie環なので左辺は群の元なのだが,それが$\gamma $と同じということ言えないと思う.
		\end{itemize}
\subsection{Discrete space-time symmetries}
\begin{itemize}
		\item Eq. (2.71)の$P $がかかって$+ $を吐くか$- $を吐くかは粒子や場の性質である.$- $を吐くほうを擬スカラー/ベクトルなどという.
\end{itemize}

	\section{Relativistic Wave Equations}
\subsection{The Klein--Gordon equation}
\subsection{Fields and particles}
\subsection{Maxwell's equations}
\subsection{The Dirac equations}
\begin{itemize}
		\item Dirac表現
				\begin{equation}
						\gamma^0 
						=
						\begin{pmatrix}
								1_2 & 0\\
								0 & -1_2
						\end{pmatrix},\quad
						\gamma^i
						=
						\begin{pmatrix}
								0 & \gamma^i\\
								-\gamma^i & 0
						\end{pmatrix}
				\end{equation}
				とChiral表現
				\begin{equation}
						\gamma^0
						=
						\begin{pmatrix}
								0 & 1_2\\
								1_2 & 0
						\end{pmatrix},\quad
						\gamma^i
						=
						\begin{pmatrix}
								0 & \gamma^i\\
								-\gamma^i & 0
						\end{pmatrix}
				\end{equation}
				の間のユニタリ変換は
				\begin{equation}
						U =
				\begin{pmatrix}
						1 & 0 & -1 & 0\\
						0 & 1 & 0 & -1\\
						1 & 0 & 1 & 0\\
						0 & 1 & 0 & 1
				\end{pmatrix}
				\end{equation}
				ととり,$U^{\dagger}\gamma^{\mu}U $でDirac表現に移る.
				他の流儀でchiral表現を
				\begin{equation}
						\gamma^i = 
						\begin{pmatrix}
								0 & -\sigma^i\\
								\sigma^i & 0
						\end{pmatrix}
				\end{equation}
				ととる本もある\cite{BB17130464}がこの場合はUnitary行列を
				\begin{equation}
				\begin{pmatrix}
						1 & 0 & 1 & 0\\
						0 & 1 & 0 & 1\\
						1 & 0 & -1 & 0\\
						0 & 1 & 0 & -1
				\end{pmatrix}
				\end{equation}
				ととればよい.こちらのほうが最初に思いつく対角化だと思う.
\end{itemize}
\subsection{Relaticistic narmalization of states}
\begin{itemize}
		\item 公式(3.82) $\delta(g(x)) = \delta(x)/\lvert\dv*{g}{x}\rvert$について.
				$\delta $関数は
				\begin{align}
						\int \dd{x} \delta(x) &= 1, \\
						\int \dd{x} f(x)\delta(x-a) &= f(a)
				\end{align}
				を満たすなにか\footnote{超関数}と思うことが大事.積分すれば$\delta $の引数がゼロになるところの関数の値だけを返す.
				今,$y=g(x)  $とおくと,$\dd{y} = g'(x)\dd{x} $の関係で変数変換して,
				\begin{equation}
						1 = \int\dd{y} \delta(x) = \int \dd{x}g'(x)\delta(g(x))
				\end{equation}
				となる.公式はこの被積分関数を比較した場合を言っている.\footnote{$g'(x)$がゼロになる場合はどうか?また,$g(x) $がゼロになる点が複数存在した場合どうなるか.今考えている場合では単調関数なので,問題にならない.}
		\item 上の注で述べたとおり,単調関数なら問題にならないのだが,そうでない場合を示すことも求められる.

				今,$g(x) $の零点を$\alpha_i \ (i=1, 2, \ldots, n)$とする.これに対して,$g $が微分可能ならば各$i $にたいして区間$[\alpha_i-\varepsilon_i, \alpha_i + \varepsilon_i] $に他の零点を含まないように$\varepsilon_i $を選ぶ.
				任意の(性質のよい)関数$f $をとり,積分
				\begin{equation}
						\int_{-\infty}^{\infty}\dd{x}f(x)\delta(x)
				\end{equation}
				において,$x' = g(x) $の変数変換を考える.このとき,$\varepsilon_i $を区間$[\alpha_i-\varepsilon, \alpha_i+\varepsilon] $で微分係数$\dv*{g}{x} $の符号が変わらないようにとると,積分区間は
				\begin{table}[htbp]
						\centering
						\begin{tabular}{c|c}\hline
								$x $ & $a \to b$\\\hline
								 $x' $ $(\dv*{g(\alpha_i)}{x}>0 )$ & $\alpha_i-\varepsilon_i \to \alpha_i + \varepsilon_i $\\\hline
								 $x' $ $(\dv*{g(\alpha_i)}{x}<0 )$& $\alpha_i+\varepsilon_i \to \alpha_i-\varepsilon_i$\\\hline
						\end{tabular}
				\end{table}

				となる.
			微分係数が負の場合は積分区間を入れ替えたときのマイナスがつく.まとめると,微分係数の絶対値をとって,
			\begin{equation}
					\delta(x) = \sum_{i}\abs{\dv{g(\alpha_i)}{x}}\delta(g(x))
			\end{equation}
			となり,求める式を得る.

		\item この小節ではベクトルの添字とスピノルの添字を区別しなければならない.\footnote{ベクトル添字には$\mu $, $\nu $, $\ldots $あたりのギリシア文字,スピノル添字には$\alpha$, $\beta $, $\ldots $あたりのギリシア文字や$a $, $b $, $\ldots $あたりのアルファベットを使いたい気持ちがある.}Gamma行列$\gamma^{\mu} $は$4\times 4 $行列で$4 $つある.このとき,$\mu $はベクトルの添字である.
				スピノル添字は$a, b,\ldots $や$\alpha, \beta,\ldots $あたりの文字をよく使う.通常の行列$A $を添字をexplicitに$A = (a_{ab}) $と書くことに倣うと$\gamma^{\mu} =(( \gamma^{\mu})_{b}^{a}) $とかける.上付き下付きはあまり気にしなくて良いが,意味はある\cite[Chap.1]{九後1989}, \cite{BB17130464}.
		\item Dirac方程式$\qty(i\gamma^\mu \partial_\mu - m)\psi(x) = 0 $は四成分スピノルの方程式であることに注意せよ.とくに$m $の後ろには$4\times 4 $単位行列が省略されている.また,スピノル添字をexplicitにかけるか?\footnote{正方行列$A =(a_{\alpha\beta})$, $B= (b_{\alpha\beta}) $の積$AB $を添字を用いて表すと$(AB)_{\alpha\gamma} = A_{\alpha\beta}B_{\beta\gamma} $とかけることを思い出すと}
				\begin{equation}
				\qty(i\gamma^{\mu}\partial_{\mu} -m)_{\beta}^{\alpha}\psi(x)^\beta = 0^{\alpha}
				\end{equation}
				である.
\end{itemize}
\subsection{Spin and statistics}

	\section{The Hydrogen Atom and Positronium}
\subsection{The ideal hydrogen atom}
\begin{itemize}
		\item ゼミでは水素原子のSch\"{o}dinger方程式を解けることが求められる.
				エネルギー固有値が離散的で整数でラベル付けられるのはなぜか,など.細かいところは\cite{BN10398292}など,標準的な量子力学の教科書を参照のこと.
		\item Bohr半径$a_0 $は水素原子の基底状態の電子がどれくらいの範囲に広がっているかを特徴づける量といえる.この良い指標として,位置の分散
				\begin{equation}
						\langle r^2 \rangle \coloneqq \int_{0}^{\infty}\dd{r} \int_{0}^{\pi} \dd{\theta}\int_{0}^{2\pi}\dd{\phi}r^2\sin\theta R^{*}(r)r^2R(r)\label{eq:pos_var}
				\end{equation}
				を考える.$R(r) $は動径波動関数でLaguerre微分方程式の解である.基底状態ではこれは$R(r) = Ce^{-r/a_0} $である.ここで$C $は規格化定数.規格化も考えておくこと\footnote{確か,$\pi a_0^2 $だった気がする.}.Eq. \eqref{eq:pos_var}の値は$3a_0^2 $となり,その平方根は大体$a_0 $である.

				もう一つ,よくされる議論として位置の期待値でなく存在確率そのものを考える.球座標で積分するとJacobianのなかの$r^2 $が出てきて,これを含めて\footnote{距離$r $の一点ではなく,半径$r $の球殻上の}存在確率$p(r) $を考えると,
				\begin{equation}
						p(r) \propto \int_{0}^{\infty}\dd{r}r^2R^*(r)R(r)
				\end{equation}
				となる.波動関数および絶対値の二乗は原点で極大だが,これを考えると$r=a_0 $で極大であることがわかる.
		\item 上の1つ目の議論は$v/c \sim \alpha $の導出でも使える.量子力学では速度というより,$p/m $を考えるべきで,$\sqrt{\langle p^2\rangle} $を計算すると$\sqrt{3}\alpha $となって,$\alpha \sim 1/137 $なので,非相対論極限でよさそう,ということ.
\end{itemize}
\subsection{Fine structure and hyperfine structure}
\begin{itemize}
		\item CEPP (4.12)はBiot--Savartの法則を使えばよい.
				電子の静止系からみると,陽子は$-\vec{v} $で運動している.これを電流だと思って,また,線素$\dd{\vec{l}}=-\vec{v}\dd{t} $と時間$t $でパラメトライズすると
				\begin{align}
						\vec{B} &= \frac{1}{4\pi}\int\frac{(-e)(-\vec{v})\dd{t}\times\vec{r}'}{(r')^3}\vec{r'}\delta(\vec{r} - \vec{r'})\\
								&= -\vec{v}\times \frac{-e}{4\pi r^2}\frac{\vec{r}}{r}\\
								&= -\vec{v} \times \vec{E}
				\end{align}
				となる.
		\item Exercise (4.3)では一般の方向へのLorentz変換が必要になる.これは速度$\vec{\beta} $に平行な方向と直行する方向に位置ベクトルを分解して,平行な成分に関してLorentz変換を考えればよい.分解はGram--Schmidtの要領で
				\begin{align}
						\vec{x} &= \vec{x}_{\parallel}+\vec{x}_{\perp},\\
						\vec{x}_{\parallel} &= \frac{\vec{\beta}\cdot\vec{x}}{|\vec{\beta}^2|}\vec{\beta}\\
						\vec{x}_{\perp} &= \vec{x} - \vec{x}_{\parallel}
				\end{align}
				として,Lorentz変換は
				\begin{align}
						t' &= \gamma(t + \beta \cdot \vec{x})\\
						\vec{x}' &= \vec{x}_{\perp} + \gamma(\vec{x}_{\parallel} + \beta t)
				\end{align}
				となる.
				
				結果は演習問題(4.3)のようになる.
				$(S)_{ij}=$
\end{itemize}
\subsection{Positronium}
\begin{itemize}
		\item 換算質量$\mu = m_1m_2/(m_1 + m_2) $はどこから出てくるか.古典論ではなく,量子論でSche\"{o}dinger方程式から導けることを見よ.重心$\vec{R}\coloneqq (m_1\vec{r}_2+m_2\vec{r}_2)/(m_1m_2) $, 相対$\vec{r}\coloneqq \vec{r}_1-\vec{r}_2$と定め,元のHamiltonian $H = -\Delta_1/(2m_1)-\Delta_2/(2m_2)+V(\abs{\vec{r}}) $を$\vec{r} $と$\vec{R} $の微分で書き直せばよい.
		\item 反粒子が粒子に対して反対のパリティーをもつ理由はDirac方程式からわかる.
				今,パリティーのもとで,Dirac方程式が不変であると思う.つまり,
				\begin{equation}
						\qty(i\gamma^{\mu}\partial_{\mu} - m)\psi(t, \vec{x}) = 0\label{eq:parity_dirac}
				\end{equation}
				をパリティー変換したものをプライムをつけて表すと
				\begin{equation}
						\qty(i\gamma^{\mu}\partial_{\mu}-m)\psi'(t, \vec{x})
				\end{equation}
				も満たしていてほしい.

				Eq. \eqref{eq:parity_dirac}を直接parity変換すると
				\begin{equation}
						\qty(i\gamma^{0}\partial_0 - \gamma^{i}\partial_{i}-m)\psi(t, -\vec{x})
				\end{equation}
				であるが,これがDiracであるには$\gamma^0 $と交換して$\gamma^i $と反可換であるものをかければ戻るが,それは$\gamma^0 $である.具体的に行列を書くと
				\begin{equation}
						\gamma^{0} = \begin{pmatrix}
								1 & 0\\
								0 & -1
						\end{pmatrix}
				\end{equation}
				なので,Dirac方程式の上の二成分が通常の粒子,下の二成分が反粒子だったことを思い出せば,パリティーは逆になることがわかる.
		\item photonの荷電共役に関しては場の理論が必要だが,Lagrangianに組み込まれる$\vec{A} $が反転しなければいけない.これがphotonに関連していて,
				1photon状態は$a^{\dagger}\ket{0} $であり,$a^{\dagger} $の方から$-1 $が出る.
				2photonなら
				$a^{\dagger}a^{\dagger}\ket{0} $なので,荷電共役に関して$(-1)^2=+1 $と出る\footnote{状態から$-1 $が出ると思ってしまうと,なぜ和でなく積なのかという疑問が湧く.場の理論を知らないとわからない.}.
		\item ポジトロニウムの崩壊でE1とM1でパリティーが違うのは,それぞれ電気,磁気双極子と磁場のcoupleがHamiltonianに組み込まれることによる.
				電場がvector, 磁場がpseudo vectorなので逆になることがわかる.
				
		\item CEPP (4.43), (4.44)は導出は求められないが,数値を代入して確かめることと解の意味を考えることは求められる.
				$S=0 $のパラポジトロニウムでは$\alpha^5 $に比例するが,この理由は何故か
\end{itemize}

	\section{Charmonium}
この章は実験の説明が多く含まれる.図を読み取ることと,Particle Data Groupなどをみて実験値の確認をすることが求められる.また,必要に応じて引用されている論文を読むと勉強になる.
\subsection{The discovery of the hadrons}
\begin{itemize}
		\item Yukawa potentialをKlein--Gordon方程式のGreen関数として導出できるか.$(\Laplace-m^2)G(\vec{x}) = -\delta(\vec{x}) $で$G(\vec{x}) $を求めよ.方針はFourier変換して,複素積分をする.
				
\end{itemize}

\subsection{Charmonium}
\begin{itemize}
		\item 粒子の崩壊は$e^{-\Gamma t} $に比例して崩壊する.$\Gamma $は崩壊幅\footnote{``width''とだけ書いてあるが,エネルギーの幅である.}で,$\tau \coloneqq 1/\Gamma $は崩壊の特徴的な時間で寿命と呼ばれる.崩壊幅が広いと寿命が短く,狭いと長寿命で安定という関係である.詳しい定義などは\cite[Chap.7]{Peskin:2019iig}でやる.
		\item Annihilation rateが$1/E_{\mathrm{CM}^2}$で減少するのは,場の理論の本を見れば書いてあるが導出は長そう\cite[Chap1, Chap5]{Peskin:1995ev}.ただし,次元解析はできて,散乱断面積$\unit{L^2} = \unit{E^{-2}} $なので.
		\item $\Upsilon $の最も軽いものは$9600\unit{MeV} $とあるが,$9460\unit{MeV} $である.
		\item $\vec{j} = \bar{\psi}\vec{\gamma}\psi $がspin $1 $なのは$3 $成分あるので$3 $次元表現だからである.$P=-1 $なのはベクトルだからで,$C=-1 $は簡単には説明できそうにないが,場の理論の本を見ると$\bar{\psi}\gamma^{\mu}\psi $が$C=-1 $であることが示されている\cite[Chap.3]{Peskin:1995ev}.
		\item CEPP Eq. (5.8)は$\psi' $のdecayが$\psi'\to \gamma + \chi, \ \chi \to \gamma + J/\psi $と二段階で起こることを示す.
				1つ目のdecayで$\chi $は$\gamma $と反跳し運動量を持つ.
				この$\chi $がdecayすることでエネルギー幅に広がりを持つため,
				$\chi $がdecayした低エネルギー側のphotonは広がる\footnote{
						静止系で$E + \delta $程度広がっているとする.
						速度を持った系でのdecayを静止系からみて$(E + \delta)/\sqrt{1-\beta^2} $と広がると思う.
						引用元\cite{Tanenbaum1978}をみるとこれを``Doppler効果''と書いてあり,
						解説記事\cite{1979342}も見つけた.
				}.
				CEPP Fig. 5.6でこれが言及されている.
		\item $J/\psi(1^{--}) $の幅が狭いことはオルソポジトロニウム$(1^{--}) $の寿命が長いことから説明できる.charmoniumでもポジトロニウムでも寿命に関して
				\begin{equation}
						\frac{(1^{--})}{(0^{-+})} \sim 1000
				\end{equation}
				であることから比較できる.\footnote{幅で計算するか,寿命で計算するかに注意しないと,幅と寿命は逆数の関係なので逆になる.}
\end{itemize}
\subsection{The light meson}
\begin{itemize}
		\item $\pi^0 \to 2\gamma $の反応は宇宙物理でよく使うプロセス.出てきたphotonを地上で捉え,そのスペクトラムを用いて,どのプロセスか,どのようにpionがあるのかを調べたりするそうである.
		\item strangenessの保存や$(u, d, s) $の$\SU(3) $は\cite{BB03663366}などに詳しい.逆にこの本は,物理としてこれらを知っている人向けの本なので,覚えておくとこの本を読むときに役立つ.
		\item mesonというのは(ほとんど)$q\bar{q} $のペアなので\footnote{$q $はquark.},
				spin $1/2 $のparticle-antiparticle boundstateとして,ポジトロニウム($e^{+}e^{-})$のmass spectrumの構造を持つ.例えば,$1^{--} $状態の$\rho.\ \omega,\ \phi $がphotonを出して$0^{-+} $状態の$\pi^{0} $にdecayするなど.
		\item strangenessは$s $に$S=\color{red}{-}\color{black}{1}  $,$\textcolor{red}{\bar{\color{black}{s}}} $に$S = \color{red}{+}\color{black}{1} $を与える.
		\item Isospinは二成分,例えば$(p, n) $を$p \sim \ket{\uparrow}(1, 0) $, $n\sim \ket{\downarrow} \sim (0, 1)$と思って.spinと同じように扱う.もともとは原子核の$(p, n) $についてのアイデアだったが,これをquark $(u, s) $にも応用できる.さらにstrangeも加えて,$\SU(3) $対称性を考えることが有用である.
		\item $\SO(3) $と$\SU(2) $が同型という記述があるが嘘.数学の本を見ると$\SU(2)/\Z_2 \cong \SO(3)$という同型があって,もし$\SU(2) \cong \SO(3) $ならば推移律より$\SU(2) \cong\SU(2)/\Z_2 $だが,この間の同型写像は構成できない.

		\item CG(Clebsh--Gordan)係数に依る素粒子過程への制限は\cite[Chap.4]{BB03663366}が参考になる.量子力学の基本的な教科書にも記述はあると思う.CG係数は$\ket{J, M} $という状態を持った粒子が$\ket{j_1, m_1} $と$\ket{j_2, m_2} $という状態の$2 $粒子にdecayするとき,decayの仕方により
				\begin{equation}
						\ket{J, M} = \sum (\bra{j_1, m_1}\otimes\bra{j_2, m_2})\ket{J, M} \ket{j_1, m_1}\otimes\ket{j_2, m_2}
				\end{equation}
				となるが,その一つに対する確率振幅を表す$ (\bra{j_1, m_1}\otimes\bra{j_2, m_2})\ket{J, M}$のことである.
		\item G-parityの定義$G = C\e^{\i\pi I_2} $の$I_2 $は$2\times 2 $単位行列ではなくIsospinの第二成分の意味である.スピンで言えば$J_2 $のことである.今,pionを考えているので,Isospin $1 $表現でIsospinの(群としての)作用は通常の空間回転と同じで
				\begin{equation}
						G = CR_2(\pi)  = C\operatorname{diag}(-1, 1, -1)
				\end{equation}
				とy軸周りの$\pi $回転となる.
		\item CEPP Eq. (5.28)からどのようにCEPP Eq. (5.29)が導かれるかだが,
				通常pionと行ったときは$\SU(2) $の随伴表現の$3 $成分の固有状態をいうようで
				\footnote{最初,$3 $成分を対角化する表現だと思って混乱した.}\cite{Bransden2015},表現のとり方はCEPP Eq. (1.59)のとおりである.
				今,Isospin第三成分は
				\begin{equation}
						I_3 = \begin{pmatrix}
								0 & -i & 0\\
								i & 0 & 0\\
								0 & 0 & 0
						\end{pmatrix}
				\end{equation}
				と取っていて,pionは
				\begin{equation}
						\ket{\pi^{\pm}} = 
						\begin{pmatrix}
								1\\\pm \i\\0
						\end{pmatrix},\quad
						\ket{\pi^{0}} = 
						\begin{pmatrix}
								0\\0\\1
						\end{pmatrix}
				\end{equation}
				と表現されている.これがわかれば目標の式変形は明らか.
		\item $q\bar{q} $ タイプのmesonのエネルギーの限界値は$1300\unit{\textcolor{red}{M}eV} $である.
\end{itemize}
\subsection{The heavy meson}
\begin{itemize}
		\item 何を比較して\footnote{クオークの重さを直接比べよ,と言われたが,この節のタイトルからしてmesonで比べるべきでは?と思ったが.}重い,軽いと言っているのか.PDGでクオークについての欄があるので,調べること.
\end{itemize}

%	\nocite{1130282270643311616}
%	\bibliography{engbook}
%	\bibliographystyle{ytamsalpha}
	%\bibliographystyle{ytamsbeta}
	\bibliography{CEPP, booklist}
	\bibliographystyle{jalpha}
\end{document}

