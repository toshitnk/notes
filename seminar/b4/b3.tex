\documentclass[dvipdfmx, a4paper]{jsarticle}
\usepackage[utf8]{inputenc}
\usepackage[top=10truemm, bottom=20truemm, left=15truemm, right=15truemm]{geometry} % mergin
\renewcommand{\headfont}{\bfseries}

% graphics
\usepackage{graphicx}
\usepackage{here}

% link

\usepackage{url}
\usepackage[dvipdfmx, linktocpage]{hyperref} 
\usepackage{xcolor}
\usepackage{pxjahyper}
\hypersetup{
	colorlinks=true,
	citecolor=blue,
	linkcolor=teal,
	urlcolor=orange,
}

% math

\usepackage{amsmath, amssymb} 
\usepackage{physics}
\usepackage{mathrsfs}
\usepackage{mathtools}

% theoremstyle
\usepackage{amsthm}
\newtheoremstyle{break}
{\topsep}{\topsep}%
{}{}%
{\bfseries}{}%
{\newline}{}%
\theoremstyle{break}
\newtheorem{thm}{Theorem}[section]
\newtheorem{defn}[thm]{Definition}
\newtheorem{eg}[thm]{Example}
\newtheorem{cl}[thm]{Claim}
\newtheorem{cor}[thm]{Corollary}
\newtheorem{fact}[thm]{Fact}
\newtheorem{rem}[thm]{Remark}
\newtheorem{prob}{Problem}[section]

\makeatletter
\newenvironment{pr}[1][\proofnam]{\par
\topsep6\p@\@plus6\p@ \trivlist
\item[\hskip\labelsep{\itshape #1}\@addpunct{\bfseries}]\ignorespaces
}{%
\endtrivlist
}
\newcommand{\proofnam}{\underline{Derivation.}}
\makeatother


% my command

\newcommand{\R}{\mathbb{R}}
\newcommand{\C}{\mathbb{C}}
\newcommand{\Z}{\mathbb{Z}}

\newcommand{\eq}[1]{Eq. \eqref{#1}}
\newcommand{\theorem}[1]{Thm. \ref{#1}}
\newcommand{\definition}[1]{Def. \ref{#1}}
\newcommand{\proposition}[1]{Prop. \ref{#1}}
\newcommand{\example}[1]{e.g.\ref{#1}}
\newcommand{\claim}[1]{Cl. \ref{#1}}
\newcommand{\corolary}[1]{Cor. \ref{#1}}
\newcommand{\remark}[1]{Rem. \ref{#1}}
\newcommand{\problem}[1]{Prob. \ref{#1}}

\renewcommand{\O}{\mathcal{O}}

\newcommand{\su}{\mathfrak{su}}
\newcommand{\SU}{\mathrm{SU}}
\newcommand{\g}{\mathfrak{g}}
\newcommand{\so}{\mathfrak{su}}
\newcommand{\SO}{\mathrm{SO}}

% number

%\makeatletter
%\@addtoreset{equation}{section}
%\makeatother
%\numberwithin{equation}{section}
%\renewcommand{\thefootnote}{\roman{footnote}.}
%\renewcommand{\appendixname}{Appendix }


%English

\renewcommand{\tablename}{Tab. }
\renewcommand{\figurename}{Fig. }
\renewcommand{\refname}{References}


\title{B4セミナー}
\author{Toshiya Tanaka}
\date{\today}

\begin{document}
	\maketitle
	\section{Introduction}
	\begin{itemize}
			\item この文書では教科書の誤植や説明不足だと感じたことをまとめています.
			\item 公式の誤植リストは\href{https://www.slac.stanford.edu/~mpeskin/ConceptsBook.html}{ここ}にあります\footnote{数式がtex記法でベタ打ちなのでつらい.MathJax使ってほしかった.}.ここにあるものも書きますが.公式文書を書くことをおすすめします.
			\item 「\href{https://www.ms.u-tokyo.ac.jp/~yasuyuki/sem.htm}{セミナーの準備の仕方について(河東泰之)}」はゼミ準備の心構えとして有名な文書なので一読するとよいでしょう.
					特に
					\begin{quote}
							黙って「何々である」とか,``It is easy to see...", ``We may assume that...", ``It is enough to show..."などと書いてあるのはすべて,なぜなのか徹底的に考えなくてはいけません.「本に書いてあるから」とか「先生がそう言うから」などの理由で,なんとなく分かったような気になるのは絶対にアウトです.そういうところは「なぜですか」と聞かれるに決まっているんですから,どうきかれてもすぐに答えられるように準備をしておく必要があります.
					\end{quote}
					ということは意識すべきです.
			\item この文書についても同じで,「この文書に書いてあるから」というのは根拠にはなりません.\footnote{誤植を訂正しているはずが,さらに誤植だったということは大いにありえます.}
	\end{itemize}

	\section{Symmetries of Space-Time}
\subsection{Relativistic particle kinematics}
\subsection{Natural units}
\begin{itemize}
		\item SIと自然単位系の変換を実際に計算すること.

		\item Eq. (2.30)は冪が間違い.正しくは$10^{-23} $.
\end{itemize}
\subsection{A little theory of discrete groups}
\begin{itemize}
	\item 物理では対称性が大事で,離散対称性と連続対称性に分けられる.前者の例はparityや時間反転に関する対称性で,後者の例は並進や回転などに関する対称性である.
	これらの操作は群(group)という構造を成し,それを古典力学や量子力学の理論は,理論の(線形)空間に作用させて不変であることを要請する.
	このようなことを調べる数学の分野を表現論という.物理でよくいう「群論」は群の表現論である.群の演算を線形空間の行列で表現するときに,量子力学ではこの線形空間が状態空間(Hilbert空間)で行列が物理量である.表現論は物理の本だと\cite{九後1989}や\cite{BN10398292}の7章などが標準的で,数学の本だと線形代数と接続がよいものとして\cite{BC13565134}がある.
		\item 対称群$\Pi_3 $の元が数学でよくある置換でないことに注意.$\Pi_3 $は元自身ではなく文字の場所にラベルをつけている.
\end{itemize}
\subsection{A little theory of continuous groups}
\begin{itemize}
		\item 対称性と保存則の関係は\cite[Chap.10, Sec.4]{BB17130464}に詳しい.
		\item 連続対称性に関する操作はLie群という構造を成し,Lie群は単位元近傍の微小変換を調べれば大体わかることが知られている.
				Lie群の単位元近傍は線形空間になっていて,行列の交換子で閉じる.この構造をLie環という.
				物理で重要な例は$\su(2) $というLie環\footnote{Lie群$G$に対し,Lie環はフラクトゥール小文字で$\g$と書くことがある.物理屋さんはあまりこの違いを気にしないので両方アルファベット大文字で書いてしまう文献もしばしばある.}で,量子力学でよくやる各運動量演算子$[J_j, J_k] = \i\epsilon_{jkl}J_l $がそれである.
				角運動量が回転に対する保存量だったことを思い出すと,$J_i$は微小回転に対応している.
				表現の次元はspinの大きさで勘定することがある.
				角運動量の大きさ$j $に対し,行列のサイズは$2j + 1 $になるので,.$2j+1 $次元表現をspin $j $表現と言ったりする.例えば,$1 $次元表現はspin $0 $表現で,$2 $次元表現はspin $1/2 $表現で$J_i $はPauli行列たちである.このあたりの話は\cite[Chap.3]{BB03663366}, \cite{BN10398292}あたりが詳しい.
		\item Eq. (2.55)は\href{https://en.wikipedia.org/wiki/Baker%E2%80%93Campbell%E2%80%93Hausdorff_formula}{BCH formula}から示せると言われる.確かに左辺を計算して,BCH formulaからLie bracketの入れ子の和の指数関数でかけることがいえ,Lie bracketで閉じているものがLie環なので左辺は群の元なのだが,それが$\gamma $と同じということ言えないと思う.
		\end{itemize}
\subsection{Discrete space-time symmetries}
\begin{itemize}
		\item Eq. (2.71)の$P $がかかって$+ $を吐くか$- $を吐くかは粒子や場の性質である.$- $を吐くほうを擬スカラー/ベクトルなどという.
\end{itemize}

	\section{Relativistic Wave Equations}
\subsection{The Klein--Gordon equation}
\subsection{Fields and particles}
\subsection{Maxwell's equations}
\subsection{The Dirac equations}
\begin{itemize}
		\item Dirac表現
				\begin{equation}
						\gamma^0 
						=
						\begin{pmatrix}
								1_2 & 0\\
								0 & -1_2
						\end{pmatrix},\quad
						\gamma^i
						=
						\begin{pmatrix}
								0 & \gamma^i\\
								-\gamma^i & 0
						\end{pmatrix}
				\end{equation}
				とChiral表現
				\begin{equation}
						\gamma^0
						=
						\begin{pmatrix}
								0 & 1_2\\
								1_2 & 0
						\end{pmatrix},\quad
						\gamma^i
						=
						\begin{pmatrix}
								0 & \gamma^i\\
								-\gamma^i & 0
						\end{pmatrix}
				\end{equation}
				の間のユニタリ変換は
				\begin{equation}
						U =
				\begin{pmatrix}
						1 & 0 & -1 & 0\\
						0 & 1 & 0 & -1\\
						1 & 0 & 1 & 0\\
						0 & 1 & 0 & 1
				\end{pmatrix}
				\end{equation}
				ととり,$U^{\dagger}\gamma^{\mu}U $でDirac表現に移る.
				他の流儀でchiral表現を
				\begin{equation}
						\gamma^i = 
						\begin{pmatrix}
								0 & -\sigma^i\\
								\sigma^i & 0
						\end{pmatrix}
				\end{equation}
				ととる本もある\cite{BB17130464}がこの場合はUnitary行列を
				\begin{equation}
				\begin{pmatrix}
						1 & 0 & 1 & 0\\
						0 & 1 & 0 & 1\\
						1 & 0 & -1 & 0\\
						0 & 1 & 0 & -1
				\end{pmatrix}
				\end{equation}
				ととればよい.こちらのほうが最初に思いつく対角化だと思う.
\end{itemize}
\subsection{Relaticistic narmalization of states}
\begin{itemize}
		\item 公式(3.82) $\delta(g(x)) = \delta(x)/\lvert\dv*{g}{x}\rvert$について.
				$\delta $関数は
				\begin{align}
						\int \dd{x} \delta(x) &= 1, \\
						\int \dd{x} f(x)\delta(x-a) &= f(a)
				\end{align}
				を満たすなにか\footnote{超関数}と思うことが大事.積分すれば$\delta $の引数がゼロになるところの関数の値だけを返す.
				今,$y=g(x)  $とおくと,$\dd{y} = g'(x)\dd{x} $の関係で変数変換して,
				\begin{equation}
						1 = \int\dd{y} \delta(x) = \int \dd{x}g'(x)\delta(g(x))
				\end{equation}
				となる.公式はこの被積分関数を比較した場合を言っている.\footnote{$g'(x)$がゼロになる場合はどうか?また,$g(x) $がゼロになる点が複数存在した場合どうなるか.今考えている場合では単調関数なので,問題にならない.}
		\item 上の注で述べたとおり,単調関数なら問題にならないのだが,そうでない場合を示すことも求められる.

				今,$g(x) $の零点を$\alpha_i \ (i=1, 2, \ldots, n)$とする.これに対して,$g $が微分可能ならば各$i $にたいして区間$[\alpha_i-\varepsilon_i, \alpha_i + \varepsilon_i] $に他の零点を含まないように$\varepsilon_i $を選ぶ.
				任意の(性質のよい)関数$f $をとり,積分
				\begin{equation}
						\int_{-\infty}^{\infty}\dd{x}f(x)\delta(x)
				\end{equation}
				において,$x' = g(x) $の変数変換を考える.このとき,$\varepsilon_i $を区間$[\alpha_i-\varepsilon, \alpha_i+\varepsilon] $で微分係数$\dv*{g}{x} $の符号が変わらないようにとると,積分区間は
				\begin{table}[htbp]
						\centering
						\begin{tabular}{c|c}\hline
								$x $ & $a \to b$\\\hline
								 $x' $ $(\dv*{g(\alpha_i)}{x}>0 )$ & $\alpha_i-\varepsilon_i \to \alpha_i + \varepsilon_i $\\\hline
								 $x' $ $(\dv*{g(\alpha_i)}{x}<0 )$& $\alpha_i+\varepsilon_i \to \alpha_i-\varepsilon_i$\\\hline
						\end{tabular}
				\end{table}

				となる.
			微分係数が負の場合は積分区間を入れ替えたときのマイナスがつく.まとめると,微分係数の絶対値をとって,
			\begin{equation}
					\delta(x) = \sum_{i}\abs{\dv{g(\alpha_i)}{x}}\delta(g(x))
			\end{equation}
			となり,求める式を得る.

		\item この小節ではベクトルの添字とスピノルの添字を区別しなければならない.\footnote{ベクトル添字には$\mu $, $\nu $, $\ldots $あたりのギリシア文字,スピノル添字には$\alpha$, $\beta $, $\ldots $あたりのギリシア文字や$a $, $b $, $\ldots $あたりのアルファベットを使いたい気持ちがある.}Gamma行列$\gamma^{\mu} $は$4\times 4 $行列で$4 $つある.このとき,$\mu $はベクトルの添字である.
				スピノル添字は$a, b,\ldots $や$\alpha, \beta,\ldots $あたりの文字をよく使う.通常の行列$A $を添字をexplicitに$A = (a_{ab}) $と書くことに倣うと$\gamma^{\mu} =(( \gamma^{\mu})_{b}^{a}) $とかける.上付き下付きはあまり気にしなくて良いが,意味はある\cite[Chap.1]{九後1989}, \cite{BB17130464}.
		\item Dirac方程式$\qty(i\gamma^\mu \partial_\mu - m)\psi(x) = 0 $は四成分スピノルの方程式であることに注意せよ.とくに$m $の後ろには$4\times 4 $単位行列が省略されている.また,スピノル添字をexplicitにかけるか?\footnote{正方行列$A =(a_{\alpha\beta})$, $B= (b_{\alpha\beta}) $の積$AB $を添字を用いて表すと$(AB)_{\alpha\gamma} = A_{\alpha\beta}B_{\beta\gamma} $とかけることを思い出すと}
				\begin{equation}
				\qty(i\gamma^{\mu}\partial_{\mu} -m)_{\beta}^{\alpha}\psi(x)^\beta = 0^{\alpha}
				\end{equation}
				である.
\end{itemize}
\subsection{Spin and statistics}

	\section{The Hydrogen Atom and Positronium}
\subsection{The ideal hydrogen atom}
\subsection{Fine structure and hyperfine structure}
\begin{itemize}
		\item Eq. (4.12)はBiot--Savartの法則を使えばよい.
				電子の静止系からみると,陽子は$-\vec{v} $で運動している.これを電流だと思って,また,線素$\dd{\vec{l}}=-\vec{v}\dd{t} $と時間$t $でパラメトライズすると
				\begin{align}
						\vec{B} &= \frac{1}{4\pi}\int\frac{(-e)(-\vec{v})\dd{t}\times\vec{r}'}{(r')^3}\vec{r'}\delta(\vec{r} - \vec{r'})\\
								&= -\vec{v}\times \frac{-e}{4\pi r^2}\frac{\vec{r}}{r}\\
								&= -\vec{v} \times \vec{E}
				\end{align}
				となる.
		\item Exercise (4.3)では一般の方向へのLorentz変換が必要になる.これは速度$\vec{beta} $に平行な方向と直行する方向に位置ベクトルを分解して,平行な成分に関してLorentz変換を考えればよい.分解はGram--Schmidtの要領で
				\begin{align}
						\vec{x} &= \vec{x}_{\parallel}+\vec{x}_{\perp},\\
						\vec{x}_{\parallel} &= \frac{\vec{\beta}\cdot\vec{x}}{|\vec{\beta}^2|}\vec{\beta}\\
						\vec{x}_{\perp} &= \vec{x} - \vec{x}_{\parallel}
				\end{align}
				として,Lorentz変換は
				\begin{align}
						t' &= \gamma(t + \beta \cdot \vec{x})\\
						\vec{x}' &= \vec{x}_{\perp} + \gamma(\vec{x}_{\parallel} + \beta t)
				\end{align}
				となる.
\end{itemize}
\subsection{Positronium}

	\nocite{BB17130464}
%	\nocite{1130282270643311616}
%	\bibliography{engbook}
%	\bibliographystyle{ytamsalpha}
	%\bibliographystyle{ytamsbeta}
	\bibliography{booklist}
	\bibliographystyle{jalpha}
\end{document}

