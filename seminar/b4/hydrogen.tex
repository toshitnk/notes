\section{The Hydrogen Atom and Positronium}
\subsection{The ideal hydrogen atom}
\begin{itemize}
		\item ゼミでは水素原子のSch\"{o}dinger方程式を解けることが求められる.
				エネルギー固有値が離散的で整数でラベル付けられるのはなぜか,など.細かいところは\cite{BN10398292}など,標準的な量子力学の教科書を参照のこと.
		\item Bohr半径$a_0 $は水素原子の基底状態の電子がどれくらいの範囲に広がっているかを特徴づける量といえる.この良い指標として,位置の分散
				\begin{equation}
						\langle r^2 \rangle \coloneqq \int_{0}^{\infty}\dd{r} \int_{0}^{\pi} \dd{\theta}\int_{0}^{2\pi}\dd{\phi}r^2\sin\theta R^{*}(r)r^2R(r)\label{eq:pos_var}
				\end{equation}
				を考える.$R(r) $は動径波動関数でLaguerre微分方程式の解である.基底状態ではこれは$R(r) = Ce^{-r/a_0} $である.ここで$C $は規格化定数.規格化も考えておくこと\footnote{確か,$\pi a_0^2 $だった気がする.}.Eq. \eqref{eq:pos_var}の値は$3a_0^2 $となり,その平方根は大体$a_0 $である.

				もう一つ,よくされる議論として位置の期待値でなく存在確率そのものを考える.球座標で積分するとJacobianのなかの$r^2 $が出てきて,これを含めて\footnote{距離$r $の一点ではなく,半径$r $の球殻上の}存在確率$p(r) $を考えると,
				\begin{equation}
						p(r) \propto \int_{0}^{\infty}\dd{r}r^2R^*(r)R(r)
				\end{equation}
				となる.波動関数および絶対値の二乗は原点で極大だが,これを考えると$r=a_0 $で極大であることがわかる.
		\item 上の1つ目の議論は$v/c \sim \alpha $の導出でも使える.量子力学では速度というより,$p/m $を考えるべきで,$\sqrt{\langle p^2\rangle} $を計算すると$\sqrt{3}\alpha $となって,$\alpha \sim 1/137 $なので,非相対論極限でよさそう,ということ.
\end{itemize}
\subsection{Fine structure and hyperfine structure}
\begin{itemize}
		\item CEPP (4.12)はBiot--Savartの法則を使えばよい.
				電子の静止系からみると,陽子は$-\vec{v} $で運動している.これを電流だと思って,また,線素$\dd{\vec{l}}=-\vec{v}\dd{t} $と時間$t $でパラメトライズすると
				\begin{align}
						\vec{B} &= \frac{1}{4\pi}\int\frac{(-e)(-\vec{v})\dd{t}\times\vec{r}'}{(r')^3}\vec{r'}\delta(\vec{r} - \vec{r'})\\
								&= -\vec{v}\times \frac{-e}{4\pi r^2}\frac{\vec{r}}{r}\\
								&= -\vec{v} \times \vec{E}
				\end{align}
				となる.
		\item Exercise (4.3)では一般の方向へのLorentz変換が必要になる.これは速度$\vec{\beta} $に平行な方向と直行する方向に位置ベクトルを分解して,平行な成分に関してLorentz変換を考えればよい.分解はGram--Schmidtの要領で
				\begin{align}
						\vec{x} &= \vec{x}_{\parallel}+\vec{x}_{\perp},\\
						\vec{x}_{\parallel} &= \frac{\vec{\beta}\cdot\vec{x}}{|\vec{\beta}^2|}\vec{\beta}\\
						\vec{x}_{\perp} &= \vec{x} - \vec{x}_{\parallel}
				\end{align}
				として,Lorentz変換は
				\begin{align}
						t' &= \gamma(t + \beta \cdot \vec{x})\\
						\vec{x}' &= \vec{x}_{\perp} + \gamma(\vec{x}_{\parallel} + \beta t)
				\end{align}
				となる.
				
				結果は演習問題(4.3)のようになる.
				$(S)_{ij}=$
\end{itemize}
\subsection{Positronium}
ポジトロニウムは京都大学の課題演習の題材にあるようで,\href{https://www-he.scphys.kyoto-u.ac.jp/gakubu/a1a2.html}{課題演習A1・A2}や\href{https://www-he.scphys.kyoto-u.ac.jp/gakubu/p1p2.html}{課題研究P1・P2}にある資料が参考になる.
\begin{itemize}
		\item 換算質量$\mu = m_1m_2/(m_1 + m_2) $はどこから出てくるか.古典論ではなく,量子論でSche\"{o}dinger方程式から導けることを見よ.重心$\vec{R}\coloneqq (m_1\vec{r}_2+m_2\vec{r}_2)/(m_1m_2) $, 相対$\vec{r}\coloneqq \vec{r}_1-\vec{r}_2$と定め,元のHamiltonian $H = -\Delta_1/(2m_1)-\Delta_2/(2m_2)+V(\abs{\vec{r}}) $を$\vec{r} $と$\vec{R} $の微分で書き直せばよい.
		\item 反粒子が粒子に対して反対のパリティーをもつ理由はDirac方程式からわかる.
				今,パリティーのもとで,Dirac方程式が不変であると思う.つまり,
				\begin{equation}
						\qty(i\gamma^{\mu}\partial_{\mu} - m)\psi(t, \vec{x}) = 0\label{eq:parity_dirac}
				\end{equation}
				をパリティー変換したものをプライムをつけて表すと
				\begin{equation}
						\qty(i\gamma^{\mu}\partial_{\mu}-m)\psi'(t, \vec{x})
				\end{equation}
				も満たしていてほしい.

				Eq. \eqref{eq:parity_dirac}を直接parity変換すると
				\begin{equation}
						\qty(i\gamma^{0}\partial_0 - \gamma^{i}\partial_{i}-m)\psi(t, -\vec{x})
				\end{equation}
				であるが,これがDiracであるには$\gamma^0 $と交換して$\gamma^i $と反可換であるものをかければ戻るが,それは$\gamma^0 $である.具体的に行列を書くと
				\begin{equation}
						\gamma^{0} = \begin{pmatrix}
								1 & 0\\
								0 & -1
						\end{pmatrix}
				\end{equation}
				なので,Dirac方程式の上の二成分が通常の粒子,下の二成分が反粒子だったことを思い出せば,パリティーは逆になることがわかる.
		\item photonの荷電共役に関しては場の理論が必要だが,Lagrangianに組み込まれる$\vec{A} $が反転しなければいけない.これがphotonに関連していて,
				1photon状態は$a^{\dagger}\ket{0} $であり,$a^{\dagger} $の方から$-1 $が出る.
				2photonなら
				$a^{\dagger}a^{\dagger}\ket{0} $なので,荷電共役に関して$(-1)^2=+1 $と出る\footnote{状態から$-1 $が出ると思ってしまうと,なぜ和でなく積なのかという疑問が湧く.場の理論を知らないとわからない.}.
		\item ポジトロニウムの崩壊でE1とM1でパリティーが違うのは,それぞれ電気,磁気双極子と磁場のcoupleがHamiltonianに組み込まれることによる.
				電場がvector, 磁場がpseudo vectorなので逆になることがわかる.
				
		\item CEPP (4.43), (4.44)は導出は求められないが,数値を代入して確かめることと解の意味を考えることは求められる.
		$S=0 $のパラポジトロニウムでは$\alpha^5 $に比例するが,これは何故か
		\footnote{
				説明されたが,よくわからなかった.
				散乱断面積$\sigma = (4\pi/v)(\alpha/m_{e})^2$で$\alpha/m_e = e^2/(4\pi m) $は古典半径で面積を表す.
				$\alpha^3 $は波動関数の規格化$1/\sqrt{\pi a^3}, (a\coloneqq 2/(m\alpha)) $から出てきそうである.
				3 photon decayはphotonが増える分$\alpha $が掛かりそうな気がする.らしい.
		}?
\end{itemize}
