\section{The Hydrogen Atom and Positronium}
\subsection{The ideal hydrogen atom}
\subsection{Fine structure and hyperfine structure}
\begin{itemize}
		\item Eq. (4.12)はBiot--Savartの法則を使えばよい.
				電子の静止系からみると,陽子は$-\vec{v} $で運動している.これを電流だと思って,また,線素$\dd{\vec{l}}=-\vec{v}\dd{t} $と時間$t $でパラメトライズすると
				\begin{align}
						\vec{B} &= \frac{1}{4\pi}\int\frac{(-e)(-\vec{v})\dd{t}\times\vec{r}'}{(r')^3}\vec{r'}\delta(\vec{r} - \vec{r'})\\
								&= -\vec{v}\times \frac{-e}{4\pi r^2}\frac{\vec{r}}{r}\\
								&= -\vec{v} \times \vec{E}
				\end{align}
				となる.
		\item Exercise (4.3)では一般の方向へのLorentz変換が必要になる.これは速度$\vec{beta} $に平行な方向と直行する方向に位置ベクトルを分解して,平行な成分に関してLorentz変換を考えればよい.分解はGram--Schmidtの要領で
				\begin{align}
						\vec{x} &= \vec{x}_{\parallel}+\vec{x}_{\perp},\\
						\vec{x}_{\parallel} &= \frac{\vec{\beta}\cdot\vec{x}}{|\vec{\beta}^2|}\vec{\beta}\\
						\vec{x}_{\perp} &= \vec{x} - \vec{x}_{\parallel}
				\end{align}
				として,Lorentz変換は
				\begin{align}
						t' &= \gamma(t + \beta \cdot \vec{x})\\
						\vec{x}' &= \vec{x}_{\perp} + \gamma(\vec{x}_{\parallel} + \beta t)
				\end{align}
				となる.
\end{itemize}
\subsection{Positronium}
