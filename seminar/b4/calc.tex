\section{Tools for calculation}
\subsection{Observables in particle expermiments}
\subsection{Master formulae for partial width and cross sections}
\subsection{Phase space}
\subsection{Examples: $\pi^{+}\pi^{-}$ scattering at the $\rho $ resonance}
\begin{itemize}
		\item Breit--WignerのFourier変換は留数積分するだけ.Errataに$-i $とあるが,$i $で良いように思う.
		\item $\pi^{+}\pi^{-}\to \rho\to \pi^{+}\pi^{-} $の$\mathcal{M} $は$\pi^{+}\pi^{-}\to \rho$と$\rho\to \pi^{+}\pi^{-}  $の$\mathcal{M} $の積ではなく,Breit--Wigner分も積に寄与している.e.g., CEPP Eq. (7.65). これはCEPP Eq. (7.44)のdiagramでvertexと$\rho^0 $の線分にそれぞれ対応していると思うとよい.
		\item CEPP Eq. (7.53)の負号はMinkowski計量の空間成分のマイナスで変更は実ベクトルと思うと良い.
		\item 最後の$a/(\pi(x^2+a^2)) $の積分は$a\to 0 $で$\delta(x) $になる函数列と思える.だから,積分して$1 $だけでなく,適当な函数$f(x) $をかけて積分すると$f(0) $になる性質も確認するとよい.他にもdelta函数に「収束」する函数列はあって,\url{https://manabitimes.jp/math/2304}などが参考になる.
\end{itemize}
