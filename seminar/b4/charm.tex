\section{Charmonium}
この章は実験の説明が多く含まれる.図を読み取ることと,Particle Data Groupなどをみて実験値の確認をすることが求められる.また,必要に応じて引用されている論文を読むと勉強になる.
\subsection{The discovery of the hadrons}
\begin{itemize}
		\item Yukawa potentialをKlein--Gordon方程式のGreen関数として導出できるか.$(\Laplace-m^2)G(\vec{x}) = -\delta(\vec{x}) $で$G(\vec{x}) $を求めよ.方針はFourier変換して,複素積分をする.
				
\end{itemize}

\subsection{Charmonium}
\begin{itemize}
		\item 粒子の崩壊は$e^{-\Gamma t} $に比例して崩壊する.$\Gamma $は崩壊幅\footnote{``width''とだけ書いてあるが,エネルギーの幅である.}で,$\tau \coloneqq 1/\Gamma $は崩壊の特徴的な時間で寿命と呼ばれる.崩壊幅が広いと寿命が短く,狭いと長寿命で安定という関係である.詳しい定義などは\cite[Chap.7]{Peskin:2019iig}でやる.
		\item Annihilation rateが$1/E_{\mathrm{CM}^2}$で減少するのは,場の理論の本を見れば書いてあるが導出は長そう\cite[Chap1, Chap5]{Peskin:1995ev}.ただし,次元解析はできて,散乱断面積$\unit{L^2} = \unit{E^{-2}} $なので.
		\item $\Upsilon $の最も軽いものは$9600\unit{MeV} $とあるが,$9460\unit{MeV} $である.
		\item $\vec{j} = \bar{\psi}\vec{\gamma}\psi $がspin $1 $なのは$3 $成分あるので$3 $次元表現だからである.$P=-1 $なのはベクトルだからで,$C=-1 $は簡単には説明できそうにないが,場の理論の本を見ると$\bar{\psi}\gamma^{\mu}\psi $が$C=-1 $であることが示されている\cite[Chap.3]{Peskin:1995ev}.
		\item CEPP Eq. (5.8)は$\psi' $のdecayが$\psi'\to \gamma + \chi, \ \chi \to \gamma + J/\psi $と二段階で起こることを示す.
				1つ目のdecayで$\chi $は$\gamma $と反跳し運動量を持つ.
				この$\chi $がdecayすることでエネルギー幅に広がりを持つため,
				$\chi $がdecayした低エネルギー側のphotonは広がる\footnote{
						静止系で$E + \delta $程度広がっているとする.
						速度を持った系でのdecayを静止系からみて$(E + \delta)/\sqrt{1-\beta^2} $と広がると思う.
						引用元\cite{Tanenbaum1978}をみるとこれを``Doppler効果''と書いてあり,
						解説記事\cite{1979342}も見つけた.
				}.
				CEPP Fig. 5.6でこれが言及されている.
		\item $J/\psi(1^{--}) $の幅が狭いことはオルソポジトロニウム$(1^{--}) $の寿命が長いことから説明できる.charmoniumでもポジトロニウムでも寿命に関して
				\begin{equation}
						\frac{(1^{--})}{(0^{-+})} \sim 1000
				\end{equation}
				であることから比較できる.\footnote{幅で計算するか,寿命で計算するかに注意しないと,幅と寿命は逆数の関係なので逆になる.}
\end{itemize}
\subsection{The light meson}
\begin{itemize}
		\item $\pi^0 \to 2\gamma $の反応は宇宙物理でよく使うプロセス.出てきたphotonを地上で捉え,そのスペクトラムを用いて,どのプロセスか,どのようにpionがあるのかを調べたりするそうである.
		\item strangenessの保存や$(u, d, s) $の$\SU(3) $は\cite{BB03663366}などに詳しい.逆にこの本は,物理としてこれらを知っている人向けの本なので,覚えておくとこの本を読むときに役立つ.
		\item mesonというのは(ほとんど)$q\bar{q} $のペアなので\footnote{$q $はquark.},
				spin $1/2 $のparticle-antiparticle boundstateとして,ポジトロニウム($e^{+}e^{-})$のmass spectrumの構造を持つ.例えば,$1^{--} $状態の$\rho.\ \omega,\ \phi $がphotonを出して$0^{-+} $状態の$\pi^{0} $にdecayするなど.
		\item strangenessは$s $に$S=\color{red}{-}\color{black}{1}  $,$\textcolor{red}{\bar{\color{black}{s}}} $に$S = \color{red}{+}\color{black}{1} $を与える.
		\item Isospinは二成分,例えば$(p, n) $を$p \sim \ket{\uparrow}(1, 0) $, $n\sim \ket{\downarrow} \sim (0, 1)$と思って.spinと同じように扱う.もともとは原子核の$(p, n) $についてのアイデアだったが,これをquark $(u, s) $にも応用できる.さらにstrangeも加えて,$\SU(3) $対称性を考えることが有用である.
		\item $\SO(3) $と$\SU(2) $が同型という記述があるが嘘.数学の本を見ると$\SU(2)/\Z_2 \cong \SO(3)$という同型があって,もし$\SU(2) \cong \SO(3) $ならば推移律より$\SU(2) \cong\SU(2)/\Z_2 $だが,この間の同型写像は構成できない.

		\item CG(Clebsh--Gordan)係数に依る素粒子過程への制限は\cite[Chap.4]{BB03663366}が参考になる.量子力学の基本的な教科書にも記述はあると思う.CG係数は$\ket{J, M} $という状態を持った粒子が$\ket{j_1, m_1} $と$\ket{j_2, m_2} $という状態の$2 $粒子にdecayするとき,decayの仕方により
				\begin{equation}
						\ket{J, M} = \sum (\bra{j_1, m_1}\otimes\bra{j_2, m_2})\ket{J, M} \ket{j_1, m_1}\otimes\ket{j_2, m_2}
				\end{equation}
				となるが,その一つに対する確率振幅を表す$ (\bra{j_1, m_1}\otimes\bra{j_2, m_2})\ket{J, M}$のことである.
		\item G-parityの定義$G = C\e^{\i\pi I_2} $の$I_2 $は$2\times 2 $単位行列ではなくIsospinの第二成分の意味である.スピンで言えば$J_2 $のことである.今,pionを考えているので,Isospin $1 $表現でIsospinの(群としての)作用は通常の空間回転と同じで
				\begin{equation}
						G = CR_2(\pi)  = C\operatorname{diag}(-1, 1, -1)
				\end{equation}
				とy軸周りの$\pi $回転となる.
		\item CEPP Eq. (5.28)からどのようにCEPP Eq. (5.29)が導かれるかだが,
				通常pionと行ったときは$\SU(2) $の随伴表現の$3 $成分の固有状態をいうようで
				\footnote{最初,$3 $成分を対角化する表現だと思って混乱した.}\cite{Bransden2015},表現のとり方はCEPP Eq. (1.59)のとおりである.
				今,Isospin第三成分は
				\begin{equation}
						I_3 = \begin{pmatrix}
								0 & -i & 0\\
								i & 0 & 0\\
								0 & 0 & 0
						\end{pmatrix}
				\end{equation}
				と取っていて,pionは
				\begin{equation}
						\ket{\pi^{\pm}} = 
						\begin{pmatrix}
								1\\\pm \i\\0
						\end{pmatrix},\quad
						\ket{\pi^{0}} = 
						\begin{pmatrix}
								0\\0\\1
						\end{pmatrix}
				\end{equation}
				と表現されている.これがわかれば目標の式変形は明らか.
		\item $q\bar{q} $ タイプのmesonのエネルギーの限界値は$1300\unit{\textcolor{red}{M}eV} $である.
\end{itemize}
\subsection{The heavy meson}
\begin{itemize}
		\item 何を比較して\footnote{クオークの重さを直接比べよ,と言われたが,この節のタイトルからしてmesonで比べるべきでは?と思ったが.}重い,軽いと言っているのか.PDGでクオークについての欄があるので,調べること.
\end{itemize}
