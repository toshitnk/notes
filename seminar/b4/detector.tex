\section{Detectors of Elementary Particles}
この節はとても早く進んだので,多めに予習しておいたほうがよい.
\subsection{Energy loss by ipnization}
\begin{itemize}
		\item ``pancake''は速度方向にLorentz収縮することのスラングのようである.\cite{Pes:2022}の和訳参照.
		\item Eq. (6.3)の積分範囲は$-\infty \to \infty$で$\gamma vt = b\tan\theta $と置換すればできる.
		\item Eq. (6.5)は微小角度と思い,弾性散乱を考えると,$\vec{p} $と$\vec{p}\ \!'$, $\varDelta \vec{p} $が作る二等辺三角形を考えると,$\varDelta p = 2p\sin\theta/2\simeq p\theta$となり,求めた$\varDelta p $を使えばこれが求まる.
		\item Eq. (6.8)は自然単位系に戻しているが,Eq. (6.9)は$\hbar $が残った形である.この後,Eq. (6.11)でも$\hbar $が残った形だが,SIでやっても$\hbar $は相殺して消えるはずである.不思議なことが起こっている.
		\item \cite{Bethe1930}の論文はドイツ語だし,大学が契約していなくて読めなかった.
\end{itemize}
\subsection{Electromagnetic showers}
\begin{itemize}
		\item Eq. (6.15)は次のように導く.$x $まで散乱せず進む確率が$P(x) $.$x + \varDelta x $までに散乱しない確率は$1-\varDelta x /\lambda$なので,$P(x + \varDelta x) = P(x)(1-\varDelta x/\lambda) $.$\varDelta \to 0 $の極限をとり整理すると,$\dv*{P(x)}{x} = -P(x)/\lambda $となる.
		\item $x\langle E(x)\rangle $の前の$x $はエネルギー比$z $のtypoである.
		\item $e^{\pm} \to \gamma \to e^{+}e^{-} $を繰り返すことで電磁シャワーになる.
\end{itemize}
\subsection{Energy loss through macroscopic properties of the medium}
\begin{itemize}
		\item Cherenkov放射で同じ運動量のpionとkaonを見分けられるのは質量が違うので,同じ運動量だと速度$\beta $が異なり,
				$\cos\theta_{\text{C}}= 1/(n\beta) $が異なるからだとおもう.
\end{itemize}
\subsection{Detector systems for collider physics}
\begin{itemize}
		\item Eq. (6.37) $\varDelta (1/p) = \varDelta p/p^2 $はそんなものという理解でも良いが,ちゃんとやろうと思うと誤差伝播と思うと良いと思う.
		\item Neutrinoはほとんど相互作用せず,検出できないが,運動量保存の差分から予測ができる.差分を使うことは,たとえばダークマターが出てくるとして,ダークマターの質量はこの差分として現れることなどに用いている.
\end{itemize}
