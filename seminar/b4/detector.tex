\section{Detectors of Elementary Particles}
この節はとても早く進んだので,多めに予習しておいたほうがよい.
\subsection{Energy loss by ipnization}
\begin{itemize}
		\item ``pancake''は速度方向にLorentz収縮することのスラングのようである.\cite{Pes:2022}の和訳参照.
		\item CEPP Eq. (6.3)の積分範囲は$-\infty \to \infty$で$\gamma vt = b\tan\theta $と置換すればできる.
		\item CEPP Eq. (6.5)は微小角度と思い,弾性散乱を考えると,$\vec{p} $と$\vec{p}\ \!'$, $\varDelta \vec{p} $が作る二等辺三角形を考えると,$\varDelta p = 2p\sin\theta/2\simeq p\theta$となり,求めた$\varDelta p $を使えばこれが求まる.
		\item CEPP Eq. (6.8)は自然単位系に戻しているが,CEPP Eq. (6.9)は$\hbar $が残った形である.この後,CEPP Eq. (6.11)でも$\hbar $が残った形だが,SIでやっても$\hbar $は相殺して消えるはずである.不思議なことが起こっている.
		\item \cite{Bethe1930}の論文はドイツ語だし,大学が契約していなくて読めなかった.
\end{itemize}
\subsection{Electromagnetic showers}
\begin{itemize}
		\item CEPP Eq. (6.15)は次のように導く.$x $まで散乱せず進む確率が$P(x) $.$x + \varDelta x $までに散乱しない確率は$1-\varDelta x /\lambda$なので,$P(x + \varDelta x) = P(x)(1-\varDelta x/\lambda) $.$\varDelta \to 0 $の極限をとり整理すると,$\dv*{P(x)}{x} = -P(x)/\lambda $となる.
\end{itemize}
