\section{Symmetries of Space-Time}
\subsection{Relativistic particle kinematics}
\subsection{Natural units}
\begin{itemize}
		\item SIと自然単位系の変換を実際に計算すること.

		\item Eq. (2.30)は冪が間違い.正しくは$10^{-23} $.
\end{itemize}
\subsection{A little theory of discrete groups}
\begin{itemize}
		\item 物理でよくいう「群論」は群の表現論である.群の演算を線形空間の行列で表現するときに,量子力学ではこの線形空間が状態空間(Hilbert空間)で行列が物理量である.
		\item 対称群$\Pi_3 $の元が数学でよくある置換でないことに注意.$\Pi_3 $は元自身ではなく文字の場所にラベルをつけている.
\end{itemize}
\subsection{A little theory of continuous groups}
\begin{itemize}
		\item 対称性と保存則の関係は\cite[Chap.10, Sec.4]{BB17130464}に詳しい.
		\item Lie群は単位元近傍の微小変換を調べれば大体わかる.
				Lie群の単位元近傍は線形空間になっていて,行列の交換子で閉じる.この構造をLie環という.
				物理で重要な例は$\su(2) $というLie環で,量子力学でよくやる各運動量演算子$[J_j, J_k] = i\epsilon_{jkl}J_l $がそれである.
				表現の次元はspinの大きさで勘定することがある.
				角運動量の大きさ$j $に対し,行列のサイズは$2j + 1 $になるので,.$2j+1 $次元表現をspin $j $表現と言ったりする.例えば,$1 $次元表現はspin $0 $表現で,$2 $次元表現はspin $1/2 $表現で$J_i $はPauli行列たちである.このあたりの話は\cite[Chap.3]{BB03663366}, \cite{BN10398292}あたりが詳しい.数学の本だと,線形代数と接続がよいものとして\cite{BC13565134}がある.
\end{itemize}
\subsection{Discrete space-time symmetries}
\begin{itemize}
		\item Eq. (2.71)の$P $がかかって$+ $を吐くか$- $を吐くかは粒子や場の性質である.$- $を吐くほうを擬スカラー/ベクトルなどという.
\end{itemize}
