\section{Relativistic Wave Equations}
\subsection{The Klein--Gordon equation}
\subsection{Fields and particles}
\subsection{Maxwell's equations}
\subsection{The Dirac equations}
\begin{itemize}
		\item Dirac表現
				\begin{equation}
						\gamma^0 
						=
						\begin{pmatrix}
								1_2 & 0\\
								0 & -1_2
						\end{pmatrix},\quad
						\gamma^i
						=
						\begin{pmatrix}
								0 & \gamma^i\\
								-\gamma^i & 0
						\end{pmatrix}
				\end{equation}
				とChiral表現
				\begin{equation}
						\gamma^0
						=
						\begin{pmatrix}
								0 & 1_2\\
								1_2 & 0
						\end{pmatrix},\quad
						\gamma^i
						=
						\begin{pmatrix}
								0 & \gamma^i\\
								-\gamma^i & 0
						\end{pmatrix}
				\end{equation}
				の間のユニタリ変換は
				\begin{equation}
						U =
				\begin{pmatrix}
						1 & 0 & -1 & 0\\
						0 & 1 & 0 & -1\\
						1 & 0 & 1 & 0\\
						0 & 1 & 0 & 1
				\end{pmatrix}
				\end{equation}
				ととり,$U^{\dagger}\gamma^{\mu}U $でDirac表現に移る.
				他の流儀でchiral表現を
				\begin{equation}
						\gamma^i = 
						\begin{pmatrix}
								0 & -\sigma^i\\
								\sigma^i & 0
						\end{pmatrix}
				\end{equation}
				ととる本もある\cite{BB17130464}がこの場合はUnitary行列を
				\begin{equation}
				\begin{pmatrix}
						1 & 0 & 1 & 0\\
						0 & 1 & 0 & 1\\
						1 & 0 & -1 & 0\\
						0 & 1 & 0 & -1
				\end{pmatrix}
				\end{equation}
				ととればよい.こちらのほうが最初に思いつく対角化だと思う.
\end{itemize}
\subsection{Relaticistic narmalization of states}
\begin{itemize}
	\item 公式(3.82) $\delta(g(x)) = \delta(x) / \abs{\dv*{g(x)}{x}}$は次のように導出できる.
		まず,$\delta$関数は
		\begin{align}
			\int \dd{x}\delta(x) &= 1\\
			\int \dd{x} f(x) \delta(x-a) = f(a)
		\end{align}
		を満たす「なにか」と定まっていることとする.今,$g(x)$を$x = \alpha$に零点を持つ単調関数として,$y = g(x)$とおくと$\dd{y} = g'(x)\dd{x}$の変数変換があるので,
		\begin{equation}
			f(\alpha) = \int_{-\infty}^{\infty}\dd{y}f(y)\delta(y-a) = \int_{-\infty}^{\infty}\dd{x}\abs{g'(x)}f(x)\delta(g(x))
		\end{equation}
		となる.絶対値がつくのは,$g(x)$が単調減少のとき,積分区間が$y\colon-\infty \to \infty$のとき,$x\colon \infty \to -\infty$だから,積分区間を反転させるマイナスと$g'(x)$が負であることに注意するとこのようになることが理解できる.教科書の場合は$x=0$で零点を持つ場合であるので,比較して積分変数$y$はダミーなので$x$としてもよく$\delta(g(x)) = \delta(x)/\abs{g'(x)}$となる.
		
		もっと一般に,$g'(x)\neq0$で$x = \alpha_i$に零点を持つ微分可能関数$g(x)$について,
		\begin{equation}
			\delta(g(x)) = \sum_i\frac{1}{\abs{\dv*{g(x)}{x}}}\delta(x-\alpha_i)
		\end{equation}
		が成り立つ.

		導出は次のようにする.
		まず,$\delta$関数が全空間の積分でなくても
		\begin{equation}
			1 = \int_{-\epsilon}^{\epsilon}\dd{x}\delta(x)
		\end{equation}
		であることに注意すると,単調関数の場合の導出は,全空間で単調でなくても零点を含む積分区間内で単調であれば同様の結果が成り立つことがわかる.

		今,$g(x)$を$x = \alpha_i,\ i = 1, 2, \ldots, n$に零点を持つ連続関数とする.ただし,$i\neq j$ならば$\alpha_i \neq \alpha_j$で$g'(\alpha_i)  \neq 0$とする.この場合,$\epsilon > 0$を区間$[\alpha_i-\epsilon, \alpha_i+\epsilon]$内で単調であるよう小さく取ると,積分は
		\begin{equation}
			\int_{-\infty}^{\infty} \dd{x} \abs{g'(x)}\delta{g(x)} 
			= \sum_{i=1}^{n}\int_{\alpha_i - \epsilon}^{\alpha_i + \epsilon}\dd{x}\abs{g'(x)}\delta{g(x)}
		\end{equation}
		と分割することができる.
		各積分について,前の結果を使うことで
		\begin{equation}
			\int_{\alpha_i-\epsilon}^{\alpha_i+\epsilon}\dd{x}\abs{g'(x)}\delta(g(x)) = \int_{g(\alpha_i-\epsilon)}^{g(\alpha_i+\epsilon)}\dd{y}\delta(y-\alpha_i)
		\end{equation}
		となり,\footnote{積分区間は単調減少なら引数内の負号は反対で場合分けする必要があるので正しくないが,許して.}
		足して比較すると
		\begin{equation}
			\delta(g(x)) = \sum_{i=1}^{n}\frac{1}{\abs{g'(x)}}\delta(x-\alpha_i)
		\end{equation}
		を得る.


		\item この小節ではベクトルの添字とスピノルの添字を区別しなければならない.\footnote{ベクトル添字には$\mu $, $\nu $, $\ldots $あたりのギリシア文字,スピノル添字には$\alpha$, $\beta $, $\ldots $あたりのギリシア文字や$a $, $b $, $\ldots $あたりのアルファベットを使いたい気持ちがある.}Gamma行列$\gamma^{\mu} $は$4\times 4 $行列で$4 $つある.このとき,$\mu $はベクトルの添字である.
				スピノル添字は$a, b,\ldots $や$\alpha, \beta,\ldots $あたりの文字をよく使う.通常の行列$A $を添字をexplicitに$A = (a_{ab}) $と書くことに倣うと$\gamma^{\mu} =(( \gamma^{\mu})_{b}^{a}) $とかける.上付き下付きはあまり気にしなくて良いが,意味はある\cite[Chap.1]{九後1989}, \cite{BB17130464}.
		\item Dirac方程式$\qty(i\gamma^\mu \partial_\mu - m)\psi(x) = 0 $は四成分スピノルの方程式であることに注意せよ.とくに$m $の後ろには$4\times 4 $単位行列が省略されている.また,スピノル添字をexplicitにかけるか?\footnote{正方行列$A =(a_{\alpha\beta})$, $B= (b_{\alpha\beta}) $の積$AB $を添字を用いて表すと$(AB)_{\alpha\gamma} = A_{\alpha\beta}B_{\beta\gamma} $とかけることを思い出すと}
				\begin{equation}
				\qty(i\gamma^{\mu}\partial_{\mu} -m)_{\beta}^{\alpha}\psi(x)^\beta = 0^{\alpha}
				\end{equation}
				である.
\end{itemize}
\subsection{Spin and statistics}
