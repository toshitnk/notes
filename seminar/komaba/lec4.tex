\section{一般相対論}

Newtonの重力理論は特殊相対論とは相容れない.平らな時空の幾何である特殊相対論をまがった時空に拡張した一般相対論というものになる.

そのまえに「力」とは?Newtonの運動方程式$F = ma $の$F $だが,摩擦力などいろいろあるが,突き詰めれば重力,電磁気力,弱い力\footnote{$\beta $崩壊を起こす.},強い力\footnote{quarkを陽子,中性子などにまとめる力.}の四つである\footnote{強い力,弱い力は常に量子的に考える.}.

力の統一については
\begin{align}
		\xymatrix{
& \text{重力} & \\
				\text{電弱力} &\ar[l]\text{電磁気力} & \ar[l] \text{電気力}\\
				& \ar[ul]^{\text{Glashow--Weinberg--Salam}}\text{弱い力} & \ar[ul]_{\text{Maxwell}}\text{磁気力}\\
				& \text{強い力} & 
		}\notag
\end{align}
となっている.

各力の量子論の現状は表\ref{tab:quantum}のようである.量子重力は現状全くできていない.
\begin{table}[htbp]
		\centering
		\caption{各力の量子論.}
		\label{tab:quantum}
		\begin{tabular}{cl}\hline
				量子電磁力学 & 物理屋的には確立.\\
				量子弱い力 & 実験とよく合う.\\
				量子強い力\footnote{hoge} & 数学的にはよく定義されていない.\\\hline
				量子重力 & 
				\begin{tabular}{l}
						実験結果は全くなし.\\
						物理屋のレベルでも全然できていない.\\
						ブラックホールの情報喪失問題はその一端.
				\end{tabular}\\\hline
		\end{tabular}
\end{table}
