\section{Introduction}
\begin{itemize}
	\item 特殊相対論の慣性系は重力のない系に定義される.
	たとえば,重力の効果によって潮汐力などが働くと,慣性系が定義できない状況が考えられる.問題なく定義できる慣性系を局所慣性系といい,
	どの程度の広さ慣性系を張れるかという特徴的な長さは曲率によってさだまる.
	\item 物理の理論は,状態・方程式・解釈からなる.
	\begin{table}[htbp]
		\centering
		\begin{tabular}{c|ccc}
		 & 状態 & 方程式 & 解釈 \\\hline
		 古典力学 & $\vec{x}\in\R^3$, $\vec{v}\in\R^3$ & $m\ddot{\vec{x}} = \vec{F}$ & $\vec{x}$, $\vec{v}$は粒子の位置および速度.\\
		 電磁気学 & $\vec{E},\ \vec{B}\colon\R^3 \to \R^3$ & Maxwell equation & 電磁場\\
		 量子力学 & $\ket{\psi}\in\mathcal{H}$ & Schr\"{o}dinger equation & 確率解釈\\
		 一般相対論 & $g_{\mu\nu}$ & $G_{\mu\nu} = 8\pi G T_{\mu_nu}$ & 物質が空間の形状を定める.
		\end{tabular}
	\end{table} 
\end{itemize}
