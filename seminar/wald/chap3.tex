\section{CURVATURE}
\subsection{Derivative Operators and Parallel Transport}
\begin{itemize}
	\item Christoffel symbolは変換性の観点からはテンソルでない.\textref{3.1.15}で左辺はテンソルの変換をするが,右辺のChristoffel symbolは$\Gamma\indices{^b_{ac}}t^c$全体としてテンソルの変換をする.
		$\nabla'_a t'^b = \partial'_a t'^b +{\Gamma'}\indices{^b_{ac}}t'^c$と書いたとき,${\Gamma'}\indices{^b_a_c} \neq \partial'_ax^d\partial_ex'^b\partial'_cx^f \Gamma\indices{^e_d_f}$ということ.
	\item Theorem 3.1.1について,uniquenessは次のように言える.
		今,二つの共変微分$\tilde{\nabla}_a$, $\tilde{\nabla}'_a$をとったとして,\textref{3.1.28}の方法で$C\indices{^c_a_b}$, ${C'}\indices{^c_a_b}$をそれぞれに対しつくると,
		\begin{align}
			0 &= \nabla_ag_{bc} = \tilde{\nabla}_a g_{bc} - C\indices{^d_a_b}g_{dc} - C\indices{^d_a_c}g_{bd}\\
			0 &= \nabla'_ag_{bc} = \tilde{\nabla}'_a g_{bc} - {C'}\indices{^d_a_b}g_{dc} - {C'}\indices{^d_a_c}g_{bd}
		\end{align}
		が成り立つ.ここで,$\nabla_a$, $\nabla'_a$が等しいことは,次のようにして言える.

		$\nabla_a$, $\nabla'_a$は共変微分なので,
		$\nabla_ag_{bc} = \nabla'_ag_bc - D\indices{^d_a_b}g_{dc} - D\indices{^d_a_c}g_{bd}$の差があるが,これはゼロであるので,
		$D\indices{^d_a_b}g_{dc} - D\indices{^d_a_c}g_{bd} = 0$でないといけない.ところが,\textref{3.1.24}以降の一連の議論を繰り返すと,\textref{3.1.28}に対応する部分より,
		$D\indices{^c_a_b} = 0$となり,作った微分は最初の微分のとり方によらないことがわかる.
\end{itemize}
\subsection{Curvature}
\subsection{Geodesics}
\subsection{Methods for Computing Curvature}

