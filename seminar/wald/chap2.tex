\section{Manifolds and Tensor Fields}
\begin{itemize}
	\item waldは多様体の定義を近傍系から定めている.近傍とは位相空間$M, \mathcal{O}$として$p\in M$を含む$O\in \mathcal{O}$のことである.
	通常,多様体は位相空間であることを用いて定義するが,近傍により開集合系を定めると位相空間になるので良いのだと思う.
	\item 多様体の位相的性質としてハウスドルフかつパラコンパクトであることを課している.
	ハウスドルフ性は通常課すのでよいとして,パラコンパクトであることは多様体上の積分を定義するときに1の分割が常に存在することを保証するらしいがよくわからない.
	\item  $S^2$が多様体であることの例では,座標変換をたとえ$(1, +)$から$(2, +)$の変換では$f_2^{+}\circ (f_1^-)^{-1}(x_1, x_2) = (\sqrt{1-(x^1)^2-(x^2)^2}, x^2)$と定めれば微分可能であることがわかる.
\end{itemize}
