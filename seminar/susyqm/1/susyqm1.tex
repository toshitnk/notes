\documentclass[english, dvipdfmx, a4paper]{jsarticle}
\usepackage[utf8]{inputenc}
\usepackage[top=10truemm, bottom=20truemm, left=15truemm, right=15truemm]{geometry} % mergin
\renewcommand{\headfont}{\bfseries}

% graphics
\usepackage{graphicx}
\usepackage{here}

% link

\usepackage{url}
\usepackage[dvipdfmx, linktocpage]{hyperref} 
\usepackage{xcolor}
\usepackage{pxjahyper}
\hypersetup{
	colorlinks=true,
	citecolor=blue,
	linkcolor=teal,
	urlcolor=orange,
}

% math

\usepackage{amsmath, amssymb} 
\usepackage{physics}
\usepackage{mathrsfs}
\usepackage{mathtools}

% theoremstyle
\usepackage{amsthm}
\newtheoremstyle{break}
{\topsep}{\topsep}%
{}{}%
{\bfseries}{}%
{\newline}{}%
\theoremstyle{break}
\newtheorem{thm}{Theorem}[section]
\newtheorem{defn}[thm]{Definition}
\newtheorem{eg}[thm]{Example}
\newtheorem{cl}[thm]{Claim}
\newtheorem{cor}[thm]{Corollary}
\newtheorem{fact}[thm]{Fact}
\newtheorem{rem}[thm]{Remark}
\newtheorem{prop}{Property}
\newtheorem{prob}{Problem}[section]

\makeatletter
\newenvironment{pr}[1][\proofnam]{\par
\topsep6\p@\@plus6\p@ \trivlist
\item[\hskip\labelsep{\itshape #1}\@addpunct{\bfseries}]\ignorespaces
}{%
\endtrivlist
}
\newcommand{\proofnam}{\underline{Derivation.}}
\makeatother
\renewcommand{\i}{\mathrm{i}}
\newcommand{\e}{\mathrm{e}}
\newcommand{\N}{\mathcal{N}}

% number

%\makeatletter
%\@addtoreset{equation}{section}
%\makeatother
%\numberwithin{equation}{section}
%\renewcommand{\thefootnote}{\roman{footnote}.}
%\renewcommand{\appendixname}{Appendix }
\title{}
\author{Toshiya Tanaka}
\date{\today}

\begin{document}
	\section{SUSY in QM}
	\begin{defn}[最小超対称関係]
		3種類のHermitian operator $H$: Hamiltonian, $Q$: Supercharge, $(-1)^F$: ???があって
		\begin{align}
			H &= Q^2\\
			Q(-1)^{F} &= -(-1)^{F}Q,\quad  \text{or}\quad \{Q, (-1)^F\} =0\\
			\qty((-1)^{F})^2 &=1
		\end{align}
		を満たす関係を最小超対称関係という.

		超対称性がある系には,必ずこの関係がある.
	\end{defn}
	実は,簡単な量子力学系にもこの構造が隠れている.
	\subsection{円周上の自由粒子}
	半径$R$の円周上の自由粒子を考える.定義域を$-\pi R\leq x\leq \pi R$とし,
	周期的境界条件$\psi(x+2\pi R) = \psi(x)$を入れる.
	Hamiltonianは$H=-1/(2m)\dv*{}{x}$であるので,Schr\"{o}dinger方程式を解くと,固有関数として,次の固有関数を得る.
	\begin{align}
		\phi_{n,+}(x) &= N_{n,+}\cos\qty(\frac{n}{R}x)\\
		\phi_{n,-}(x) &= N_{n, -} \sin\qty(\frac{n}{R}x).
	\end{align}
	これらの固有エネルギーは,
	\begin{equation}
		E=\frac{1}{2mR^2}n^2
	\end{equation}
	で,各固有空間は$2$次元あることがわかる.Hamiltonianを``因数分解''してsuperchargeを得る.
	$H = (-\i/(\sqrt{2m})\dv*{}{x})^2$より,$Q\coloneqq-\i/(\sqrt{2m})\dv*{}{x}=p/\sqrt{2m}$とする.

	また,parity $\mathcal{P}$は$(-1)^{F}$の働きをする.

	よって,この系には,最小超対称関係を満たす演算子たちが存在することがわかる.
	これらはHermitianであることも確かめられる\footnote{$A$がHermitianとは,
	今定まっている内積$\langle\psi, \phi\rangle=\int\dd{x}\qty(\psi(x))^*\phi(x)$に対して,$\langle A\psi, \phi\rangle = \langle \psi , A\phi\rangle$が成り立つことである.}.

	今,周期的境界条件で考えたが,ひねった境界条件$\psi(x+2\pi R) = \e^{\i\theta}\psi(x)$を入れると面白い\footnote{Aharanov--Bohmのように,磁場を使うと,実際に作ることができる.}.
	$\theta$の連続変形でスペクトラムの構造は連続的に変化し
	\footnote{spectral flowという.},$\theta = n\pi$のところではSUSYの構造が現れるが,その他のところでは現れない.
	実際計算すると,固有エネルギーと固有状態は
	\begin{align}
		\psi_n &= N_{n}\e^{\i(n+\theta/(2\pi))x/R}\\
		E_n &= \frac{1}{2mR^2}\qty(n+\frac{\theta}{2\pi})^2
	\end{align}
	となる.

	$\theta \neq n\pi$でSUSYが壊れているのは,parityが上手くいっていないからである.
	境界条件を考えると,$\psi(x+2\pi R) = \e^{\i\theta}\psi(x)$だが,
	$x' = -x-2\pi R$とおくと,$\psi(x'+2\pi R)=\e^{-\i\theta}\psi(x')$となって
	しまい,$\theta\neq n\pi$ではparityで境界条件が不変でないので,同じ系の中で対応が作れない.
	\subsection{超対称性の基本性質}

	SUSYがある系は最小超対称関係
	\begin{itemize}
		\item $H=Q^2$
		\item $\qty{Q, (-1)^{F}}=0$
		\item $\qty((-1)^{F})=1$
	\end{itemize}
	が必ずある.

	この関係から,$[H, Q]=0$, $[H, (-1)^F]$がなりたつので,$H$と$(-1)^{F}$の
	同時固有状態$\ket{E, \lambda}$をとることができる.$\qty((-1)^{F})^2=1$
	なので,$\lambda = \pm1$である.



	以下の4つの性質が成り立つ.
	\begin{prop}
		エネルギー固有値が非負.$E\geq0$.
	\end{prop}
	次の式変形からわかる\footnote{$Q$がHermitianであることは本質的である.}.
	\begin{align}
		E &= \mel{E, \lambda}{H}{E, \lambda}\\
		  &= \mel{E, \lambda}{Q^2}{E, \lambda}\\
		  &= \|Q\ket{E, \lambda}\|^2\\
		  &\geq0.
	\end{align}\qed
	\begin{prop}
		正エネルギー状態は,$(-1)^{F}$の固有値が$\pm1$の固有状態$\ket{E, \pm}$で対を成し,エネルギー固有値は縮退する.
	\end{prop}
	まず,$E>0$として,$\ket{E, +}$を考える.
	$(-1)^{F}Q\ket{E, +} = -Q(-1)^{F}\ket{E, +} = -Q\ket{E, +}$なので,$Q\ket{E, +}\propto\ket{E, -}$.
	$\ket{E, -}$についても同様にして,$Q\ket{E, -}\propto\ket{E, +}$である.

	また,比例定数は	
	\begin{align}
		\|Q\ket{E,+}\|^2 &= \mel{E, +}{Q^{\dag}Q}{E, +}\\
						 &= \expval{H}{E, +}\\
						 &= E
	\end{align}
	となるので\footnote{phaseは実にとると},
	\begin{equation}
		Q\ket{E,\pm} = \sqrt{E}\ket{E, \mp}\label{eq:normalization}
	\end{equation}
	と決まる\footnote{$Q$は$H$を``因数分解''して作ったことを思い出すと,大きさは$\sqrt{E}$になると思える.}.
	
	このとき,$\ket{E, \pm}$は$Q$を通じて対を成しており,supermultipletを成すという.
	この状況を模式的に$\ket{E, +}\overset{Q}{\longleftrightarrow}\ket{E, -}$と書く.\qed

	\begin{prop}
		ゼロエネルギー状態\footnote{SUSYの文脈でこのような状態をBPS stateという.}は必ずしも縮退しない.
		ゼロエネルギー状態が存在するならば,$Q\ket{E=0}=0$を満たす.
	\end{prop}
	Eq. \eqref{eq:normalization}に$E=0$を代入すると直ちにわかる.
	$E\neq0$のときとは異なり,$Q$を通じたsupermultipletをなさない.
	この状況を$\ket{E=0, +}\overset{Q}{\longrightarrow} 0 \overset{Q}{\longleftarrow} \ket{E=0, -}$と書く.\qed

	ゼロエネルギー状態が$Q\ket{E=0, \pm}$を満たすことは,ゼロエネルギー状態は1階の微分方程式の解であることを意味する.


	\begin{prop}
		Witten index $\Delta_{\text{W}} \coloneqq \N_{E=0}^{+}-\N_{E=0}^{-}$はtopological invariant.

		ここで,$\N_{E=0}^{\pm}$は$(-1)^F$の固有値が$\pm1$の固有状態の数である.
	\end{prop}
	topological invariantとは,理論のパラメータの連続変形で不変な量という意味で用いる.
	$S^1$上の自由粒子の例では$m$や$R$を大きくとると,$n\neq0$に置いても$E_n\to0$となるが,
	もともとnon zeroであるものは対で存在するので,ゼロエネルギー状態の数の差は変わらない.\qed
\end{document}
