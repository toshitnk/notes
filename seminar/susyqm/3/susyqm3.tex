\documentclass[english, dvipdfmx, a4paper]{jsarticle}
\usepackage{docmute}
\usepackage[utf8]{inputenc}
\usepackage[top=10truemm, bottom=20truemm, left=15truemm, right=15truemm]{geometry} % mergin
\renewcommand{\headfont}{\bfseries}

% graphics
\usepackage{graphicx}
\usepackage{here}

% link

\usepackage{url}
\usepackage[dvipdfmx, linktocpage]{hyperref} 
\usepackage{xcolor}
\usepackage{pxjahyper}
\hypersetup{
	colorlinks=true,
	citecolor=blue,
	linkcolor=teal,
	urlcolor=orange,
}

% math

\usepackage{amsmath, amssymb} 
\usepackage{physics}
\usepackage{mathrsfs}
\usepackage{mathtools}

% theoremstyle
\usepackage{amsthm}
\newtheoremstyle{break}
{\topsep}{\topsep}%
{}{}%
{\bfseries}{}%
{\newline}{}%
\theoremstyle{break}
\newtheorem{thm}{Theorem}[section]
\newtheorem{defn}[thm]{Definition}
\newtheorem{eg}[thm]{Example}
\newtheorem{cl}[thm]{Claim}
\newtheorem{cor}[thm]{Corollary}
\newtheorem{fact}[thm]{Fact}
\newtheorem{rem}[thm]{Remark}
\newtheorem{prob}{Problem}[section]
\newtheorem{prop}{Property}

\makeatletter
\newenvironment{pr}[1][\proofnam]{\par
\topsep6\p@\@plus6\p@ \trivlist
\item[\hskip\labelsep{\itshape #1}\@addpunct{\bfseries}]\ignorespaces
}{%
\endtrivlist
}
\newcommand{\proofnam}{\underline{Derivation.}}
\makeatother
% ordinary
	\newcommand{\R}{\mathbb{R}}
	\newcommand{\C}{\mathbb{C}}
	\newcommand{\Z}{\mathbb{Z}}
	
	
	% Physics %%%%%%%%%%%%%%%%%%%%%%%%%%%%%%%%%%%%
	
	% Feynman slash
	\newcommand{\slashed}[1]{#1\!\!\!/}
	% order
	
	%Lie algebra
	\renewcommand{\O}{\mathcal{O}}
	\newcommand{\SO}{\mathrm{SO}}
	\newcommand{\so}{\mathfrak{so}}
	\newcommand{\SU}{\mathrm{SU}}
	\newcommand{\su}{\mathfrak{su}}
	\newcommand{\SP}{\mathrm{SP}}
	\renewcommand{\sp}{\mathfrak{sp}}
	\newcommand{\SL}{\mathrm{SL}}
	\renewcommand{\sl}{\mathfrak{sl}}
	\newcommand{\GL}{\mathrm{GL}}
	\newcommand{\gl}{\mathfrak{gl}}
	\newcommand{\U}{\mathrm{U}}
	\renewcommand{\u}{\mathfrak{u}}
\renewcommand{\i}{\mathrm{i}}
\newcommand{\e}{\mathrm{e}}
\newcommand{\N}{\mathcal{N}}
\newcommand{\Q}{\mathcal{Q}}
\renewcommand{\P}{\mathcal{P}}
% number

%\makeatletter
%\@addtoreset{equation}{section}
%\makeatother
%\numberwithin{equation}{section}
%\renewcommand{\thefootnote}{\roman{footnote}.}
%\renewcommand{\appendixname}{Appendix }

\title{SUSYQM}
\author{Toshiya Tanaka}
\date{\today}

\begin{document}
	\section{Witten model}
	\begin{defn}[Witten model]
		Hamiltonianを2成分で
		\begin{equation}
			H = \mqty(A^{\dag}A & 0 \\ 0 & AA^{\dag}) = \mqty(H_+ & 0 \\ 0 & H_-)
		\end{equation}
		の形で与える量子力学系をWitten modelや$\N = 2$超対称量子力学($\N = 2$ SUSYQM)という.量子力学なので,具体的に$A$は
		\begin{equation}
			A \coloneqq \frac{1}{\sqrt{2m}}\qty(-\i\hbar\dv[2]{}{x} - \i W'(x))
		\end{equation}
		と書け,$W(x)$はsuperpotential\footnote{$W'(x)$のことをsuperpotentialという流儀もあって,実際Wittenの原論文はそうだが,$W(x)$のほうがふさわしいらしい.}という.
		また,
		\begin{align}
			H_+ &= \frac{-\hbar^2}{2m}\dv[2]{}{x} + \frac{1}{2m}\qty(W'(x))^2 - \frac{\hbar}{2m}W''(x)\\
			H_- &= \frac{-\hbar^2}{2m}\dv[2]{}{x} + \frac{1}{2m}\qty(W'(x))^2 + \frac{\hbar}{2m}W''(x)
		\end{align}
		である.
		これに伴い,波動関数は二成分考える必要があり,$\Psi(x) = (\psi_+(x), \psi_-(x))^{\top}$と書く.
	\end{defn}

	Witten modelは定義した$H$に加え,
	\begin{equation}
		Q \coloneqq \mqty(0 & A^\dag \\ A & 0), \quad (-1)^{F} = \mqty(1 & 0\\ 0 & -1)
	\end{equation}
	を定めると,これらは最小超対称関係を満たす.

	これから定義に従って計算すると,実superchargeは
	\begin{equation}
		Q_1 = \mqty(0 & A^\dag \\ A & 0), \quad Q_2 = \mqty(0 & \i A^\dag \\ -\i A & 0),
	\end{equation}
	複素superchargeは
	\begin{align}
		\Q &= \mqty(0 & 0 \\ \sqrt{2}A) = \sqrt{2}A\sigma_-\\
		\Q^{\dag} &=\mqty(0 & \sqrt{2}A^{\dag} \\ 0 & 0) = \sqrt{2}A^{\dag}\sigma_+
	\end{align}
	となる.$\sigma_{\pm}$は冪零行列なので,複素superchargeの冪零性はここから
	自然にわかる.

	エネルギーの縮退の構造に関して,今までどおり,$H$と$(-1)^{F}$の同時固有状態を考える.
	$(-1)^{F} = \sigma_3$の形から,固有状態は$(\psi_+(x), 0)^{\top}$, $(0, \psi_-(x))^{\top}$であり,$H\Psi_E(x) = E\Psi_E(x)$は
	\begin{align}
		A^{\dag}A\psi_{E, +}(x) &= E \psi_{E, +}(x)\\
		AA^{\dag}\psi_{E, -}(x) &= E\psi_{E, -}(x)
	\end{align}
	を解けば良いことになる.

	\begin{thm}
		ここで,$E > 0$の場合
		\begin{align}
			A\psi_{E, +}(x) &= \sqrt{E} \psi_{E, -}(x),\label{eq:annihilate}\\
			A^{\dag}\psi_{E, -}(x) &= \sqrt{E}\psi_{E, +}\label{eq:create}
		\end{align}
			の超対称関係がある.
	\end{thm}
	今,Eq. \eqref{eq:annihilate}の一つの解$\psi_{E, +}(x)$を取って,$A\psi_{E, +}(x)$を考えると,
	$AA^{\dag}(A\psi_{E, +})(x)  = EA\psi_{E, +}(x)$よりEq. \eqref{eq:create}の解になる.規格化は$\|A\psi_{E, +}\|^2 = \langle \psi_{E, +}, A^{\dag}A\psi_{E, +}\rangle = E$なので$A\psi_{E, +}(x) = \sqrt{E}\psi_{E, -}(x)$である.もうひとつも同様.\qed

	Witten modelは二成分で考えているが,別個の2つの量子力学系$H_+=A^{\dag}A$, $H_-=AA^{\dag}$を二つ取ってきたと思うと,その間に$A$と$A^{\dag}$を通して
	対応がついたことになる.$H_+$と$H_-$を超対称パートナーHamiltonianという.


	Witten modelのゼロエネルギー状態を調べると,Witten indexに幾何的解釈をつける
	ことができる.
	ゼロエネルギー状態は他とは異なり,一階の微分方程式の解になっている.
	$A\psi_{0, +}(x) = 0 $, $A^{\dag}\psi_{E, -}(x) = 0$だが,微分を明らかに書くと,
	\begin{align}
		\frac{-\i\hbar}{\sqrt{2m}}\qty(\dv{}{x}+\frac{1}{\hbar}W'(x))\psi_{0, +}(x) &= 0\\
		\frac{-\i\hbar}{\sqrt{2m}}\qty(\dv{}{x}-\frac{1}{\hbar}W'(x))\psi_{0, -}(x) &= 0
	\end{align}
	なので,簡単に解けて,
	\begin{align}
		\psi_{0, +}(x) &= N_{0, +}\e^{-W(x)/\hbar}\\
		\psi_{0, -}(x) &= N_{0, -}\e^{W(x)/\hbar}
	\end{align}
	となる.$N_{0, \pm}$は適当な規格化定数である.
	ただし,量子力学としては,この解であり規格化可能なものだけが
	取りうる状態なので規格化可能性を調べる必要がある.

	便宜上,superpotentialが$p$次多項式で$W(x) = a_0 + a_1x + a_2x^2 +\cdots + a_px^{p}$, $a_p\neq 0$で与えられるとする.規格化可能性を考えると,ゼロエネルギー状態はあるとしたらsuperpotentialの最高次が偶数次の場合で,符号が正なら$(-1)^{F}$の固有値が$+1$の方に存在し,負なら$-1$の方に存在することになる.\ref{tab:normalizability}.
	\begin{table}[htbp]
		\centering
		\label{tab:normalizability}
		\caption{$\checkmark$の入っている場合が規格化可能.}
		\begin{tabular}{c | cccc}\hline
			 & $W(\infty)$ & $W(-\infty)$ & $\psi_{0, +}$ & $\psi_{0, -}$\\\hline
			$a_p > 0$, $p$: even & $\infty$ & $\infty$ & $\checkmark$ & $\times$\\
			$a_p > 0$, $p$: odd & $\infty$ & $-\infty$ & $\times$ & $\times$\\
			$a_p < 0$, $p$: even & $-\infty$ & $\infty$ & $\times$ & $\checkmark$\\
			$a_p < 0$, $p$: odd & $-\infty$ & $-\infty$ & $\times$ & $\times$\\\hline
		\end{tabular}
	\end{table}

	Witten indexは
	\begin{equation}
		\Delta_{\text{W}} = 
		\begin{cases}
			+1,\quad & a_p>0, p:\text{ even}\\
			-1,\quad & a_p<0, p:\text{ even}\\
			0, \quad & p:\text{ odd}
		\end{cases}
	\end{equation}
	だが,これはsuperpotentialの$(\text{二回微分が正の極値の数}) - (\text{負の極値の数})$に等しいことが知られている\footnote{Witten1981に書いてあるらしいが,見つけられなかった.}.
	感覚的には,superpotentialを連続変形しても漸近的振る舞いを変えない限り,この数が変わらないこととしてWitten indexがtopological不変量であることが理解できる.


	現象論にもWitten indexは重要で,
	現状,SUSYは現実世界で観測されていないので,low energyではSUSYは破れていないといけない.

	ground state $\ket{\text{vac}}$として,$\Q\ket{\text{vac}} = \Q^{\dag}\ket{\text{vac}} = 0$のときSUSYは破れていない,そうでないときSUSYは破れているという.
	エネルギーの言葉ではこれはground energy $E_0 = 0$のときSUSYが破れている,そうでないときSUSYが破れていないことになる.

	Witten indexの言葉では,$\Delta_\text{W}\neq0$なら,かならずゼロエネルギー状態
	があるので,SUSYは破れていない.
	SUSYが破れているならば,$\Delta_\text{W} = 0$である.(逆は成り立たない.)

	SUSYが破れている現象論模型を作るためには,$\Delta_{\text{W}} = 0$を満たすように作らなければいけない.
\end{document}
