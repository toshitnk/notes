\documentclass[english, dvipdfmx, a4paper]{jsarticle}
\usepackage{docmute}
\usepackage[utf8]{inputenc}
\usepackage[top=10truemm, bottom=20truemm, left=15truemm, right=15truemm]{geometry} % mergin
\renewcommand{\headfont}{\bfseries}

% graphics
\usepackage{graphicx}
\usepackage{here}

% link

\usepackage{url}
\usepackage[dvipdfmx, linktocpage]{hyperref} 
\usepackage{xcolor}
\usepackage{pxjahyper}
\hypersetup{
	colorlinks=true,
	citecolor=blue,
	linkcolor=teal,
	urlcolor=orange,
}

% math

\usepackage{amsmath, amssymb} 
\usepackage{physics}
\usepackage{mathrsfs}
\usepackage{mathtools}

% theoremstyle
\usepackage{amsthm}
\newtheoremstyle{break}
{\topsep}{\topsep}%
{}{}%
{\bfseries}{}%
{\newline}{}%
\theoremstyle{break}
\newtheorem{thm}{Theorem}[section]
\newtheorem{defn}[thm]{Definition}
\newtheorem{eg}[thm]{Example}
\newtheorem{cl}[thm]{Claim}
\newtheorem{cor}[thm]{Corollary}
\newtheorem{fact}[thm]{Fact}
\newtheorem{rem}[thm]{Remark}
\newtheorem{prob}{Problem}[section]
\newtheorem{prop}{Property}

\makeatletter
\newenvironment{pr}[1][\proofnam]{\par
\topsep6\p@\@plus6\p@ \trivlist
\item[\hskip\labelsep{\itshape #1}\@addpunct{\bfseries}]\ignorespaces
}{%
\endtrivlist
}
\newcommand{\proofnam}{\underline{Derivation.}}
\makeatother
\newcommand{\Z}{\mathbb{Z}}
\newcommand{\C}{\mathbb{C}}
\renewcommand{\i}{\mathrm{i}}
\newcommand{\e}{\mathrm{e}}
\newcommand{\N}{\mathcal{N}}
\newcommand{\Q}{\mathcal{Q}}
\renewcommand{\P}{\mathcal{P}}
% number

%\makeatletter
%\@addtoreset{equation}{section}
%\makeatother
%\numberwithin{equation}{section}
%\renewcommand{\thefootnote}{\roman{footnote}.}
%\renewcommand{\appendixname}{Appendix }

\title{SUSYQM}
\author{Toshiya Tanaka}
\date{\today}

\begin{document}
\section{\texorpdfstring{$\N=2$}{N=2}超対称代数}
	\subsection{一般性質}
	\begin{defn}[実superchargeと$\N = 2$\footnote{$\N$はsuperchargeの数である.ところが,実(エルミート)で数える流儀と複素で数える流儀があり,文献を比較する際は確認する必要がある.今は実で数えている.}超対称代数]
		最小超対称関係を満たす三つのエルミート演算子$H$, $Q$, $(-1)^{F}$から,
		エルミート演算子\footnote{エルミートという意味で,実superchargeという.}
		$Q_1 \coloneqq Q$, $Q_2\coloneqq -\i Q(-1)^{F}$
		を定める.
		$H$, $Q_1$, $Q_2$は,		\begin{align}
			\qty{Q_i, Q_j} &= 2H\delta_{ij}\\
			\qty[H, Q_i] &= 0
		\end{align}
		の関係を満たす.
	\end{defn}
	これは,最小超対称関係と等価だが,$(-1)^{F}$を加えて議論することが多い.
	$(-1)^{F}$は余分な演算子なので,束縛関係$\i Q_1Q_2 = H(-1)^F$で結ばれる\footnote{$(-1)^{F}$という余分な演算子が現れたとき,$H$をかければ最小限の$Q_1$, $Q_2$のみで書けるということ.}.
	\begin{defn}[複素superchargeと$\N = 2$超対称代数]
		$\Q \coloneqq (Q_1 + \i Q_2)/\sqrt{2}$を定める.このとき,$\Q$, $H$は
		\begin{align}
			\qty{\Q^{\dag}, \Q} &= 2H\\
			\qty{\Q, \Q} = \qty{\Q^{\dag}, \Q^{\dag}} &= 0 \label{eq:nilp}\\
			\qty[H, \Q] = \qty[H, \Q^{\dag}] &= 0
		\end{align}
		の関係を満たす.
	\end{defn}
	Eq. \eqref{eq:nilp}は$\Q^2 = (\Q^{\dag})^2 = 0$と書け,$2$乗してゼロになるこの性質をnilpotencyという
	\footnote{nilpotencyは場の理論のGlassmann数の対称性であるBRS対称性による保存量であるBRS charge, de Rham coholomogyなどでも重要な働きをするそう.実際SUSYQMで微分幾何をやるという話もある.}.
	
	複素の場合も,$\N = 2$超対称代数に$(-1)^F$を加え考えることが便利である.
	この場合の束縛関係は
	$\qty(\Q^{\dag}\Q - \Q\Q^{\dag})/2 = H(-1)^{F}$である.左辺を実superchargeで計算すれば,$iQ_1Q_2$になり,実の場合の束縛を使うと導ける.

	複素superchargeで基本性質を見る.
	\setcounter{prop}{0}
	\begin{prop}
		$E\geq0$.
	\end{prop}
	\begin{prop}
		$E>0$なら次のsupermultipletの構造がある.
		\begin{equation}
			0\xleftarrow[\Q]{}\ket{E, -}\xleftrightarrow[\Q]{\Q^{\dag}}\ket{E, +} \xrightarrow[]{\Q^{\dag}} 0\label{eq:positive_spectrum}
		\end{equation}
	\end{prop}
	\begin{prop}
		$E=0$なら,$\Q$, $\Q^{\dag}$で消える.
		\begin{equation}
			0 \xleftarrow[\Q]{}\ket{E=0}\xrightarrow[]{\Q^{\dag}}0\label{eq:zero_spectrum}
		\end{equation}
	\end{prop}
	\begin{prop}
		Witten indexがトポロジカルな量.
	\end{prop}
	Witten indexについては全く同じ.
	\subsection{\texorpdfstring{$S^1$}{S1}上自由粒子での具体例}
	以下では,上で見た$\N=2$超対称代数を,具体的に$S^1$上の自由粒子の例について見る.


	前回,この系は$H\coloneqq -1/(2m)\dv*[2]{}{x}$, $Q\coloneqq -\i/(2m)\dv*{}{x}$, $(-1)^{F}\coloneqq \mathcal{P}$が最小超対称関係を満たすことを見たが,もう少し具体的に見る.

	この系では,周期的境界条件から,エネルギーは$E_n = n^2/(2mR^2),\ n\in\Z_{\geq0}$で対応する固有状態は二つあり,
	\begin{align}
		\psi_{n, +}(x) &= N_{n, +}\cos(nx/R)\quad n\in\Z_{\geq 0}\\
		\psi_{n, -}(x) &= N_{n, -}\sin(nx/R)\quad n \in \Z_{>0}
	\end{align}
	であるが,これは,$Q$が一階微分演算子であり,
	$\sin$と$\cos$が入れ替わることを考えると,$Q$を通して移り合う.

	また,規格化定数のphaseを$N_{n, -} = \i N_{n, +} \eqqcolon \i N_n$ととると,
	\begin{align}
		Q\psi_{n, +} &= \sqrt{n^2/(2mR^2)}\i N_n\sin(nx/R)\\
					 &= \sqrt{E_n}\psi_{n, -}
	\end{align}
	と規格化定数も含めて,一般の結果が再現できる.

	次に,複素superchargeを用いて考える.以下では,具体的な微分などは考えず,
	$H$と$\Q$と$\mathcal{P}$の代数的性質を用いて考察する.

	今,$\P_{\pm}\coloneqq (1 \pm \P)/2$を定めると,
	複素superchargeは$\Q = (Q + \i(-\i Q \mathcal{P}))/\sqrt{2}=\sqrt{2}Q/\P_+$, $\Q^{\dag} = \sqrt{2}Q\P_-$となる.
	
	\begin{thm}
		$\P_{+}$, $\P_{-}$はそれぞれ関数$f(x)$の偶関数成分,奇関数成分を取り出す演算子である.
	\end{thm}
	$f(x) = (f(x) + f(-x))/2 + (f(x) - f(-x))/2$と書き直したとき,第一項は偶関数であり,
	第二項は奇関数である.これらを偶関数成分,奇関数成分と呼ぶ.

	$\P_{\pm}(f(x)) = (f(x) \pm f(-x))/2$は実際に成り立つ.\qed
	\begin{thm}
		$\P_{\pm}$は射影演算子である.すなわち,次の性質を満たす.
		\begin{itemize}
			\item $(\P_{\pm})^2 = \P_{\pm}$
			\item $\P_+ + \P_- = 1$
			\item $\P_+\P_- = 0$
		\end{itemize}
	\end{thm}

	これを用いて,Eq. \eqref{eq:positive_spectrum}のスペクトラムは理解できる.
	\begin{itemize}
		\item $\psi_{n, +}$は偶関数なので,$\Q^{\dag}\sim\P_-$で消える.
		\item $\psi_{n, -}$は奇関数なので,$\Q\sim\P_+$で消える.
		\item $\psi_{n, +}$は偶関数なので,$\Q\sim Q$で奇関数に移る.
		\item $\psi_{n, -}$は奇関数なので,$\Q^{\dag}\sim Q$で偶関数に移る.
	\end{itemize}

	また, Eq. \eqref{eq:zero_spectrum}のスペクトラムは次のように理解できる.
	\begin{itemize}
		\item ground stateは定数関数なので,微分で消える.
		\item 偶関数でもあるので,$\P_-$でも消える.
	\end{itemize}
\end{document}
