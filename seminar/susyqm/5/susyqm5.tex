\documentclass[english, dvipdfmx, a4paper]{jsarticle}
\usepackage{docmute}
\usepackage[utf8]{inputenc}
\usepackage[top=10truemm, bottom=20truemm, left=15truemm, right=15truemm]{geometry} % mergin
\renewcommand{\headfont}{\bfseries}

% graphics
\usepackage{graphicx}
\usepackage{here}

% link

\usepackage{url}
\usepackage[dvipdfmx, linktocpage]{hyperref} 
\usepackage{xcolor}
\usepackage{pxjahyper}
\hypersetup{
	colorlinks=true,
	citecolor=blue,
	linkcolor=teal,
	urlcolor=orange,
}

% math

\usepackage{amsmath, amssymb} 
\usepackage{physics}
\usepackage{mathrsfs}
\usepackage{mathtools}

% theoremstyle
\usepackage{amsthm}
\newtheoremstyle{break}
{\topsep}{\topsep}%
{}{}%
{\bfseries}{}%
{\newline}{}%
\theoremstyle{break}
\newtheorem{thm}{Theorem}
\newtheorem{defn}[thm]{Definition}
\newtheorem{eg}[thm]{Example}
\newtheorem{cl}[thm]{Claim}
\newtheorem{cor}[thm]{Corollary}
\newtheorem{fact}[thm]{Fact}
\newtheorem{rem}[thm]{Remark}
\newtheorem{prob}{Problem}
\newtheorem{prop}{Property}

\makeatletter
\newenvironment{pr}[1][\proofnam]{\par
\topsep6\p@\@plus6\p@ \trivlist
\item[\hskip\labelsep{\itshape #1}\@addpunct{\bfseries}]\ignorespaces
}{%
\endtrivlist
}
\newcommand{\proofnam}{\underline{Derivation.}}
\makeatother
% ordinary
	\newcommand{\R}{\mathbb{R}}
	\newcommand{\C}{\mathbb{C}}
	\newcommand{\Z}{\mathbb{Z}}
	
	
	% Physics %%%%%%%%%%%%%%%%%%%%%%%%%%%%%%%%%%%%
	
	% Feynman slash
	\newcommand{\slashed}[1]{#1\!\!\!/}
	% order
	
	%Lie algebra
	\renewcommand{\O}{\mathcal{O}}
	\newcommand{\SO}{\mathrm{SO}}
	\newcommand{\so}{\mathfrak{so}}
	\newcommand{\SU}{\mathrm{SU}}
	\newcommand{\su}{\mathfrak{su}}
	\newcommand{\SP}{\mathrm{SP}}
	\renewcommand{\sp}{\mathfrak{sp}}
	\newcommand{\SL}{\mathrm{SL}}
	\renewcommand{\sl}{\mathfrak{sl}}
	\newcommand{\GL}{\mathrm{GL}}
	\newcommand{\gl}{\mathfrak{gl}}
	\newcommand{\U}{\mathrm{U}}
	\renewcommand{\u}{\mathfrak{u}}
\renewcommand{\i}{\mathrm{i}}
\newcommand{\e}{\mathrm{e}}
\newcommand{\N}{\mathcal{N}}
\newcommand{\Q}{\mathcal{Q}}
\renewcommand{\P}{\mathcal{P}}
\renewcommand{\S}{\text{S}}
% number

\renewcommand{\thefootnote}{\roman{footnote}.}
\renewcommand{\appendixname}{Appendix }

\title{SUSYQM}
\author{Toshiya Tanaka}
\date{\today}

\begin{document}
	可解な量子力学系があったとき,ground stateから$W'(x)$を構成し,超対称パートナーとして可解系を見つけること,さらに一般化して,パラメータ$\mathcal{E}$として$H = A^{\dagger}A + \mathcal{E}$に対する超対称パートナーを見つけ,可解な量子力学系のファミリーを作る方法を見た.
	
	前者の構成法に対する具体例を見る.
	\begin{eg}
		ポテンシャルを
		\begin{equation}
			V_{(1)} (x) = 
			\begin{cases}
				-\displaystyle\frac{\pi^2\hbar^2}{2mL^2}, \quad &(0 < x < L)\\
				\infty, &(x > 0,\ L < x)
			\end{cases}
		\end{equation}
		で与える.
		Ground stateは$\psi_{0, +} (x) = \sqrt{2/L}\sin(\pi x/L)$であり,第$n$励起状態は$\psi_{n, +}(x) = \sqrt{2/L}\sin((n+1)\pi x/L)$でエネルギーは$E_{n} = n(n+2)\pi^2\hbar^2/(2mL^2)$である.

		今,基底状態を用いて,$W'_{(1)}(x) = -\hbar \psi_0'(x)/\psi_0(x) = -\pi\hbar/L\cot(\pi x/L) $と定めると,ポテンシャルは
		\begin{align}
			V_{(1)}(x) &= -\frac{\pi^2\hbar^2}{2mL^2}\\
			V_{(2)} (x) &= \frac{\pi^2\hbar^2}{2mL^2}\qty(\frac{2}{\sin^2(\pi x/L)} - 1)
		\end{align}
		となり,これはWitten modelのところで見たものと同じである.

		これら二つのポテンシャルからなる系のエネルギーは超対称性から同じスペクトラムになり,
		二つ目の系の固有状態は既知である$\psi_{n, (1)}(x)$を用いて,
		$\psi_{n, (2)} (x) = A\psi_{n, (1)}(x)$で与えられる.ここで,$A$はintertwinerで
		\begin{equation}
			A = \frac{-\i\hbar}{\sqrt{2m}}\qty(\dv{}{x}+\frac{1}{\hbar}W'(x))
			 = \frac{-\i\hbar}{\sqrt{2m}}\qty(\dv{}{x} - \frac{\pi}{L}\cot\qty(\frac{\pi x}{L}))
		\end{equation}
		である.具体的に幾つか計算し,$H_{(2)}$の固有状態になっていて,固有値は$H_{(1)}$に等しいことが確認できる.
	\end{eg}

	\subsection{形状不変性}
	SUSYでは,片方の系がexactly solvableであるとき,各エネルギー固有値,固有状態にパートナーを対応付けてexactly solvableにすることが出来た.
	つまり,この手法は片方の系がexactly solvableでないと使えないが,
	形状不変性(Shape invariance: SI)と組み合わせることで,zero energy stateさえわかっていれば二つの系がexactly solvableであることが言える.

	\begin{defn}[Shape invariance]
		系のパラメータ$a$を明示し,超対称パートナーHamiltonian $H_{\pm}(a)$があったとする.これらに形状不変性があるとは
		\begin{equation}
			H_{-}(a) = H_+(a_1) + \varepsilon(a_1), \quad a_1 = f(a)
		\end{equation}
		なる関係があることをいう.

		言葉でいうと,パラメータを$f$で変換して,定数$\varepsilon(a_1)$の違いでパートナーに一致するときに,形状不変性があるという.
	\end{defn}
	\begin{eg}
		定義域$-\pi/2 < x < \pi/2$でsuperpotentialを$W(x; a) = \hbar a \log(\cos x)$で与える.このとき,ポテンシャルは
		\begin{equation}
			V_{\pm} (x; a) = \frac{\hbar^2}{2m} \qty(\frac{a(a\mp1)}{\cos^2x} - a^2)
		\end{equation}
		で,これは,$a\mapsto a+1$とすると$V_-(x; a) = V_+(x; a+1) + (a+1)^2-a^2$となる.つまり,$a_1 = a+1$, $\varepsilon(a_1) = a_1^2 - (a_1-1)^2$の形状不変性がある.

		この状況は,up to 定数シフトで$H_-(a)$と$H_+(a+1)$が同じHamiltonianで,エネルギー固有状態は$\psi_{n, +}(x; a) = \psi_{n, +}(x; a+1),\ (n\geq0)$であり,
		エネルギー固有値は$E_{n+1, -}(a) = E_{n, +}(a+1) + \varepsilon(a+1)$である.
		$H_+(a+1)$のgroundstate $(n=0)$と$H_-(a)$のgroundstate $(n=1)$が対応するので,labelがずれている.

		これで,$H_{\pm}(a)$の全てのエネルギー固有状態及びエネルギー固有値がわかることになる.
		まず,形状不変性から$\psi_{1, -}(x; a) = \psi_{0, +}(x; a+1)\propto \cos^{a+1}(x)$となり,SUSYのintertwinerを用いて,$\psi_{1, +}(x; a) = A^{\dagger}(a)\psi_{1, -}(x; a)$となる.更に形状不変性を使うと$\psi_{2, -}(x; a) = \psi_{1, +}(x; a+1) = A^{\dagger}(a+1)\cos^{a+1}(x)$と順次求めることができる.

		energyに関しても,
		SUSYより$E_{n, +}(a) = E_{n, -}$があるので,
		$E_{1, -}(a) = E_{0, +}(a+1) + \hbar^2((a+1)^2-a^2)/(2m) = \hbar^2((a+1)^2-a^2)/(2m) = E_{1, +}(a)$で,$E_{2, +}(a) = E_{1, +}(a+1) + \hbar^2((a+1)^2-a^2)/(2m) = \hbar^2((a+2)^2-a^2)/(2m) = E_{2, +}(a)$などと求められる.
		\qed
	\end{eg}

	同様の手法で一般論を作れる.
	\begin{thm}
		$H_+(a)$のエネルギー固有値とエネルギー固有状態は$\psi_{0, +}(x; a)$から次のようにして作れる.

		\begin{align}
			\psi_{n, +}(x; a) &= \prod_{k=1}^n\frac{1}{\sqrt{E_k(a_{n-k})}}A^{\dagger}(a)A^{\dagger}(a_1)\cdots A^{\dagger}(a_{n-1})\psi_{0, +}(x; a)\\
			E_n(a) &= \sum_{k=1}^{n}\varepsilon(a_k)
		\end{align}
		ここで,$a_n\coloneqq f(a_{n-1},\ a_0\coloneqq a)$で$n\geq 0$である.
	\end{thm}

	\begin{eg}[調和振動子]
		\begin{equation}
			H_{\pm} (\omega) =- \frac{\hbar^2}{2m} \dv[2]{}{x} + \frac{1}{2}m\omega^2x^2\mp\frac{\hbar\omega}{2}
		\end{equation}
		で,ゼロエネルギー解は$\psi_{0, +}(x; \omega) = (m\omega/\pi \hbar)^{1/4}\e^{-m\omega x^2/(2\hbar)}$である.形状不変性は$H_-(\omega) = H_+(\omega) + \hbar\omega$で,$\varepsilon(\omega) = \hbar\omega$である.

		これより,エネルギー固有値は$E_n(\omega) = \sum_{k=1}^{n}\varepsilon(\omega) = n\hbar\omega$で,固有状態はHermite多項式
		\begin{equation}
			H_n(q) = (-1)^n\e^{q^2/2}\qty(\dv{}{q} - q)^n\e^{-q^2/2}
		\end{equation}
		を使って,
		\begin{align}
			\psi_{n, +}(x; \omega) &= \prod_{k=1}^{n}\frac{1}{\sqrt{E_k}}(A^{\dagger})^n\psi_{0, +}(x; \omega)\\
								   &= \frac{1}{\sqrt{(\hbar\omega)^n}}\qty(\frac{-\i\hbar}{\sqrt{2m}}\qty(\dv{}{x} - \frac{m\omega}{\hbar}x))^n\qty(\frac{m\omega}{\pi\hbar})^{1/4}\e^{-m\omega x^2/2\hbar}\\
								   &= \frac{\i^n}{2^n\sqrt{n!}}\qty(\frac{m\omega}{\pi\hbar})^{1/4}\e^{-m\omega x^2/2\hbar}H_n\qty(\sqrt{\frac{m\omega}{\hbar}}x)
		\end{align}
		となる.
		\qed
	\end{eg}
	\begin{eg}
		superpotentialを$W(x; a, b) = -\hbar \log(\cos^ax + \sin^bx)$で与える.ポテンシャルは
		\begin{equation}
			V_{\pm}(x; a, b) = \frac{\hbar^2}{2m}\qty(\frac{a(a\mp1)}{\cos^2x} + \frac{b(b\mp1)}{\sin^2x} - (a+b)^2)
		\end{equation}
		であり,形状不変性は$a_1=a+1$, $b_1 = b+1$で$\varepsilon(a_1, b_1) = 2\hbar^2(a_1+b_1-1)/m$となる.
		\qed
	\end{eg}
	超対称性と形状不変性を用いた可解な量子力学系については,\href{https://arxiv.org/abs/hep-th/9405029}{arXiv:hep-th/9405029}が有名である.
\end{document}
