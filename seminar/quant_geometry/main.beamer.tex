\documentclass[dvipdfmx]{beamer}

\usepackage{pxjahyper}
\setlength{\parindent}{11pt}

% theme
\usefonttheme{professionalfonts}
\usetheme{Madrid}
\usecolortheme{default}
\setbeamertemplate{enumerate item}[default]
\setbeamertemplate{caption}[numbered]
\setbeamertemplate{navigation symbols}{}
\usepackage[scaled]{helvet}
\renewcommand{\sfdefault}{phv}
\renewcommand{\familydefault}{\sfdefault}
\renewcommand{\kanjifamilydefault}{\gtdefault}
\setbeamertemplate{theorems}[numbered]
\definecolor{Crimson}{HTML}{dc143c}
\setbeamercolor{alerted text}{fg=Crimson}

% graphics
\usepackage{graphicx}
\usepackage{here}

% math
\usepackage{amsmath, amssymb}
\usepackage{physics}
\usepackage{mathrsfs}
\usepackage{mathtools}
\mathtoolsset{showonlyrefs=true}
\usepackage{tikz}
\usepackage[all]{xy}
\mathversion{bold}
\definecolor{math}{HTML}{100080}
\everymath{\color{math}}

% theoremstyle
\usepackage{amsthm}
\newtheoremstyle{break}
{\topsep}{\topsep}%
{}{}%
{\bfseries}{}%
{\newline}{}%
\theoremstyle{break}
\newtheorem{thm}{Theorem}
\newtheorem{defn}[thm]{Definition}
\newtheorem{eg}[thm]{Example}
\newtheorem{cl}[thm]{Claim}
\newtheorem{cor}[thm]{Corollary}
\newtheorem{rem}[thm]{Remark}
\newtheorem{ax}[thm]{Axiom}
\newtheorem{prob}{Problem}

\makeatletter
\newenvironment{pr}[1][\proofnam]{\par
\topsep6\p@\@plus6\p@ \trivlist
\item[\hskip\labelsep{\itshape #1}\@addpunct{\bfseries}]\ignorespaces
}{%
\endtrivlist
}
\newcommand{\proofnam}{\underline{Derivation.}}
\makeatother



% link

\usepackage{url}
\usepackage{hyperref} 
\usepackage{xcolor}
\usepackage{pxjahyper}
\definecolor{link}{HTML}{4b0082}
\hypersetup{
	colorlinks=true,
	citecolor=link,
	linkcolor=link,
	urlcolor=orange,
}

% my command
\newcommand{\hilb}{\mathcal{H}}
\newcommand{\R}{\mathbb{R}}
\newcommand{\C}{\mathbb{C}}
\newcommand{\Z}{\mathbb{Z}}
\renewcommand{\L}{\mathcal{L}}


% title
\author{谷村省吾}
\institute[名大情報]{名古屋大学情報学研究科}
\date[April 30, 2022]{2022年4月28, 29日}
%\author[田中]{田中寿弥}
%\institute[富大理物]{富山大学理学部物理学科}
\title[\textcolor{white}{保存量の一般化}]{
保存量の一般化}
\subtitle{関数から微分形式へ}
\begin{document}
\begin{frame}
				「量子と古典の物理と幾何」研究会\ @オンライン
		\maketitle
\end{frame}

\begin{frame}{微分方程式で定められる力学系}
		\begin{itemize}
				\item 力学変数 
						\begin{align}
								\vec{x}\colon \R\ni t \mapsto \vec{x}(t) = \qty(x^1(t),x^2(t),  \ldots, x^n(t)) \in \R^n
						\end{align}
				\item ベクトル場 
						\begin{align}
								\vec{V}\colon \R^n \ni \vec{x}\mapsto \vec{V}(\vec{x}) = \qty(V^1(\vec{x}), V^2(\vec{x}), \ldots, V^n(\vec{x}))\in \R^n 
						\end{align}
				\item 運動方程式 
						\begin{align}
								\dv{\vec{x}}{t} = \vec{V}(\vec{x}(t))
						\end{align}
				\item 初期値問題の解(flow)
						\begin{align}
								\vec{x}(t) &= \vec{F}(\vec{x}(0), t)\\
										   &= \phi_t(\vec{x}(0))
						\end{align}
		\end{itemize}
\end{frame}

\begin{frame}
		\begin{itemize}
				\item 関数(物理量,observable) $A\colon \R^n \ni x \mapsto A(x)\in \R $
				\item $A $の時間変化
						\begin{align}
								\dv{}{t}A(x(t)) &= \sum_{i=1}^n\pdv{A}{x^i}\dv{x^i}{t}\\
												&= \sum_i\pdv{A}{x^i}V^i\\
												&= \vec{V}\cdot \operatorname{grad}A\\
												&= \langle \dd{A}, \vec{V}\rangle\\
												&= i_V\dd{A}
						\end{align}
				\item 保存量 $\displaystyle\dv{A}{t} = 0 $
		\end{itemize}
\end{frame}

\begin{frame}
		\begin{itemize}
				\item 保存量$A $があると,解軌道は保存量の定値面に限定される.
				\item $A $の値$c $の等値集合$A^{-1}(c) = \qty{\vec{x}\mid A(\vec{x}) = c} $
				\item 独立な保存量$A_1, A_2, \ldots, A_k $があれば,解軌道の存在領域を$(n-k) $次元に下げられる.
				\item $k = n-1 $個の保存量が見つかれば1次元的な解軌道が決まる.\alert{超可積分}
		\end{itemize}
\end{frame}

\begin{frame}{多様体上の力学系}
		\begin{itemize}
				\item $M\colon $ $n $次元多様体,$x\in M $
				\item $\vec{V}\colon $$M $上のベクトル場
				\item 運動方程式
						\begin{align}
								\dv{x}{t} = \vec{V}(x)
						\end{align}
				\item 解としてのフロー $x(t) = \phi_t(x(0)) $
				\item $\omega\colon $ $M $上の$p $-form
				\item $\phi_t^{*}\omega\colon $ $\phi_t $による$\omega $のpullback
				\item Lie微分(ベクトル場$\vec{V})$に沿った$\omega $の時間変化)
						\begin{align}
								\left.\dv{}{t}\phi_t^{*}\omega\right|_{t=0} \eqqcolon \L _V\omega
						\end{align}
		\end{itemize}
\end{frame}

\begin{frame}
		\begin{eg}
				$1 $-form $\omega =\omega_i\dd{x^i}$, $\vec{V} = V^{k}\pdv{}{x^{k}} $であれば,
				\begin{align}
						\L_{\vec{V}}\omega = V^k\pdv{\omega_i}{x^k}\dd{x^i} + \omega_k\pdv{V^k}{x^i}\dd{x^i}
				\end{align}
		\end{eg}
\end{frame}

\begin{frame}{保存微分形式}
		\begin{itemize}
				\item $\L_{\vec{V}}\omega = 0 $を満たす$\omega  $のこと
		\end{itemize}
		\begin{align}
				 &\L_{\vec{V}}\omega = 0\\
				\Leftrightarrow\quad &\phi_t^{*} \omega = \omega \\
				\Leftrightarrow\quad & \int_{\alert{C(t)}}\omega  = \int_{\phi_tC(0)}\omega = \int_{C(0)}\phi_t^{*}\omega = \int_{\alert{C(0)}}\omega
		\end{align}
		\begin{itemize}
				\item $C(t) $はこの意味で保存量
				\item 特に$p=0 $の場合が,$A(x(t)) = A(x(0)) $
		\end{itemize}
\end{frame}

\begin{frame}{cohomology保存量}
		\begin{itemize}
				\item $p $-form $\omega $がclosed($\dd{\omega}=0)$)かつ$\L_{\vec{{V}}}\omega = \dd{\alpha} $なる$(p-1) $-form $\alpha $が存在.
				\item この場合,$\phi_t^{*}\omega - \omega = \dd{\beta} $なる$(p-1) $-form $\beta $が存在.
		\end{itemize}
		\begin{align}
				\int_{C(t)}\omega &= \int_{\phi_t C(0)}\omega \\
								  &= \int_{C(0)}\phi_t^{*}\omega\\
								  &= \int_{C(0)}(\omega + \dd{\beta})\\
								  &= \int_{C(0)}\omega + \int_{\partial C(0)}\beta\\
								  &= \int_{C(0)} +0.
		\end{align}
\end{frame}

\begin{frame}
			\begin{block}{保存微分形式}
					$\L_{\vec{V}}\omega = 0 $, $\phi_t^{*} \omega = \omega$
			\end{block}
			\begin{block}{積分保存量,保存積分}
					\begin{align}
							\int_{C(t)}\omega = \int_{C(0)}\omega
					\end{align}
					特に$0 $-form $A $の場合$A(x(t)) = A(x(0)) $
			\end{block}
			\begin{itemize}
					\item 一本の解軌道に関する保存量ではなく,無数の解軌道族に対する保存量になっている.

					\item ただ,このような保存量があったからといって解軌道の絞り込み・決定に役立つか?

					\item レベルセットみたいなものはないか?

					\item cotangent bundle にはあると言えるか?
			\end{itemize}
\end{frame}


\begin{frame}{Hamilton力学系の場合}
		\begin{itemize}
				\item $M\colon $ $2n $次元manifold
				\item $\omega\colon $ symplectic-form

						$M $上での$2 $-formで$\dd{\omega} = 0 $ non degenerate
				\item $H\colon M\to \R $ Hamiltonian
				\item $\vec{V}_H\colon $ Hamilton vector field
				\begin{align}
						\vec{V}_H = \sum_i\qty(\pdv{H}{p_i}\pdv{}{q^i} - \pdv{H}{q^i}\pdv{}{p_i})
				\end{align}
		\end{itemize}
\end{frame}


\begin{frame}{symplectic form $\color{white}{\omega}$由来の保存量$\color{white}{Q = \displaystyle\int_C\omega \wedge \cdots\wedge\omega} $}
		$Q $のよいところ.
		\begin{itemize}
				\item Hamiltonianがexplicitにtime-dependent $H(q, p, \lambda(t)) $であっても保存量であり続けること.
				\item 外場や制御変数$\lambda(t) $を時間変化させるシステムといえば,
		\end{itemize}
		\begin{center}
		\alert{熱力学系(熱機関)だ!}
		\end{center}
\end{frame}

\begin{frame}{純粋力学の立場から見た熱力学}
		\begin{align}
				\xymatrix{
						& \overset{\textcolor{black}{\text{少自由度,シンプル}}}{\text{work reservor, 変数$\vec{w}$}}\\
						\text{object system, 変数$\vec{S}$}\ar[rd]^{\text{エネルギーのやり取り}}\ar[ru] _{\text{相互作用}}& \\
															& \underset{\color{black}{\text{大自由度,複雑,力学的挙動がよくわからない}}}{\text{heat reservor, 変数$\vec{h}$}}
				}
		\end{align}
\end{frame}


\begin{frame}{課題}
		\begin{enumerate}
				\item symplectic form以外の保存微分形式を持つ力学系
		\end{enumerate}
\end{frame}
\end{document}

