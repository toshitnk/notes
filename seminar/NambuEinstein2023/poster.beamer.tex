\documentclass[dvipdfmx]{beamer}
\usepackage[orientation=portrait, size=a1, scale=1.7]{beamerposter}
\usepackage{pxjahyper}
\setlength{\parindent}{11pt}
\usepackage{ascmac}
% theme
\usefonttheme{professionalfonts}
\usetheme{Madrid}
\usecolortheme{default}
\setbeamertemplate{enumerate item}[default]
\setbeamertemplate{caption}[numbered]
\setbeamertemplate{navigation symbols}{}
\usepackage[scaled]{helvet}
\renewcommand{\sfdefault}{phv}
\renewcommand{\familydefault}{\sfdefault}
\renewcommand{\kanjifamilydefault}{\gtdefault}
\setbeamertemplate{theorems}[numbered]
\definecolor{Crimson}{HTML}{dc143c}
\setbeamercolor{alerted text}{fg=Crimson}

% 二段組
\usepackage{multicol}

% graphics
\usepackage{graphicx}
\usepackage{here}

% math
\usepackage{amsmath, amssymb}
\usepackage{physics}
\usepackage{mathrsfs}
\usepackage{mathtools}
\usepackage{stmaryrd}
%\usepackage{tikz}
%\usepackage[all]{xy}




% link

\usepackage{url}
\usepackage{hyperref} 
\usepackage{xcolor}
\usepackage{pxjahyper}
\definecolor{link}{HTML}{4b0082}
\hypersetup{
	colorlinks=true,
	citecolor=link,
	linkcolor=link,
	urlcolor=orange,
}

% my command
\newcommand{\hilb}{\mathcal{H}}
\newcommand{\R}{\mathbb{R}}
\newcommand{\C}{\mathbb{C}}
\newcommand{\Z}{\mathbb{Z}}

\definecolor{myblue}{HTML}{00008b}
\setbeamercolor{headline}{fg = white, bg = myblue}
\setbeamertemplate{headline}{
		\leavevmode
		\begin{beamercolorbox}[wd=\paperwidth]{headline}
		\centering
		\vskip4ex
		\usebeamercolor{title in headline}{
				\color{fg}\textbf{\LARGE{\inserttitle}}\\[2.5ex]
		}
		\usebeamercolor{author in headline}{
				\color{fg}\Large{\insertauthor}
		}\\[1.5ex]
		\usebeamercolor{institute in headline}{
				\color{fg}\large{\insertinstitute}
		}

		\vskip4ex
		\end{beamercolorbox}
}
\setbeamertemplate{block begin}{%
    \vskip.5ex
	\begin{beamercolorbox}[
		leftskip=0.5em,
        colsep*=.75ex
		]{block title}%
		\usebeamerfont*{block title}\insertblocktitle
    \end{beamercolorbox}%
    \begin{beamercolorbox}[
        leftskip=-0.5em,
        colsep*=.75ex,
        sep=1ex,
        vmode
		]{block body}%
}
\setbeamertemplate{block end}{%
    \end{beamercolorbox}
    \vskip1.5ex
}
\mathversion{bold}
%\definecolor{math}{HTML}{100080}
%\everymath{\color{math}}


% my defs
\newcommand{\N}{\mathcal{N}}
\newcommand{\Ztt}{\mathbb{Z}_2^2}
\renewcommand{\i}{\mathrm{i}}
\newcommand{\Q}{\mathcal{Q}}
\newcommand{\g}{\mathfrak{g}}
\renewcommand{\acomm}[2]{\mathbf{\{} #1, #2 \mathbf{\}}}
\renewcommand{\comm}[2]{\mathbf{[} #1, #2 \mathbf{]}}
\newcommand{\zcomm}[2]{\mathbf{\llbracket} #1, #2 \mathbf{\rrbracket}}
% title
\author{\inst{1}}
\institute{\inst{1}大阪公理}
\title[\textcolor{white}{}]{超場形式による$\textcolor{white}{\mathbf{\Ztt}}$超対称な古典力学}
\begin{document}
\begin{frame}
\begin{columns}[t]
		\begin{column}{0.47\textwidth}
				\begin{block}{$\Ztt$超対称性について}
							系のHamiltonian $H$,操作を指定する演算子を$\mathcal{O}$.

							量子力学において,操作$\mathcal{O}$に対して系が対称性を持つとは,$[H, \mathcal{O}] = 0$が成り立つこと.
					\begin{itembox}[l]{通常の対称性の例}
							回転を指定する演算子は角運動量演算子$J_i$で.
							\[\comm{J_i}{J_j} = \i\epsilon_{ijk}J_k\]
							を満たす.

							回転対称性を持つ系では$[H, J_i] = 0$が成り立つ.
					\end{itembox}
					通常の対称性は演算子の\underline{交換子}により定まる.このような構造を\underline{Lie代数}という.
					\begin{itembox}[l]{超対称性}
						bosonとfermionを入れ替える操作を指定する演算子は$\Q$, $\Q^\dag$ (superchargeという)で,
						\[\acomm{\Q}{\Q} =2 H\]

						を満たす.

						超対称性を持つ系では$[H, \Q] = [H, \Q^\dag] = 0$が成り立つ.
					\end{itembox}
					超対称性を考えるには\underline{\alert{反}交換子}が必要.

					交換子と反交換子を統一的に記述するためにLie代数を$\Z_2$で次数付する.演算子の空間が$\g = \g_0 \oplus \g_1$と直和分解されていて,$H\in \g_0$, $\Q\in \g_1$であると考える.
					$\zcomm{A}{B}\coloneqq AB - (-1)^{\deg A\deg B}BA$
				\end{block}

				\begin{block}{超場形式}
					超対称性の古典論を考えるのに有効な方法として,超場形式が知られている.

					\begin{table}{h}
						\centering
						\begin{tabular}{ccc}
							通常の場 & 時空$(t, x)$ & 	
							超場形式
						\end{tabular}
					\end{table}
					$\theta$は$\theta^2 = 0$を満たす古典量(Glassmann数という)で,$(t, \theta)$が張る空間を超空間,その上の関数$\Phi(t, \theta)$を超場という.


				\end{block}


				\begin{block}{HOGE}
				\end{block}
		\end{column}
		\begin{column}{0.47\textwidth}
				\begin{block}{PIYO}
				\end{block}
		\end{column}
\end{columns}
\end{frame}
\end{document}


