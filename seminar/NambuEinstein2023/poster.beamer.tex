\documentclass[dvipdfmx]{beamer}
\usepackage[orientation=portrait, size=a1, scale=1.5]{beamerposter}
\usepackage{pxjahyper}
\setlength{\parindent}{11pt}
\usepackage{ascmac}
% theme
\usefonttheme{professionalfonts}
\usetheme{Madrid}
\usecolortheme{default}
\setbeamertemplate{enumerate item}[default]
\setbeamertemplate{caption}[numbered]
\setbeamertemplate{navigation symbols}{}
\usepackage[scaled]{helvet}
\renewcommand{\sfdefault}{phv}
%\renewcommand{\baselinestretch}{1.2}
\renewcommand{\familydefault}{\sfdefault}
\renewcommand{\kanjifamilydefault}{\gtdefault}
\setbeamertemplate{theorems}[numbered]
\definecolor{Crimson}{HTML}{dc143c}
\setbeamercolor{alerted text}{fg=Crimson}

% 二段組
\usepackage{multicol}

% graphics
\usepackage{graphicx}
\usepackage{here}

% math
\usepackage{amsmath, amssymb}
\usepackage{physics}
\usepackage{mathrsfs}
\usepackage{mathtools}
\usepackage{stmaryrd}
%\usepackage{tikz}
%\usepackage[all]{xy}




% link

\usepackage{url}
\usepackage{hyperref} 
\usepackage{xcolor}
\usepackage{pxjahyper}
\definecolor{link}{HTML}{4b0082}
\hypersetup{
	colorlinks=true,
	citecolor=link,
	linkcolor=link,
	urlcolor=white,
}

% my command
\newcommand{\hilb}{\mathcal{H}}
\newcommand{\R}{\mathbb{R}}
\newcommand{\C}{\mathbb{C}}
\newcommand{\Z}{\mathbb{Z}}

\definecolor{myblue}{HTML}{00008b}
\setbeamercolor{headline}{fg = white, bg = myblue}
\setbeamertemplate{headline}{
		\leavevmode
		\begin{beamercolorbox}[wd=\paperwidth]{headline}
		\centering
		\vskip4ex
		\usebeamercolor{title in headline}{
				\color{fg}\textbf{\LARGE{\inserttitle}}\\[2.5ex]
		}
		\usebeamercolor{author in headline}{
				\color{fg}\Large{\insertauthor}
		}\\[1.5ex]
		\usebeamercolor{institute in headline}{
				\color{fg}\large{\insertinstitute}
		}

		\vskip4ex
		\end{beamercolorbox}
}
\setbeamertemplate{block begin}{%
    \vskip.5ex
	\begin{beamercolorbox}[
		leftskip=0.5em,
        colsep*=.75ex
		]{block title}%
		\usebeamerfont*{block title}\insertblocktitle
    \end{beamercolorbox}%
    \begin{beamercolorbox}[
        leftskip=-0.5em,
        colsep*=.75ex,
        sep=1ex,
        vmode
		]{block body}%
}
\setbeamertemplate{block end}{%
    \end{beamercolorbox}
    \vskip1.5ex
}
\mathversion{bold}
%\definecolor{math}{HTML}{100080}
%\everymath{\color{math}}
\usepackage{slashed}


\newcommand{\new}{\vspace{0.5\baselineskip}}
% my defs
\newcommand{\N}{\mathcal{N}}
\newcommand{\Ztt}{\mathbb{Z}_2^2}
\renewcommand{\i}{\mathrm{i}}
\newcommand{\Q}{\mathcal{Q}}
\newcommand{\g}{\mathfrak{g}}
%\renewcommand{\acomm}[2]{\mathbf{\{} #1, #2 \mathbf{\}}}
%\renewcommand{\comm}[2]{\mathbf{[} #1, #2 \mathbf{]}}
\newcommand{\zcomm}[2]{\mathbf{\llbracket} #1, #2 \mathbf{\rrbracket}}
% title
\author{\inst{1}}
\institute{\inst{1}大阪公理}
\title[\textcolor{white}{}]{超場形式による$\textcolor{white}{\mathbf{\Ztt}}$超対称な古典力学}
\begin{document}
\begin{frame}
\begin{columns}[t]
		\begin{column}{0.47\textwidth}
				\begin{block}{$\Ztt$超対称性について}
							系のHamiltonian $H$,操作を指定する演算子を$\mathcal{O}$.

							量子力学において,操作$\mathcal{O}$に対して系が対称性を持つとは,$[H, \mathcal{O}] = 0$が成り立つこと.
					\begin{itembox}[l]{通常の対称性の例}
							回転を指定する演算子は角運動量演算子$J_i$で.
							\[\comm{J_i}{J_j} = \i\epsilon_{ijk}J_k\]
							を満たす.

							回転対称性を持つ系では$[H, J_i] = 0$が成り立つ.
					\end{itembox}
					通常の対称性は演算子の\underline{交換子}により定まる.

					このような構造を\underline{Lie代数}という.
					\begin{itembox}[l]{超対称性(\underline{SU}per \underline{SY}mmetry)}
						bosonとfermionを入れ替える操作を指定する演算子は$\Q$, $\Q^\dag$ (superchargeという)で,
						\[\acomm{\Q}{\Q} =2 H\]

						を満たす.

						超対称性を持つ系では$[H, \Q] = [H, \Q^\dag] = 0$が成り立つ.
					\end{itembox}
					超対称性を考えるには\underline{\alert{反}交換子}が必要.
	\new

					\alert{$\Z_2$ graded} Lie代数を考えることで,交換子と\alert{反}交換子を統一的に記述できる.

					演算子が棲む空間が二種類$\g_0$, $\g_1$あるとして,空間の次数の積により演算を定める.
					\begin{table}[h]
						\centering
						\begin{tabular}{c|cc}\hline
							 & $\g_0$& $\g_1$\\\hline
							$\g_0$ & $\comm{\bullet}{\bullet}$ & $\comm{\bullet}{\bullet}$\\
							$\g_1$ & $\comm{\bullet}{\bullet}$&$\acomm{\bullet}{\bullet}$\\\hline
						\end{tabular}
					\end{table}

					$H\in\g_0$, $\Q, \Q^\dag\in\g_1$とすると,超対称性の代数を満たす.
					\begin{itembox}{$\Ztt$超対称性}
						次数付を$\Ztt\coloneqq \Z_2 \times \Z_2 = {(0, 0), (1, 0), (0, 1), (1, 1)}$に拡張し,交換子を次数の標準内積で定める.
						\begin{table}[h]
							\centering
							\begin{tabular}{c|cccc}\hline
								 & $(0, 0)$ & $(1, 0)$ & $(0, 1)$ & $(1, 1)$\\\hline
								$(0, 0)$ & $\comm{\bullet}{\bullet}$ & $\comm{\bullet}{\bullet}$& $\comm{\bullet}{\bullet}$ & $\comm{\bullet}{\bullet}$\\
								$(1, 0)$ & $\comm{\bullet}{\bullet}$ & $\acomm{\bullet}{\bullet}$& $\comm{\bullet}{\bullet}$ & $\acomm{\bullet}{\bullet}$\\
								$(0, 1)$ & $\comm{\bullet}{\bullet}$ & $\comm{\bullet}{\bullet}$& $\acomm{\bullet}{\bullet}$ & $\acomm{\bullet}{\bullet}$\\
								$(1, 1)$ & $\comm{\bullet}{\bullet}$ & $\acomm{\bullet}{\bullet}$& $\acomm{\bullet}{\bullet}$ & $\comm{\bullet}{\bullet}$\\\hline
							\end{tabular}
						\end{table}
						\begin{alignat*}{3}
							\acomm{Q_{10}}{Q_{10}} &= \acomm{Q_{01}}{Q_{01}} = 2H, &\quad &\acomm{Q_{10}}{Q_{01}} = \i Z\\
							\acomm{Q_{10}}{Z} &= \acomm{Q_{01}}{Z} = 0, & & \acomm{Z}{Z} = 0
						\end{alignat*}
					\end{itembox}
				\end{block}

				\begin{block}{超場形式}
					超対称性の古典論を考えるのに有効な方法として,超場形式が知られている.

					通常の時空$(t, x)$に追加で$\theta$, $\theta^\dag$軸を加えた空間を\underline{超空間}という.ただし,$\theta^2 = (\theta^\dag) = 0$となるGlassman数.
					\begin{table}[h]
						\centering

						\begin{tabular}{c|cc}
							通常の場 & 時空$(t, x)$ &場$\phi(t, x)$ 	\\\hline
							超場形式 & \alert{超}空間$(t, x, \alert{\theta}, \alert{\theta^\dag})$ &\alert{超}場$\Phi(t, x, \alert{\theta}, \alert{\theta^\dag})$
						\end{tabular}
					\end{table}
					超場はBosonとFermionを自動的に含む.
					\[\Phi(t, \theta, \theta^\dag) = q(t) + \theta\eta^{\dag} - \theta^{\dag}\eta + \theta\theta^{\dag}F(t)\]
					$q$, $F$: Boson, $\eta$, $\eta^\dag$ : Fermion.
					超対称不変なactionは超場の超空間上での積分から次のように構成できる.
					\begin{align*}
					S &= \int\dd{t}\dd{\theta}\dd{\theta^\dag} (-\frac{1}{2}(D\Phi)(D^\dag\Phi) + W(\Phi))\\
					&= \int\dd{t}\qty(\frac{1}{2}\dot{q}^2 + \frac{1}{2}F^2 + W'(q)F + \i \eta^{\dag}\dot{\eta} + W''(q)\eta^{\dag}\eta)
					\end{align*}

					\vskip\baselineskip
					$\Z_2^2$超対称性についても,このように古典論を構築できないか.
				\end{block}


		\end{column}
		\begin{column}{0.47\textwidth}
		\begin{block}{$\Z_2^2$超空間上の積分 \scriptsize{\href{https://arxiv.org/abs/2305.07836}{[N.Aizawa, R.Ito arXiv:2305.07836]}}}
		\end{block}

		\begin{block}{$\Ztt$超場形式}
		$(t, \theta_{10}, \theta_{01}, z)$による空間を\alert{$\Ztt$}超空間といい,

		その上の関数$\Phi(t, \theta_{10}, \theta_{01}, z)$を$\Ztt$超場という.
			\[S = -\int\dd{t}\dd{z}\dd{\theta_{10}}\dd{\theta_{01}}\frac{z}{\sqrt{y}}(2D_{10}\Phi D_{01}\Phi) + \alpha V(\Phi)\]
			これを上で定義した積分を使い計算をすると,
			\small{
			\begin{align*}
				S &= \frac{1}{2} (\partial_{\mu} \varphi_{00} \partial^{\mu} \varphi_{00} + \partial_{\mu} \varphi_{11} \partial^{\mu} \varphi_{11}) +2 A_{00}^2 + 2 A_{11}^2
	+i \overline{\Psi}_{10}  \slashed{\partial} \Psi_{10} 
	+i \overline{\Psi}_{01}  \slashed{\partial} \Psi_{01}
	\nonumber \\
%	&-2\alpha \left[ A_{11} V_{00} + A_{00} V_{11} + \frac{1}{2}(\overline{\Psi}_{10} \gamma^3 \Psi_{01} - \overline{\Psi}_{01}\gamma^3 \Psi_{10}) \partial_{00} V_{00} 
%	-\frac{i}{2}( \overline{\Psi}_{10} \gamma^3 \Psi_{10} + \overline{\Psi}_{01} \gamma^3 \Psi_{01}) \partial_{00} V_{11}\right.
%	\nonumber \\
	&-2\alpha ( A_{11} V_{00} + A_{00} V_{11} )
	\nonumber \\
	& - \alpha \big[ (\overline{\Psi}_{10} \gamma^3 \Psi_{01} - \overline{\Psi}_{01}\gamma^3 \Psi_{10}) \partial_{00} V_{00}
	-i( \overline{\Psi}_{10} \gamma^3 \Psi_{10} + \overline{\Psi}_{01} \gamma^3 \Psi_{01}) \partial_{00} V_{11} \big]
\end{align*}}
		\end{block}
		\end{column}
\end{columns}
\end{frame}
\end{document}


