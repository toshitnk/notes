\documentclass[dvipdfmx, a4paper]{jsarticle}
\usepackage[utf8]{inputenc}
\usepackage[top=10truemm, bottom=20truemm, left=15truemm, right=15truemm]{geometry} % mergin
\renewcommand{\headfont}{\bfseries}

% graphics
\usepackage{graphicx}
\usepackage{here}

% link

\usepackage{url}
\usepackage[dvipdfmx, linktocpage]{hyperref} 
\usepackage{xcolor}
\usepackage{pxjahyper}
\hypersetup{
	colorlinks=true,
	citecolor=blue,
	linkcolor=teal,
	urlcolor=orange,
}

% math

\usepackage{amsmath, amssymb} 
\usepackage{physics}
\usepackage{mathrsfs}
\usepackage{mathtools}

% theoremstyle
\usepackage{amsthm}
\newtheoremstyle{break}
{\topsep}{\topsep}%
{}{}%
{\bfseries}{}%
{\newline}{}%
\theoremstyle{break}
\newtheorem{thm}{Theorem}[section]
\newtheorem{defn}[thm]{Definition}
\newtheorem{eg}[thm]{Example}
\newtheorem{cl}[thm]{Claim}
\newtheorem{cor}[thm]{Corollary}
\newtheorem{fact}[thm]{Fact}
\newtheorem{rem}[thm]{Remark}
\newtheorem{prob}{Problem}[section]

\makeatletter
\newenvironment{pr}[1][\proofnam]{\par
\topsep6\p@\@plus6\p@ \trivlist
\item[\hskip\labelsep{\itshape #1}\@addpunct{\bfseries}]\ignorespaces
}{%
\endtrivlist
}
\newcommand{\proofnam}{\underline{Derivation.}}
\makeatother


% my command

\newcommand{\R}{\mathbb{R}}
\newcommand{\C}{\mathbb{C}}
\newcommand{\Z}{\mathbb{Z}}

\newcommand{\U}{\mathrm{U}}
\newcommand{\SU}{\mathrm{SU}}
\newcommand{\su}{\mathfrak{su}}
\newcommand{\N}{\mathcal{N}}
\renewcommand{\H}{\mathcal{H}}

\newcommand{\eq}[1]{Eq. \eqref{#1}}
\newcommand{\theorem}[1]{Thm. \ref{#1}}
\newcommand{\definition}[1]{Def. \ref{#1}}
\newcommand{\proposition}[1]{Prop. \ref{#1}}
\newcommand{\example}[1]{e.g.\ref{#1}}
\newcommand{\claim}[1]{Cl. \ref{#1}}
\newcommand{\corolary}[1]{Cor. \ref{#1}}
\newcommand{\remark}[1]{Rem. \ref{#1}}
\newcommand{\problem}[1]{Prob. \ref{#1}}

\renewcommand{\O}{\mathcal{O}}


% number

%\makeatletter
%\@addtoreset{equation}{section}
%\makeatother
%\numberwithin{equation}{section}
%\renewcommand{\thefootnote}{\roman{footnote}.}
%\renewcommand{\appendixname}{Appendix }


%English

\renewcommand{\tablename}{Tab. }
\renewcommand{\figurename}{Fig. }



\title{場の量子論の数学と2次元4次元対応}
\date{\today}

\begin{document}
	\maketitle
	\tableofcontents
	\section{2次元4次元対応とは}
	\begin{itemize}
			\item 「物理」では,二次元の共形場理論,特にLiuville理論と四次元のゲージ理論,特に$\SU(2)$の対応.
			\item 数学では,無限次元代数,Virasoroとインスタントンモジュライ空間の幾何の対応
	\end{itemize}
	2009年に見つかったときにはびっくり.対応の両側とも35年くらい別個に数理物理で研究されていた.\footnote{いろいろprecursorはあったが.}
	なぜ対応があるか.六次元の"$\N =(2, 0)$理論"を$S^4\times C$(Riemann面)で考える.$C$が小さい場合$C$で定まる四次元のQFTがあり$S^4$が小さいと$S^4$で定まる二次元のQFTがある.
	ここでトポロジカルな物理量を考える.トポロジカルというのは$S^4$と$C$のサイズによらず,どちらで計算しても答えがかわらないものである.

	\section{場の量子論とはどんなものか.}

	0+1d QFT = 1d QFTとは普通の量子力学のことである.$\H$をHilbert空間(状態空間)とし,この上の作用素$A_1, A_2, \ldots$を考える.この中で時間発展を決める特別なものがあり,それをHamiltonianといい$H$と書くことにする.
	一次元時空の各点にHilbert空間があり,$t_1$の時間発展を$e^{-t_1H}$が指定する.

	閉じた1d多様体に対しては二点A, Bがあると
	\begin{align}
			\tr_{\H} e^{-t_1 H }A e^{-t_2 H}B\in\C
	\end{align}

	という複素数を対応させる.

	開いた1d多様体に対しては線型写像
	\begin{align}
			e^{-tH}\colon \H\to \H
	\end{align}
	を対応させる.

	これらは空間の切り貼りに対してcompatibleである.

	一般化すると,(D-1)+1d QFT = D-d QFT $Q$とは
	境界なしD-d多様体$M$を与えられると分配関数$Z_{Q}(M)\in\C$を出力し,さらにいろいろと条件を満たすものである.

	場の量子論の作り方は大きく分けて三通りある.
	\begin{itemize}
			\item 公理系を満たすデータを手で与える.\label{given}

					e.g. 自由場の理論,topological QFT
			\item 経路積分を行う.\label{pathint}

					e.g. 4d pure gauge theory
					\begin{align}
							Z_{\text{QFT}, d=4, G} (M) \coloneqq \int_{M\text{上の$G$接続全体}}\exp{-\int \tr \abs{F}^2 \dd{\operatorname{vol}_M}}\qty[\mathcal{D}\operatorname{Vol}]
					\end{align}
					\begin{itemize}
							\item 数学的にきちんと構成し,性質を調べたら賞金一億円
							\item スパコン上に近似して乗せられる.これは実験をよく再現する.
					\end{itemize}
			\item 超弦理論に押し付ける.\label{superstring}

					e.g. 10dのquantum gravity theory
	\end{itemize}

	\section{2d Liouvilleと4d $\N=2$ $\SU(2)$ $w/4$ flavors}
	2d Liuvilleは経路積分による構成(\ref{pathint})で始まったが,結局直接定義する構成(\ref{given})になった2d 共形場理論である.

	$f$をholomorophicな$\C$上の変換として微小変換
	$z \mapsto z' = z + \sum_{n}\epsilon_n z^{n+1}$で与えられる.

	生成子は$\xi_n\coloneqq z^{n+1}\partial_{z}, \bar{\xi}_n \coloneqq \bar{z}^{n+1}\bar{\partial}_z$で交換関係は
	\begin{align}
			[\xi_m, \xi_n] &= (m-n)\xi_{m+n}\\
			[\bar{\xi}_m, \bar{\xi}_n] &= (m-n)\bar{\xi}_{m+n}\\
			[\xi_m, \bar{\xi}_n] &= 0
	\end{align}
	\section{6dからの視点}

	
	\bibliography{books}
	\bibliographystyle{ytamsalpha}
\end{document}
