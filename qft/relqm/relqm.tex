\documentclass[english, dvipdfmx, a4paper]{jsarticle}
\usepackage[utf8]{inputenc}
\usepackage[top=10truemm, bottom=20truemm, left=15truemm, right=15truemm]{geometry} % mergin
\renewcommand{\headfont}{\bfseries}

% graphics
\usepackage{graphicx}
\usepackage{here}

% link

\usepackage{url}
\usepackage[dvipdfmx, linktocpage]{hyperref} 
\usepackage{xcolor}
\usepackage{pxjahyper}
\hypersetup{
	colorlinks=true,
	citecolor=blue,
	linkcolor=teal,
	urlcolor=orange,
}

% math

\usepackage{amsmath, amssymb} 
\usepackage{physics}
\usepackage{mathrsfs}
\usepackage{mathtools}

% theoremstyle
\usepackage{amsthm}
\newtheoremstyle{break}
{\topsep}{\topsep}%
{}{}%
{\bfseries}{}%
{\newline}{}%
\theoremstyle{break}
\newtheorem{thm}{Theorem}[section]
\newtheorem{defn}[thm]{Definition}
\newtheorem{eg}[thm]{Example}
\newtheorem{cl}[thm]{Claim}
\newtheorem{cor}[thm]{Corollary}
\newtheorem{fact}[thm]{Fact}
\newtheorem{rem}[thm]{Remark}
\newtheorem{prob}{Problem}[section]

\makeatletter
\newenvironment{pr}[1][\proofnam]{\par
\topsep6\p@\@plus6\p@ \trivlist
\item[\hskip\labelsep{\itshape #1}\@addpunct{\bfseries}]\ignorespaces
}{%
\endtrivlist
}
\newcommand{\proofnam}{\underline{Derivation.}}
\makeatother


% my command

\newcommand{\R}{\mathbb{R}}
\newcommand{\C}{\mathbb{C}}
\newcommand{\Z}{\mathbb{Z}}

\newcommand{\eq}[1]{Eq. \eqref{#1}}
\newcommand{\theorem}[1]{Thm. \ref{#1}}
\newcommand{\definition}[1]{Def. \ref{#1}}
\newcommand{\proposition}[1]{Prop. \ref{#1}}
\newcommand{\example}[1]{e.g.\ref{#1}}
\newcommand{\claim}[1]{Cl. \ref{#1}}
\newcommand{\corolary}[1]{Cor. \ref{#1}}
\newcommand{\remark}[1]{Rem. \ref{#1}}
\newcommand{\problem}[1]{Prob. \ref{#1}}
\newcommand{\slashed}[1]{#1\!\!\!/}
\renewcommand{\O}{\mathcal{O}}


% number

%\makeatletter
%\@addtoreset{equation}{section}
%\makeatother
%\numberwithin{equation}{section}
%\renewcommand{\thefootnote}{\roman{footnote}.}
%\renewcommand{\appendixname}{Appendix }

\title{相対論的量子力学}
\author{Toshiya Tanaka}
\date{\today}

\begin{document}
	\maketitle

	\section{Klein--Gordon方程式}
	Klein--Gordon方程式は
	\begin{equation}
			\qty(\partial_\mu\partial^{\mu} + m^2)\phi(t, \vec{x}) = 0\label{eq:klein-gordon}
	\end{equation}
	で,電荷$q $をもつときは共変微分$D_{\mu}\coloneqq \partial_{\mu} + iqA_{\mu} $を用いて
	\begin{equation}
			\qty(D_\mu D^{\mu} + m^2)\phi(t, \vec{x}) = 0\label{eq:covariantKG}
	\end{equation}
	である.

	\subsection{非相対論極限}
	Eq.\eqref{eq:covariantKG}の非相対論極限をとるとSchr\"{o}dinger方程式が出ることを議論する.非相対論極限では静止エネルギーよりもポテンシャルや運動などのエネルギーが小さいことをいう.$m \gg qA^0, \dd{\phi}/\dd{t}. $
	相対論特有の静止エネルギー項を消すため,Klein--Gordon方程式の解$\phi(t, \vec{x}) $とSchr\"{o}dinger方程式の解$\psi(t, \vec{x}) $は$\phi(t, \vec{x}) = e^{-imt}\psi(t, \vec{x}) $と細工をしなければならない.

	$\psi(t, \vec{x}) $をEq. \eqref{eq:covariantKG}に代入して,
	\begin{equation}
			\qty(\pdv{}{t} + iqA^0)^2\phi = \qty((\vec{\nabla} - iq\vec{A})^2 -m^2)\phi
	\end{equation}
	となる.
	左辺を計算すると\footnote{\cite[p.55]{BB17130464}では,関数$f(t) $に対して成り立つ公式
	\begin{equation*}
			\qty(\pdv{}{t})^ne^{-imt}f(t) = e^{-imt}\qty(\pdv{}{t} - im)^n f(t)
	\end{equation*}
	を用いる,と書いてあるが,微分と関数の和の冪$\qty(\pdv*{}{t} + A^0(t))^n $に対しても成り立つのだろうか.
},
	\begin{align}
			\qty(\pdv{}{t} + iqA^0)^2 \phi 
			&= \qty(\pdv{}{t} + iqA^0)e^{-imt}\qty(-im\psi + \dv{\psi}{t} + iqA^0\psi)\\
			&= e^{-imt}\qty(-m^2\psi -2im\pdv{\psi}{t} + \pdv[2]{\psi}{t} + iq\pdv{A^0}{t}\psi + iqA^0\pdv{\psi}{t} + 2mqA^0\psi + iqA^0\pdv{\psi}{t} - q^2\qty(A^0)^2\psi)
	\end{align}
	である.非相対論極限では$m $が支配的で他のエネルギーは小さいので,小さい量が単独で存在する項\footnote{$-2im\pdv*{\psi}{t} $などは$m $がかかっているので残すが,$iq\pdv*{A^0}{t} $は落とす.}を無視して
	\begin{equation}
			\to e^{-imt}\qty(-m^2\psi -2im\dv{\psi}{t} + 2mqA^0\psi  iqA^0\dv{\psi}{t} - q^2\qty(A^0)^2\psi)
	\end{equation}
	となり,
	\begin{equation}
			i\pdv{}{t}\psi = \qty(-\frac{1}{2m}\qty(\vec{\nabla} - iq\vec{A})^2 +qA^0)\psi
	\end{equation}
	のようにSchr\"{o}dinger方程式に帰着する.
	\bibliography{booklist}
	\bibliographystyle{jalpha}
	%\bibliographystyle{ytamsalpha}
	%\bibliographystyle{ytamsbeta}
\end{document}

