\section{回転群について}
\cite{sugiura_yamanouchi}を参考にした.

\begin{prop}
		三次元Euclid空間の直交するベクトル$e_1$, $e_2$, $e_3$を取り,
		各$e_j$が軸となる回転の成す群を$H_j$と書く.

		回転群$\SO(3)$は$H_j$のうち二つで生成される.特に,$H_2$, $H_3$で生成されるとすると,
		任意の$g\in \SO(3)$に対して,$r, t\in H_3$, $s\in H_2$が存在して
		$g = rst$となる.
\end{prop}
\begin{pr}
		ベクトル$ge_3$に対し,$r'\in H_3$で$e_1$, $e_3$の張る面上に移し,
		$s'\in H_2$で$e_3$に戻すことを考えると,$e_3 = s'r'ge_3$である.
		$t\coloneqq s'r'ge_3$とすると,$e_3$を変えないので$t\in H_3$で$s = (s')^{-1}$, $r = (r')^{-1}$とおくと,$g = rst$である.
\end{pr}
