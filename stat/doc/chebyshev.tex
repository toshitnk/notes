\begin{abstract}

  確率を物理に用いることの妥当性について,Chebyshev不等式と大数の法則の
  物理的解釈から議論する.なお,
  本稿の議論は,ほとんど\cite{Tasaki_statmech}の焼き直しである.
\end{abstract}
\section{測定精度と確率}
  \begin{thm}[Chebyshev不等式]
    確率分布$\vec{p}$,物理量$f$とする.
    任意の$\varepsilon>0$に対し,次の不等式が成り立つ.
    \begin{align}
      \mathrm{Prob} \qty(\abs{f-\langle f \rangle_{\vec{p}}} \geq \varepsilon) \leq \qty(\frac{\sigma_{\vec{p}}[f]}{\varepsilon})^2\label{chebyshev}
    \end{align}
    ここで,$\langle\bullet\rangle$は期待値で$\sigma[\bullet]$はゆらぎ(標準偏差)である.
  \end{thm}
  \begin{pr}
    まず,次の量を定める.
    \begin{align}
      \theta \coloneqq 
        \begin{cases}
          1,\quad (\abs{f_i-\langle f \rangle _{\vec{p}}}\geq \varepsilon)\\
          0,\quad (\abs{f_i-\langle f \rangle _{\vec{p}}} < \varepsilon)
        \end{cases}
    \end{align}
    これは,物理量の期待値からのズレが,$\varepsilon$より大きいとき1,
    そうでないとき0を返す関数で,$\theta$の期待値はEq.\eqref{chebyshev}の左辺の確率に等しい.

    ところで,その期待値は
    \begin{align}
      \sum_{i} \theta_ip_i &\leq \sum_{i}\qty(\frac{f_i-\langle f \rangle_{\vec{p}}}{\varepsilon})^2\\
      &=\qty(\frac{\sigma_{\vec{p}}[f]}{\varepsilon})^2
    \end{align}
    と評価でき,不等式が示せる.\qed
  \end{pr}

  Chebyshev不等式\eqref{chebyshev}は次のように解釈することができる.物理量$f$のゆらぎ$\sigma_{\vec{p}}[f]$が
  測定精度より十分小さければ,すなわち$\sigma_{vec{p}}[f]/\varepsilon \ll 1$ならば,
  測定値$f$の期待値$\langle f \rangle_{\vec{p}}$からのずれが有意に現れる\footnote{測定の分解能$\varepsilon$より大きく現れる.}確率は
  a.s.\footnote{almost surely}ゼロである.
\section{試行回数と確率}
  \begin{thm}[大数の法則]
    $f$を物理量,$\vec{p}$を一つの系の確率分布とし,同じ系を$N$個考える.全系の確率分布が$\tilde{\vec{p}}$とする.このとき,任意の$\varepsilon>0$に対し,
    \begin{align}
      \lim_{N\to\infty}\mathrm{Prob}_{\tilde{\vec{p}}}\qty(\abs{\frac{1}{N}\sum_{i=1}^{N}f_i - \langle f \rangle_{\vec{p}}}\geq \varepsilon ) = 0
    \end{align}
    が成り立つ.すなわち,a.s.で平均値が出るということ.
  \end{thm}
  \begin{pr}
    一つの系の物理量$f$の期待値を$\mu$,$f^2$の期待値を$\sigma^2$とする.このとき,ゆらぎは$\sigma = \sqrt{\nu - \mu^2}$である.
    今,平均値という物理量を
    \begin{align}
      m = \frac{1}{N} \sum_{i=1}^{N}f_i
    \end{align}
    とし,この期待値は
    \begin{align}
      \langle m \rangle_{\tilde{\vec{p}}} = \mu
    \end{align}
    $m^2$の期待値は
    \begin{align}
      \langle m^2 \rangle_{\tilde\vec{p}} = \frac{1}{N^2} \sum_{i}\sum_{j}\langle f_if_j\rangle_{(\vec{p}_i,\vec{p}_j)}.
    \end{align}
    これは独立性より,$i\neq j$については
    \begin{align}
      \langle f_if_j \rangle &= \langle f_i \rangle_{\vec{p}} \langle f_j \rangle_{\vec{p}}\\
      &= \mu^2,
    \end{align}
    $i=j$については
    \begin{align}
      \langle f_i f_i \rangle &= \langle f_i^2 \rangle = \nu
    \end{align}
    と計算できて,
    \begin{align}
      \langle m^2 \rangle _{\tilde{\vec{p}}} &= \frac{1}{N^2} \qty(\qty(N^2 - N)\mu^2 + N\nu)
    \end{align}
    となる.\footnote{めんどくて添字抜かしたので,考えて.}全体のゆらぎは
    \begin{align}
      \sigma_{\tilde{\vec{p}}}[f] &= \sqrt{\frac{\nu-\mu^2}{N}}\\
      &= \frac{\sigma_{\vec{p}}[f]}{\sqrt{N}}
    \end{align}
    となる.ここでChebyshev不等式\eqref{chebyshev}を用いると,
    \begin{align}
      \mathrm{Prob}_{\tilde{\vec{p}}}\qty(\abs{\frac{1}{N}\sum_{i=1}^{N}f_i - \langle f \rangle_{\vec{p}}} > \varepsilon) = \qty(\frac{\sigma_{\vec{p}}[f]}{\sqrt{N}}) \underset{N \to \infty}{\to} 0
    \end{align}
    と示せる.\qed
  \end{pr}
  具体例を一つ.
  \begin{eg}
    $N\sim10^{24}$個のサイコロを振ることを考える.測定精度を$\varepsilon\sim10^{-8}$とすると,
    平均値
    \begin{align}
      m\coloneqq \frac{1}{N}\sum_{i=1}^{N}f_i
    \end{align}
    が期待値$3.5000000$に一致しない確率は,$\langle m \rangle = 7/2,\ \langle m^2 \rangle = 35 / (12 N) + (7/2)^2,\ \sigma[m] = \sqrt{35/12N}$なので\footnote{大数の法則の導出と同様にできる.}
    Chebyshev不等式\eqref{chebyshev}より
    \begin{align}
      \mathrm{Prob}(\abs{m-7/2} > 10^{-8}) < \qty(\frac{\sigma}{\varepsilon})\sim \qty(\frac{10^{-12}}{10^{-8}})^{2}= 10^{-8}
    \end{align}
    となる.\qed
  \end{eg}
  今,$\chi$をある事象$A$が起きたとき1,そうでないとき0を返す関数として,
  $A$が起こる確率は,$p = \langle \chi \rangle$である.このとき,大数の法則で
  $f=\chi$と置くと,次が成り立つ.
  \begin{cor}
    $N$回の試行で事象$A$がおこる回数を$N_A$とする.このとき,
    \begin{align}
      \lim_{N\to\infty}\mathrm{Prob}\qty(\abs{\frac{N_A}{N}-p}\geq \varepsilon ) = 0  
    \end{align}
    が成り立つ.
  \end{cor}
  この系は,$N$が大きいとき,$N_A/N$の比という物理的なものが,確率$p$とa.s.で等しいと解釈できる.
\section{結論}
  \begin{itemize}
    \item Chebyshev不等式\eqref{chebyshev}より,測定値が期待値からずれる確率を具体的に評価できる.
    \item 大数の法則およびその系より,たくさんあれば物理量の比と確率がa.s.で等しい.
  \end{itemize}