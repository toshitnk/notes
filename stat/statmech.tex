\documentclass[dvipdfmx,a4paper, english]{jsarticle}
\usepackage[utf8]{inputenc}
\usepackage[dvipdfmx]{hyperref} %リンクを有効にする
%\usepackage{tikz}
%\usepackage[all]{xy}
\usepackage{here}
\usepackage[hang,small,bf]{caption}
\usepackage[subrefformat=parens]{subcaption}
\captionsetup{compatibility=false}
\usepackage{url} %同上
\usepackage{amsmath,amssymb} %数式とか
\usepackage{mathtools} 
\usepackage{braket,physics} %ブラケット
\usepackage{bm} %ベクトル
\usepackage[top=20truemm,bottom=20truemm,left=15truemm,right=15truemm]{geometry} %余白設定
\usepackage{latexsym} 
\usepackage{amsthm}
\newtheoremstyle{break}
{\topsep}{\topsep}%
{}{}%
{\bfseries}{}%
{\newline}{}%
\theoremstyle{break}
\newtheorem{thm}{Theorem}[section]
\newtheorem{defn}[thm]{Definition}
\newtheorem{eg}[thm]{Example}
\newtheorem{cl}[thm]{Claim}
\newtheorem{lem}[thm]{Lemma}
\newtheorem{ax}[thm]{Axiom}
\newtheorem{ppl}[thm]{Principle}
\newtheorem{cor}[thm]{Corollary}
\newtheorem{fact}[thm]{Fact}
\newtheorem{rem}[thm]{Remark}
\makeatletter
\newenvironment{pr}[1][\proofnam]{\par
%\newenvironment{Proof}[1][\Proofname]{\par
\topsep6\p@\@plus6\p@ \trivlist
\item[\hskip\labelsep{\itshape #1}\@addpunct{\bfseries}]\ignorespaces
}{%
\endtrivlist
}
\newcommand{\proofnam}{\underline{Derivation.}}
\makeatother
\usepackage{latexsym} 
\newcommand{\R}{\mathbb{R}}
\newcommand{\Z}{\mathbb{Z}}
\newcommand{\N}{\mathbb{N}}
\newcommand{\C}{\mathbb{C}}
%\newcommand{\D}{\mathcal{D}}
\renewcommand{\O}{\mathcal{O}}
%\newcommand{\SO}{\mathrm{SO}}
%\newcommand{\SU}{\mathrm{SU}}
%\newcommand{\so}{\mathfrak{so}}
%\newcommand{\su}{\mathfrak{su}}
%\newcommand{\GL}{\mathrm{GL}}
%\newcommand{\g}{\mathfrak{g}}
\newcommand{\diff}{\mathrm{d}}
\newcommand{\drv}[2]{\frac{\mathrm{d} #1}{\mathrm{d} #2}} %常微分
\newcommand{\drvn}[3]{\frac{\mathrm{d}^{#1} #2}{\mathrm{d} #3^{#1}}}%高階常微分
\newcommand{\pdrvn}[3]{\frac{\partial^{#1}#2}{\partial #3^{#1}}}%高階偏微分
%\newcommand{\h}{\mathcal{H}}
\usepackage{mathrsfs} %花文字 \mathscr
\renewcommand{\headfont}{\bfseries}
\usepackage{ascmac}
\usepackage{framed,color}%枠等
\definecolor{lightgray}{rgb}{0.75,0.75,0.75}
\usepackage{comment}
\usepackage[dvipdfmx]{color, hyperref}
\usepackage{pxjahyper}
\usepackage{cite}
\usepackage{xcolor}
\hypersetup{
    colorlinks=true,
    citecolor=blue,
    linkcolor=teal,
    urlcolor=orange,
 }
\newcommand{\eq}[1]{Eq. \eqref{#1}}
\newcommand{\theorem}[1]{Thm. \ref{#1}}
\newcommand{\definition}[1]{Def. \ref{#1}}
\newcommand{\proposition}[1]{Prop. \ref{#1}}
\newcommand{\example}[1]{e.g.\ref{#1}}
\newcommand{\claim}[1]{Cl. \ref{#1}}
\newcommand{\corolary}[1]{Cor. \ref{#1}}
\newcommand{\remark}[1]{Rem. \ref{#1}}
\newcommand{\problem}[1]{Prob. \ref{#1}}
\newcommand{\tab}[1]{Table. \ref{#1}}
\renewcommand{\O}{\mathcal{O}}
\newcommand{\mf}{\mathsf{H}}
\usepackage{comment}
\makeatletter
\@addtoreset{equation}{section}
\makeatother
\numberwithin{equation}{section}
\renewcommand{\thefootnote}{\roman{footnote}.}
\renewcommand{\appendixname}{Appendix }

\title{Statistical Mechanics}
\author{Toshiya Tanaka}
\date{\today}

\begin{document}
\maketitle
\section{物理で確率を使うこと}
\begin{abstract}

  確率を物理に用いることの妥当性について,Chebyshev不等式と大数の法則の
  物理的解釈から議論する.なお,
  本稿の議論は,ほとんど\cite{Tasaki_statmech}の焼き直しである.
\end{abstract}
\section{測定精度と確率}
  \begin{thm}[Chebyshev不等式]
    確率分布$\vec{p}$,物理量$f$とする.
    任意の$\varepsilon>0$に対し,次の不等式が成り立つ.
    \begin{align}
      \mathrm{Prob} \qty(\abs{f-\langle f \rangle_{\vec{p}}} \geq \varepsilon) \leq \qty(\frac{\sigma_{\vec{p}}[f]}{\varepsilon})^2\label{chebyshev}
    \end{align}
    ここで,$\langle\bullet\rangle$は期待値で$\sigma[\bullet]$はゆらぎ(標準偏差)である.
  \end{thm}
  \begin{pr}
    まず,次の量を定める.
    \begin{align}
      \theta \coloneqq 
        \begin{cases}
          1,\quad (\abs{f_i-\langle f \rangle _{\vec{p}}}\geq \varepsilon)\\
          0,\quad (\abs{f_i-\langle f \rangle _{\vec{p}}} < \varepsilon)
        \end{cases}
    \end{align}
    これは,物理量の期待値からのズレが,$\varepsilon$より大きいとき1,
    そうでないとき0を返す関数で,$\theta$の期待値はEq.\eqref{chebyshev}の左辺の確率に等しい.

    ところで,その期待値は
    \begin{align}
      \sum_{i} \theta_ip_i &\leq \sum_{i}\qty(\frac{f_i-\langle f \rangle_{\vec{p}}}{\varepsilon})^2\\
      &=\qty(\frac{\sigma_{\vec{p}}[f]}{\varepsilon})^2
    \end{align}
    と評価でき,不等式が示せる.\qed
  \end{pr}

  Chebyshev不等式\eqref{chebyshev}は次のように解釈することができる.物理量$f$のゆらぎ$\sigma_{\vec{p}}[f]$が
  測定精度より十分小さければ,すなわち$\sigma_{vec{p}}[f]/\varepsilon \ll 1$ならば,
  測定値$f$の期待値$\langle f \rangle_{\vec{p}}$からのずれが有意に現れる\footnote{測定の分解能$\varepsilon$より大きく現れる.}確率は
  a.s.\footnote{almost surely}ゼロである.
\section{試行回数と確率}
  \begin{thm}[大数の法則]
    $f$を物理量,$\vec{p}$を一つの系の確率分布とし,同じ系を$N$個考える.全系の確率分布が$\tilde{\vec{p}}$とする.このとき,任意の$\varepsilon>0$に対し,
    \begin{align}
      \lim_{N\to\infty}\mathrm{Prob}_{\tilde{\vec{p}}}\qty(\abs{\frac{1}{N}\sum_{i=1}^{N}f_i - \langle f \rangle_{\vec{p}}}\geq \varepsilon ) = 0
    \end{align}
    が成り立つ.すなわち,a.s.で平均値が出るということ.
  \end{thm}
  \begin{pr}
    一つの系の物理量$f$の期待値を$\mu$,$f^2$の期待値を$\sigma^2$とする.このとき,ゆらぎは$\sigma = \sqrt{\nu - \mu^2}$である.
    今,平均値という物理量を
    \begin{align}
      m = \frac{1}{N} \sum_{i=1}^{N}f_i
    \end{align}
    とし,この期待値は
    \begin{align}
      \langle m \rangle_{\tilde{\vec{p}}} = \mu
    \end{align}
    $m^2$の期待値は
    \begin{align}
      \langle m^2 \rangle_{\tilde\vec{p}} = \frac{1}{N^2} \sum_{i}\sum_{j}\langle f_if_j\rangle_{(\vec{p}_i,\vec{p}_j)}.
    \end{align}
    これは独立性より,$i\neq j$については
    \begin{align}
      \langle f_if_j \rangle &= \langle f_i \rangle_{\vec{p}} \langle f_j \rangle_{\vec{p}}\\
      &= \mu^2,
    \end{align}
    $i=j$については
    \begin{align}
      \langle f_i f_i \rangle &= \langle f_i^2 \rangle = \nu
    \end{align}
    と計算できて,
    \begin{align}
      \langle m^2 \rangle _{\tilde{\vec{p}}} &= \frac{1}{N^2} \qty(\qty(N^2 - N)\mu^2 + N\nu)
    \end{align}
    となる.\footnote{めんどくて添字抜かしたので,考えて.}全体のゆらぎは
    \begin{align}
      \sigma_{\tilde{\vec{p}}}[f] &= \sqrt{\frac{\nu-\mu^2}{N}}\\
      &= \frac{\sigma_{\vec{p}}[f]}{\sqrt{N}}
    \end{align}
    となる.ここでChebyshev不等式\eqref{chebyshev}を用いると,
    \begin{align}
      \mathrm{Prob}_{\tilde{\vec{p}}}\qty(\abs{\frac{1}{N}\sum_{i=1}^{N}f_i - \langle f \rangle_{\vec{p}}} > \varepsilon) = \qty(\frac{\sigma_{\vec{p}}[f]}{\sqrt{N}}) \underset{N \to \infty}{\to} 0
    \end{align}
    と示せる.\qed
  \end{pr}
  具体例を一つ.
  \begin{eg}
    $N\sim10^{24}$個のサイコロを振ることを考える.測定精度を$\varepsilon\sim10^{-8}$とすると,
    平均値
    \begin{align}
      m\coloneqq \frac{1}{N}\sum_{i=1}^{N}f_i
    \end{align}
    が期待値$3.5000000$に一致しない確率は,$\langle m \rangle = 7/2,\ \langle m^2 \rangle = 35 / (12 N) + (7/2)^2,\ \sigma[m] = \sqrt{35/12N}$なので\footnote{大数の法則の導出と同様にできる.}
    Chebyshev不等式\eqref{chebyshev}より
    \begin{align}
      \mathrm{Prob}(\abs{m-7/2} > 10^{-8}) < \qty(\frac{\sigma}{\varepsilon})\sim \qty(\frac{10^{-12}}{10^{-8}})^{2}= 10^{-8}
    \end{align}
    となる.\qed
  \end{eg}
  今,$\chi$をある事象$A$が起きたとき1,そうでないとき0を返す関数として,
  $A$が起こる確率は,$p = \langle \chi \rangle$である.このとき,大数の法則で
  $f=\chi$と置くと,次が成り立つ.
  \begin{cor}
    $N$回の試行で事象$A$がおこる回数を$N_A$とする.このとき,
    \begin{align}
      \lim_{N\to\infty}\mathrm{Prob}\qty(\abs{\frac{N_A}{N}-p}\geq \varepsilon ) = 0  
    \end{align}
    が成り立つ.
  \end{cor}
  この系は,$N$が大きいとき,$N_A/N$の比という物理的なものが,確率$p$とa.s.で等しいと解釈できる.
\section{結論}
  \begin{itemize}
    \item Chebyshev不等式\eqref{chebyshev}より,測定値が期待値からずれる確率を具体的に評価できる.
    \item 大数の法則およびその系より,たくさんあれば物理量の比と確率がa.s.で等しい.
  \end{itemize}
\section{Legendre変換について}
\begin{abstract}
		物理でLegendre変換は解析力学のLagrangianとHamiltonianの関係や熱力学関数たちの関係として与えられる.
		物理では言及なしにある程度連続微分可能性を仮定するが,
		熱力学で相転移を扱うときは熱力学関数の微分微分不可能性が効いてくる.
		多くの教科書では$f^{*}(\alpha) = f(x(\alpha)) - \alpha x(\alpha)$と
		微分を用いた定義をするが,相転移点も含めた熱力学を議論するとき,$\max$, $\min$
		\footnote{厳密には$\sup$, $\inf$で与えるが,存在は保証されるそうである\url{https://mathlog.info/articles/829}.}で与える必要がある\cite[付録H]{Tasaki_thermo}, \cite[Chap.11]{thermo_shimizu}.
\end{abstract}
\section{Legendre変換について}
導出として,凸関数$f$に対して,傾き$\alpha$の線を引いたとき,$f$との交点を
持った状態で直線の切片を最小にすることを考える.
$x =x^{*}$を通る直線を引くとき,直線の方程式は
\begin{align}
		y = \alpha x + f(x^{*}) - \alpha x^{*} 
\end{align}
となり,各$\alpha$に対し,切片の最小値を対応させることを考えると次の定義に至る.
\begin{defn}[Legendre変換]
		凸関数$f\colon\R\to\R$に対し,このLegendre変換$f^{*}$を
		\begin{align}
				f^{*}(\alpha) &= -\min_{x\in\R}\qty(f(x)-\alpha x)\\
				 &= \max_{x\in\R}\qty(\alpha x - f(x))\label{def_legendre}
		\end{align}
		と定める.
\end{defn}
このとき,素直に考えると負符号はつかないように思うが,
凸関数を凸関数に移す対称性を持たせるためにはこれが必要である.
\begin{eg}
		関数
		\begin{align}
				f(x)=
				\begin{dcases}
						\frac{x^3+x}{4} ,& (0\leq x \leq 1)\\
						x-\frac{1}{2} ,& (1 \leq x \leq 2)\\
						\frac{x^2}{2} - x + \frac{3}{2} ,& (2\leq x)
				\end{dcases}
		\end{align}
		のLegendre変換は
		\begin{align}
				f^{*}(p) = 
				\begin{dcases}
						\frac{1}{2}\qty(\frac{4p-1}{3})^{3/2} ,& (1/4 < p < 1)\\
						\frac{1}{2}\qty(p^2 + 2p - 2) ,& (1\leq p)
				\end{dcases}
		\end{align}
		となる.

		\footnote{この例は\url{https://ja.wikipedia.org/wiki/\%E3\%83\%AB\%E3\%82\%B8\%E3\%83\%A3\%E3\%83\%B3\%E3\%83\%89\%E3\%83\%AB\%E5\%A4\%89\%E6\%8F\%9B}による.}
\end{eg}
Figure. \ref{fig:before_legendre}の青い線部分が,
Figure. \ref{fig:after_legendre}の一点に潰れていて,微分不可能点になっている.
熱力学関数にこのようなことが起こるとき,相転移が起こっていることがわかる.
\begin{figure}[htbp]
		\centering
		\begin{minipage}{0.45\linewidth}
				\centering
				\includegraphics[width=7cm]{./doc/img/eg1_legendre.png}
				\caption{Legendre変換前の関数$f$.}
				\label{fig:before_legendre}
		\end{minipage}
		\begin{minipage}{0.45\linewidth}
				\centering
				\includegraphics[width=7cm]{./doc/img/eg2_legendre.png}
				\caption{$f$をLegendre変換した関数$f^{*}$}
				\label{fig:after_legendre}
		\end{minipage}
\end{figure}

\section{磁性体の統計力学}
	\subsection{方針}
	大まかな流れは,次のようである.
	\begin{enumerate}
			\item 分配関数の計算
			\item エネルギー,磁化,磁化率の期待値の計算
			\item 温度,磁場に対する振る舞いを考察
	\end{enumerate}
	この方針は変えず,個々の系に対し様々なテクニックを使う.

	\subsection{すべてのspinが独立にある場合}
	$N$粒子系を考える.
	粒子$j$のspinを$\sigma_j = \pm 1$で指定し,スピン角運動量の固有値は$\pm\mu_0$とする.
	磁場$\mf$中にある系のエネルギー固有値は
	\begin{equation}
			E = -\sum_{j=1}^{N}\mu_0\sigma_j\mf
	\end{equation}
	で,一つの粒子だけに注目したとき
	\begin{align}
			E_j = -\mu_0\sigma_j\mf
	\end{align}
	である.
	spin1つの期待値は期待値の定義から
	\begin{align}
			\expval{\sigma_j} &= \frac{1}{Z_j(\beta)}\qty(\mu_0e^{\beta\mu_0\mf} - \mu_0e^{\beta\mu_0\mf})\\
							  &= \mu_0\tanh(\beta\mu_0\mf)\label{eq:onespin}
	\end{align}
	である.

	独立なので,一粒子の情報がわかれば十分で,一粒子の分配関数は
	\begin{align}
			Z_j(\beta) &= e^{\beta\mu_0\mf} + e^{-\beta\mu_0\mf}\\
				&= 2\cosh(\beta\mu_0\mf)
	\end{align}
	である.よって,$N$粒子あったとき,分配関数は
	\begin{equation}
			Z(\beta) = \qty(2\cosh(\beta\mu_0\mf))
	\end{equation}
	となる.
	エネルギー期待値は
	\begin{align}
			\langle H\rangle &= -\pdv{}{\beta}\log(2\cosh(\beta\mu_0\mf))\\
							 &= -N\mu_0\mf\tanh(\beta\mu_0\mf)
	\end{align}
	である.

	\begin{defn}[磁化]
			磁化を
			\begin{equation}
					m = \frac{1}{N}\sum_{j = 1}^{N}\mu_0\sigma_j
			\end{equation}
			と定める.スピンの平均値と思ってよい.
	\end{defn}
	磁化の期待値は,\eq{eq:onespin}と期待値の線形性から
	\begin{align}
			\expval{m} &= \frac{1}{N}\sum_{j = 1}^{N}\mu_0\expval{\sigma_j}\\
					   &= \mu_0\tanh(\beta\mu_0\mf)
	\end{align}
	である.

	\begin{defn}[$\mf=0$での磁化率]
			磁化率$\chi$を
			\begin{equation}
					\chi = \left.\pdv{m}{\mf}\right|_{\mf=0}
			\end{equation}
			と定める.
			磁場$\mf$を揺すったときの磁石になりやすさと解釈できる.
	\end{defn}

	磁化率も計算する.
	\begin{equation}
			\chi = \frac{\mu_{0}^{2}}{k_{\text{B}}T}\label{eq:single_suscep}
	\end{equation}
	となる.

	本筋とは外れるが,エントロピーを計算する.そのためにまず,Helmholtz free energyの計算をする.
	\begin{align}
			F(\beta, \mf, N)&= -\frac{1}{\beta}\log Z(\beta)\\
							&= -Nk_{\text{B}}T\log\qty(2\cosh\qty(\frac{\mu_0\mf}{k_{\text{B}}T}))
	\end{align}
	ここから,エントロピーが計算できて,
	\begin{align}
			S(\beta, \mf, N) &= -\pdv{}{T}F(\beta, \mf, N)\\
							 &= Nk_{\text{B}}\frac{\mu_0\mf}{k_{\text{B}}T}\qty(\cosh\qty(\frac{\mu_0\mf}{k_{\text{B}}T}) - \log \qty(2\cosh\qty(\frac{\mu_0\mf}{k_{\text{B}}T})))
	\end{align}
	となり,$\mf/k_{\text{B}}$単位で現れる.これを用いて,$(T_1, \mf_1)\rightarrow(T_2, \mf_2)$の断熱準静操作を行うとき,エントロピーが普遍なので,この単位も不変である.磁場$\mf$をゆっくり変えることで温度を変えることが\footnote{主に低温を作る.}できる.これを断熱消磁と呼ぶ.

	\subsection{ペアを形成し,ペア同士は独立な場合}
	次に,二つの粒子が相互作用しペアを形成していて,ペア同士は相互作用していない状況を考える.
	粒子数$N$はわかりやすく.偶数としておく.

	このときのエネルギー固有値は
	\begin{equation}
			E_{\sigma} = -J\sum_{j = 1}^{N/2}\sigma_{2j-1}\sigma_{2j} - \mu_0 \mf \sum_{k=1}^{N}\sigma_k\label{eq:exchange}
	\end{equation}
	とする.第一項が相互作用による項でHeisenberg交換相互作用とよばれる.

	$J$は物質による定数で,$J$が正の場合ペアのスピンは揃いやすく,負の場合,反対を向きやすい.

	以上の設定で考える.

	まず,ペアたちは独立なので,$(\sigma_1, \sigma_2)$だけ選んで考える.
	\eq{eq:exchange}にでてくる量を計算し,\tab{tab:exchange_interaction}
	\begin{table}[tbp]
			\centering
			\label{tab:exchange_interaction}
			\caption{\eq{eq:exchange}のスピンを含む箇所の計算結果.}
			\begin{tabular}{c|cccc}\hline
					$(\sigma_1, \sigma_2)$ & $(1, 1)$ & $(1, -1)$ & $(-1, 1)$ & $(-1, -1)$\\\hline
					$\sigma_1\sigma_2$ & $1$ & $-1$ & $-1$ & $1$\\
					$\sigma_1 + \sigma_2$ & $2$ & $0$ & $0$ & $-2$\\
					$E_{N=2}$ & $-J-2\mu_0\mf$ & $J$ & $J$ & $-J+2\mu_0\mf$\\\hline
			\end{tabular}
	\end{table}

	これで,分配関数が計算できる.

	\begin{align}
			Z(\beta) &= e^{-\beta(-J - 2\mu_0\mf)} + 2e^{-\beta J} + e^{-\beta(-J+2\mu_0\mf)}\\
					 &= 2e^{\beta J}\qty(\cosh(2\beta\mu_0 \mf) + e^{-2\beta J}).
	\end{align}

	エネルギー期待値は
	\begin{align}
			\langle H \rangle &= -\pdv{}{\beta}\log Z(\beta)\\
							  &= -\frac{N}{2}\frac{J(e^{-2\beta J} - \cosh(2\beta\mu_0\mf)) - 2\mu_0\mf\sinh(2\beta\mu_0 \mf)}{\cosh(2\beta\mu_0\mf) + e^{-2\beta J}}
	\end{align}
	となる.ここで,$\mf = 0$とすると,
	\begin{equation}
			\langle H \rangle = -\frac{N}{2}\tanh\beta J
	\end{equation}
	である.

	spinのペアの期待値は
	\begin{align}
			\langle \sigma_1 + \sigma_2\rangle &= \frac{1}{Z(\beta)}\qty(2e^{\beta(J + 2\mu_0\mf)} - e^{-\beta(J-2\mu_0\mf)})\\
										&= \frac{2\sinh(2\beta \mu_0 \mf)}{\cosh(2\beta\mu_0 /mf) + e^{-2\beta J}}
	\end{align}
	磁化は
	\begin{align}
			\langle m \rangle &= \left\langle \frac{1}{N}\sum_{j = 1}^{N}\sigma_j\right\rangle\\
							  &= \frac{1}{N}\sum_{j = 1}^{N/2}\langle \sigma_{2j-1}\sigma_{2j}\rangle\\
							  &= \frac{\sinh(2\beta\mu_0\mf)}{\cosh(2\beta\mu_0\mf) + e^{-2\beta J}}\\
							  &= \frac{2\beta\mu_0^2\mf + \O\qty((\beta\mu_0\mf)^3)}{1 + \O\qty((\beta\mu_0\mf)^2) + e^{-2\beta\mu_0\mf}}
	\end{align}
	となる.最後は磁化率を求めるために$\mf$を含む冪で展開した.

	磁化率は$\mf$で微分して$\mf = 0$とするので,二次以上の項は消えて,
	\begin{align}
			\chi(\beta) &= \left.\pdv{\expval{m}}{\mf}\right|_{\mf=0}\\
					 &= \frac{2\beta\mu_0^2}{1 + e^{2\beta J}}
	\end{align}
	となる.

	\subsection{1-d Ising model}
	エネルギー固有値は
	\begin{equation}
			E_{\sigma} = -J\sum_{\langle i, j \rangle} \sigma_i\sigma_j - \mu_0\mf \sum_{i=1}^{N}\sigma_i
	\end{equation}
	とする.ここで,$\langle i, j\rangle$は隣り合う$i$, $j$に関して和をとることを表し,周期的境界条件$\sigma_{N+1} = \sigma_1$を入れると,
	\begin{equation}
			E_{\sigma} = -J\sum_{i = 1}^{N}\sigma_{i}\sigma_{i+1} - \mu_0\mf\sum_{i=1}^{N}\qty(\frac{\sigma_i+\sigma_{i+1}}{2})
	\end{equation}
	と書き換えることができる.

	分配関数は
	\begin{align}
			Z(\beta) &=\sum_{\sigma} e^{E_{\sigma}}\\
			 &= \sum_{\sigma}e^{\sum_{i=1}^{N}\qty(\beta J\sigma_i\sigma_j + \beta \mu_0\mf (\sigma_i + \sigma_{i+1})/2)}\\
			 &= \sum_{\sigma_1 = \pm 1}\cdots\sum_{\sigma_N=\pm 1}\prod_{i=1}^{N}e^{\beta J\sigma_i\sigma_j + \beta \mu_0\mf (\sigma_i + \sigma_{i+1})/2}\\
			 &= \sum_{\sigma_1 = \pm 1}\cdots\sum_{\sigma_N=\pm 1}\prod_{i=1}^{N}M_{\sigma_i, \sigma_{i+1}}
	\end{align}
	となる.行列$M$は転送行列とよばれ,
	\begin{equation}
			M = \begin{pmatrix}
					e^{\beta J + \beta\mu_0\mf} & e^{-\beta J}\\
					e^{-\beta_j} & e^{\beta J - \beta \mu_0\mf}
			\end{pmatrix}
	\end{equation}
	である.さらに分配関数は計算できて,
	\begin{align}
			Z(\beta) &= \Tr\qty(M^N)\\
					 &= \lambda_{+}^{N} + \lambda_{-}^{N}
	\end{align}
	で,$\lambda_{\pm}$は転送行列の固有値で
	\begin{equation}
			\lambda_{\pm} = e^{\beta J}\cosh\beta\mu_0\mf \pm \sqrt{e^{2\beta J}\cosh^2\beta\mu_0\mf - 2\sinh\beta J}
	\end{equation}
	である.

	これを用いて,各物理量を求める.
	自由エネルギー密度は
	\begin{align}
			f_N(\beta) &= -\frac{1}{\beta N}\log Z(\beta)\\
					   &= -\frac{1}{\beta}\log\lambda_{+} - \frac{1}{\beta N}\log\qty(1 + \frac{\lambda_{-}}{\lambda_{+}})
	\end{align}
	なので,$N\to \infty$で$f(\beta ) = -\log\lambda_{+}/\beta$となる.

	磁化は
	\begin{equation}
			m(\beta, \mf) = \frac{\mu_0\sinh\beta\mu_0\mf}{\sqrt{\sinh^2\beta\mu_0\mf + e^{-4\beta J}}}
	\end{equation}
	で,磁化率は
	\begin{equation}
			\chi(\beta) = \beta \mu_0^2e^{2\beta J}
	\end{equation}
	である.

	これらは連続関数で1-d Ising modelは相転移を起こさないこと\footnote{強いていえば,$\beta\to\infty$で発散するので,絶対零度が相転移点ということもできる. }がわかる.

	\subsection{2-d Ising model}
	二次元では相転移を起こす.しかし,直接解くのは大変である\footnote{2022/04/19時点では私は解けていない.}.

	他の状況でも使える,かなり雑だが有用な近似である平均場近似の方法と相転移温度を正しく導出できるKramers--Wannier双対性を用いた議論を行うことにする.

%	\subsubsection{平均場近似}
	\subsubsection{Kramers--Wannier双対性}
	ここでは見やすさの都合上,分配関数を$\beta J \eqqcolon K$の関数と見て考えることにする.
	また,外部磁場$\mf = 0$とする.

	ひたすら分配関数を書き換えてゆく.まず,
	\begin{equation}
			e^{K\sigma_i \sigma_j} = \cosh K \sum_{t\in\qty{0, 1}}\qty(\sigma_i \sigma_j \tanh K)^t
	\end{equation}
	と書き換える.よくわからないが,正しいことは少し計算すればわかる.これは各スピンのペアに対して定まり,ペアに関して積をとることはそれらを結ぶ辺について積をとることと同じであることに注意すると,次のように書き換えることができる.

	\begin{align}
			Z(K) &= \sum_{\sigma} e^{K\sum_{\langle i, j\rangle}\sigma_i\sigma_j}\\
				 &= \sum_{\sigma} \prod_{\langle i, j \rangle}e^{K\sigma_i \sigma_j}\\
				 &= \qty(\cosh K)^{2N} \sum_{\sigma}\sum_{\qty{t}}\prod_{\text{辺}}\qty(\sigma_i \sigma_j\tanh K)^t
	\end{align}
	$\sum_{\qty{t}}$は,各辺に$0, 1$を割り当てるすべての配位に関する和で,$\prod_{\text{辺}}$は一つ選んだ配位に関して積をとる操作である.
	この積について,$t=0$が与えれられた辺は$1$で寄与する.
	$t = 1$が与えられた配位について考える.先に$\sigma$についての和をとることを考える.これは有限和なので,文句なしに可能である.
	このとき,各$\sigma_i$でくくると,$n\in\Z$とすると
	\begin{align}
			\sum _{\sigma_i = \pm1}\sigma_i^{2n-1} &= 0,\\
			\sum_{\sigma_i = \pm1} \sigma_i^{2n} &= 2
	\end{align}
	であるので,$t=1$が偶数個集まる点からは$2\tanh K$で寄与する.
	模型に対応付けると,$t = 1$の線が一つだけ集まることは線の端があることに対応して,一つでもこれがあると積はゼロになるので,閉曲線になっているところだけが積に寄与することになる.

	結局,分配関数は
	\begin{align}
			Z(K) &= \qty(\cosh K)^{2N}\sum_{\qty{\text{閉曲線}}}\qty(2\tanh K)^{\text{閉曲線の長さ}}\\
				 &= \qty(2\cosh K)^{2N}\sum_{\qty{\text{閉曲線}}}\qty(2\tanh K)^{\text{閉曲線の長さ}}
	\end{align}
	となる.ここで,格子には閉曲線の中と外という二つの状態ができる.これをスピンの上下に対応させることを考える.このスピン配位を$\mu$と書くことにする.

	今,閉曲線に関して和を取っているが,これはスピンの配位$\mu$に関して和を取っているとおもってもよい.
	また,閉曲線周りでは隣接するスピンは逆向きで,$\mu_i \mu_j = -1$になるので,$(1-\mu_i\mu_j)/2$で隣接する$i, j$について和をとると閉曲線で$1$, そうでない場所で$0$と区別ができる.
	さて,分配関数は
	\begin{align}
			Z(K) &= \qty(2\cosh K)^{2N} \sum_{\qty{\mu}} \prod_{\text{辺}}(\tanh K)^{(1-\mu_i\mu_j)/2}\\ 
			&= \qty(2\cosh K)^{2N}(\tanh K)^N\sum_{\mu}\prod_{\langle i, j \rangle}\qty(\tanh K)^{\mu_i\mu_j/2}\\
			&= (\sinh 2K)^{N/2}Z(K')
	\end{align}
	と変形できる.ここで,$e^{K'} = \tanh^{-1/2} K$とおいた.つまり,パラメータ$K$の二次元イジングモデルは$K'$の二次元イジングモデルと関係していることがわかる.この関係をKramers--Wanneir双対性\cite{PhysRev.60.252}という.

	さて,この変換を二回行うと
	\begin{equation}
			Z(K) = (\sinh K \sinh K')^NZ(K)
	\end{equation}
	となるので,$\sinh 2K \sinh 2K' = 1$でないといけない.$\sinh$は単調関数なので,変なことが起こるとすれば$K = K'$と予想できる.
	これを解くと,
	\begin{equation}
			K_{\text{c}} = \frac{1}{2}\log(1 + \sqrt{2})
	\end{equation}
	となる.この解は正しい臨界温度を与えることが知られている.













\bibliography{booklist, condmat, link}
\bibliographystyle{ytamsalpha}

\end{document}
