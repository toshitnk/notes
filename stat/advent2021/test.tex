\documentclass[dvipdfmx,a4paper]{jsarticle}
\usepackage[utf8]{inputenc}
\usepackage[dvipdfmx]{hyperref} %リンクを有効にする
%\usepackage{tikz}
%\usepackage[all]{xy}
\usepackage{here}
\usepackage{url} %同上
\usepackage{amsmath,amssymb} %数式とか
\usepackage{mathtools} 
\usepackage{braket,physics} %ブラケット
\usepackage{bm} %ベクトル
\usepackage[top=20truemm,bottom=20truemm,left=15truemm,right=15truemm]{geometry} %余白設定
\usepackage{latexsym} 
\usepackage{amsthm}
\newtheoremstyle{break}
{\topsep}{\topsep}%
{}{}%
{\bfseries}{}%
{\newline}{}%
\theoremstyle{break}
\newtheorem{thm}{Theorem}[section]
\newtheorem{defn}[thm]{Definition}
\newtheorem{eg}[thm]{Example}
\newtheorem{cl}[thm]{Claim}
\newtheorem{lem}[thm]{Lemma}
\newtheorem{ax}[thm]{Axiom}
\newtheorem{cor}[thm]{Corollary}
\newtheorem{fact}[thm]{Fact}
\newtheorem{rem}[thm]{Remark}
\makeatletter
\newenvironment{pr}[1][\proofnam]{\par
%\newenvironment{Proof}[1][\Proofname]{\par
\topsep6\p@\@plus6\p@ \trivlist
\item[\hskip\labelsep{\itshape #1}\@addpunct{\bfseries}]\ignorespaces
}{%
\endtrivlist
}
\newcommand{\proofnam}{\underline{Derivation.}}
\makeatother
\usepackage{latexsym} 
\newcommand{\R}{\mathbb{R}}
\newcommand{\Z}{\mathbb{Z}}
\newcommand{\N}{\mathbb{N}}
\newcommand{\C}{\mathbb{C}}
%\newcommand{\D}{\mathcal{D}}
\renewcommand{\O}{\mathcal{O}}
%\newcommand{\SO}{\mathrm{SO}}
%\newcommand{\SU}{\mathrm{SU}}
%\newcommand{\so}{\mathfrak{so}}
%\newcommand{\su}{\mathfrak{su}}
%\newcommand{\GL}{\mathrm{GL}}
%\newcommand{\g}{\mathfrak{g}}
\newcommand{\diff}{\mathrm{d}}
\newcommand{\drv}[2]{\frac{\mathrm{d} #1}{\mathrm{d} #2}} %常微分
\newcommand{\drvn}[3]{\frac{\mathrm{d}^{#1} #2}{\mathrm{d} #3^{#1}}}%高階常微分
\newcommand{\pdrvn}[3]{\frac{\partial^{#1}#2}{\partial #3^{#1}}}%高階偏微分
%\newcommand{\h}{\mathcal{H}}
\usepackage{mathrsfs} %花文字 \mathscr
\renewcommand{\headfont}{\bfseries}
\usepackage{ascmac}
\usepackage{framed,color}%枠等
\definecolor{lightgray}{rgb}{0.75,0.75,0.75}
\usepackage{comment}
\usepackage[dvipdfmx]{color, hyperref}
\usepackage{pxjahyper}
\usepackage{cite}
\usepackage{xcolor}
\hypersetup{
    colorlinks=true,
    citecolor=blue,
    linkcolor=teal,
    urlcolor=orange,
 }
\makeatletter
\@addtoreset{equation}{section}
\makeatother
\numberwithin{equation}{section}
\renewcommand{\thefootnote}{\roman{footnote}.}
\renewcommand{\appendixname}{Appendix }
\title{統計力学を始めるモチベーションのために}
\author{わっふる。\footnote{Twitter: \url{https://twitter.com/tos_shiii}}}
\date{\today}

\begin{document}
\maketitle

\begin{abstract}
    このpdfは\href{https://adventar.org/calendars/6268}{統合ゼミコミュニティーアドベントカレンダー2021}の12日目の記事です.
    
    この記事は,\cite{Tasaki_statmech}などの文献で統計力学を勉強し始めたが,面白さがわからない
    と悩む人を第一の対象とし執筆された.統計力学でcanonical分布を考えると,分配関数というものが導出されるが,
    まず分配関数の一般的性質について知られていることをまとめ,それが具体的な系に
    応用される様子を銅の比熱を例にとって議論する.分配関数が使えることがわかったのち,canonical分布の導出を行う.
\end{abstract}
\section{はじめに}
    統計力学の入門書として,\cite{Tasaki_statmech}があげられることが多い気がする.ミーハーな私はもちろんそれを聞いてすぐ二巻とも購入し
    積読した記憶がある.統計力学を勉強するのには,熱力学と量子力学を勉強する必要があると聞いた私は,それを勉強したあと,B2の夏頃だっただろうか,
    すぐに田崎本に取り掛かった.まず,二章に確率の話,三章に量子力学と状態数の勘定についての話.ここまでは,多少の近似の評価に首はかしげつつ問題なく進んだ気がする.
    しかし,よく考えれば,統計力学はやっていないではないか.四章にcanonical分布の導出.近似のオンパレード.分配関数って何だ?統計力学って一体何者?
    このあたりで力尽きた.

    統計力学の面白さを感じ始めたのは,B3前期の物性物理学の授業
    \footnote{ウチのカリキュラムはめちゃくちゃで,量子力学と同時,統計力学の授業が始まる前に物性の授業があって,
    比熱とかを扱う.しかし,本記事で述べるように,先に応用を扱っておくのも悪くないの...か?}
    だった.比熱のEinstein modelやDebye modelを扱って,実験データととても合うということに面白さを感じた.
    統計力学の面白さは,一般論から多種多様な現象が説明できることではないかと思っている.本記事では私の経験から,
    最初に知っていれば退屈することなく一般論を学べたのではないかという統計力学(というか分配関数)の応用例について述べ,
    実際の実験データ\cite{doi:10.1063/1.555728}との比較も行う.

    なお,著者は統計力学を勉強中であり不正確な記述や,誤解している点も多いかと思う.
    そのようなことにお気づきの際は,ご教示いただければ幸いである.
\bibliography{books, link, papers}
\bibliographystyle{ytamsalpha}
\end{document}