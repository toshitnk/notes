\documentclass[dvipdfmx, a4paper]{jsarticle}
\usepackage[utf8]{inputenc}
\usepackage[top=10truemm, bottom=20truemm, left=15truemm, right=15truemm]{geometry} % mergin
\renewcommand{\headfont}{\bfseries}

% graphics
\usepackage{graphicx}
\usepackage{here}

% link

\usepackage{url}
\usepackage[dvipdfmx, linktocpage]{hyperref} 
\usepackage{xcolor}
\usepackage{pxjahyper}
\hypersetup{
	colorlinks=true,
	citecolor=blue,
	linkcolor=teal,
	urlcolor=orange,
}

% math

\usepackage{amsmath, amssymb} 
\usepackage{physics}
\usepackage{mathrsfs}
\usepackage{mathtools}

% theoremstyle
\usepackage{amsthm}
\newtheoremstyle{break}
{\topsep}{\topsep}%
{}{}%
{\bfseries}{}%
{\newline}{}%
\theoremstyle{break}
\newtheorem{thm}{Theorem}[section]
\newtheorem{defn}[thm]{Definition}
\newtheorem{eg}[thm]{Example}
\newtheorem{cl}[thm]{Claim}
\newtheorem{cor}[thm]{Corollary}
\newtheorem{fact}[thm]{Fact}
\newtheorem{rem}[thm]{Remark}
\newtheorem{prob}{Problem}[section]

\makeatletter
\newenvironment{pr}[1][\proofnam]{\par
\topsep6\p@\@plus6\p@ \trivlist
\item[\hskip\labelsep{\itshape #1}\@addpunct{\bfseries}]\ignorespaces
}{%
\endtrivlist
}
\newcommand{\proofnam}{\underline{Derivation.}}
\makeatother


% my command

\newcommand{\R}{\mathbb{R}}
\newcommand{\C}{\mathbb{C}}
\newcommand{\Z}{\mathbb{Z}}

\newcommand{\eq}[1]{Eq. \eqref{#1}}
\newcommand{\theorem}[1]{Thm. \ref{#1}}
\newcommand{\definition}[1]{Def. \ref{#1}}
\newcommand{\proposition}[1]{Prop. \ref{#1}}
\newcommand{\example}[1]{e.g.\ref{#1}}
\newcommand{\claim}[1]{Cl. \ref{#1}}
\newcommand{\corolary}[1]{Cor. \ref{#1}}
\newcommand{\remark}[1]{Rem. \ref{#1}}
\newcommand{\problem}[1]{Prob. \ref{#1}}

\renewcommand{\O}{\mathcal{O}}

\newcommand{\mf}{\mathsf{H}}
% number

%\makeatletter
%\@addtoreset{equation}{section}
%\makeatother
%\numberwithin{equation}{section}
%\renewcommand{\thefootnote}{\roman{footnote}.}
%\renewcommand{\appendixname}{Appendix }


%English

\renewcommand{\tablename}{Tab. }
\renewcommand{\figurename}{Fig. }
\renewcommand{\refname}{References}


\title{磁性体の統計力学}
\author{Toshiya Tanaka}
\date{\today}

\begin{document}
	\maketitle

	\section{方針}
	大まかな流れは,次のようである.
	\begin{enumerate}
			\item 分配関数の計算
			\item エネルギー,磁化,磁化率の期待値の計算
			\item 温度,磁場に対する振る舞いを考察
	\end{enumerate}
	この方針は変えず,個々の系に対し様々なテクニックを使う.

	\section{すべてのspinが独立にある場合}
	$N$粒子系を考える.
	粒子$j$のspinを$\sigma_j = \pm 1$で指定し,スピン角運動量の固有値は$\pm\mu_0$とする.
	磁場$\mf$中にある系のエネルギー固有値は
	\begin{equation}
			E = -\sum_{j=1}^{N}\mu_0\sigma_j\mf
	\end{equation}
	で,一つの粒子だけに注目したとき
	\begin{align}
			E_j = -\mu_0\sigma_j\mf
	\end{align}
	である.
	spin1つの期待値は期待値の定義から
	\begin{align}
			\expval{\sigma_j} &= \frac{1}{Z_j(\beta)}\qty(\mu_0e^{\beta\mu_0\mf} - \mu_0e^{\beta\mu_0\mf})\\
							  &= \mu_0\tanh(\beta\mu_0\mf)\label{eq:onespin}
	\end{align}
	である.

	独立なので,一粒子の情報がわかれば十分で,一粒子の分配関数は
	\begin{align}
			Z_j(\beta) &= e^{\beta\mu_0\mf} + e^{-\beta\mu_0\mf}\\
				&= 2\cosh(\beta\mu_0\mf)
	\end{align}
	である.よって,$N$粒子あったとき,分配関数は
	\begin{equation}
			Z(\beta) = \qty(2\cosh(\beta\mu_0\mf))
	\end{equation}
	となる.
	エネルギー期待値は
	\begin{align}
			\langle H\rangle &= -\pdv{}{\beta}\log(2\cosh(\beta\mu_0\mf))\\
							 &= -N\mu_0\mf\tanh(\beta\mu_0\mf)
	\end{align}
	である.

	\begin{defn}[磁化]
			磁化を
			\begin{equation}
					m = \frac{1}{N}\sum_{j = 1}^{N}\mu_0\sigma_j
			\end{equation}
			と定める.スピンの平均値と思ってよい.
	\end{defn}
	磁化の期待値は,\eq{eq:onespin}と期待値の線形性から
	\begin{align}
			\expval{m} &= \frac{1}{N}\sum_{j = 1}^{N}\mu_0\expval{\sigma_j}\\
					   &= \mu_0\tanh(\beta\mu_0\mf)
	\end{align}
	である.

	\begin{defn}[$\mf=0$での磁化率]
			磁化率$\chi$を
			\begin{equation}
					\chi = \left.\pdv{m}{\mf}\right|_{\mf=0}
			\end{equation}
			と定める.
			磁場$\mf$を揺すったときの磁石になりやすさと解釈できる.
	\end{defn}

	磁化率も計算する.
	\begin{equation}
			\chi &= \frac{\mu_0^2}{k_{\text{B}}T}\label{eq:single_suscep}
	\end{equation}
	となる.

	本筋とは外れるが,エントロピーを計算する.そのためにまず,Helmholtz free energyの計算をする.
	\begin{align}
			F(\beta, \mf, N)&= -\frac{1}{\beta}\log Z(\beta)\\
							&= -Nk_{\text{B}}T\log\qty(2\cosh\qty(\frac{\mu_0\mf}{k_{\text{B}}T}))
	\end{align}
	ここから,エントロピーが計算できて,
	\begin{align}
			S(\beta, \mf, N) &= -\pdv{}{T}F(\beta, \mf, N)\\
							 &= Nk_{\text{B}}\frac{\mu_0\mf}{k_{\text{B}}T}\qty(\cosh\qty(\frac{\mu_0\mf}{k_{\text{B}}T}) - \log \qty(2\cosh\qty(\frac{\mu_0\mf}{k_{\text{B}}T})))
	\end{align}
	となり,$\mf/k_{\text{B}}$単位で現れる.これを用いて,$(T_1, \mf_1)\rightarrow(T_2, H_2)$の断熱準静操作を行うとき,エントロピーが普遍なので,この単位も不変である.磁場$\mf$をゆっくり変えることで温度を変えることが\footnote{主に低温を作る.}できる.これを断熱消磁と呼ぶ.

	\nocite{BA88185786}
	\bibliography{booklist}
	\bibliographystyle{jalpha}
	%\bibliographystyle{ytamsbeta}
\end{document}

